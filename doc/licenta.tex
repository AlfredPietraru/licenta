\documentclass[12pt,a4paper]{report}
\usepackage[utf8]{inputenc} 
\usepackage[romanian]{babel}  
\renewcommand\familydefault{\sfdefault} 
\usepackage[margin=2.54cm]{geometry}	
\usepackage{graphicx} 
\usepackage{amsmath}

\usepackage{textcase}
\usepackage[titletoc, title]{appendix}
\usepackage{titlesec}
\titleformat{\chapter}{\large\bfseries\MakeUppercase}{\thechapter}{2ex}{}[\vspace*{-1.5cm}]
\titleformat*{\section}{\large\bfseries}
\titleformat*{\subsection}{\large\bfseries}
\titleformat*{\subsubsection}{\large\bfseries}

\usepackage{chngcntr}
\counterwithout{figure}{chapter} 
\counterwithout{table}{chapter} 
\counterwithout{equation}{chapter} 

\usepackage{booktabs} 
\usepackage{url} 
\usepackage[bookmarks,unicode,hidelinks]{hyperref}
 
\usepackage{array} 
\usepackage{paralist} 
\usepackage{verbatim} 
\usepackage{subfig} 
\usepackage{enumitem}
\setlist{noitemsep}

\usepackage{fancyhdr}
\pagestyle{empty}
\renewcommand{\headrulewidth}{0pt}
\renewcommand{\footrulewidth}{0pt}
\lhead{}\chead{}\rhead{}
\lfoot{}\cfoot{\thepage}\rfoot{}
\usepackage[backend=biber, style=numeric, sorting=none]{biblatex}
\addbibresource{bibliography.bib}


\newcommand{\HeaderLineSpace}{-0.25cm}
\newcommand{\UniTextRO}{UNIVERSITATEA NATIONALA DE STIINTA SI TEHNOLOGIE POLITEHNICA BUCURESTI \\[\HeaderLineSpace] 
FACULTATEA DE AUTOMATICĂ ȘI CALCULATOARE \\[\HeaderLineSpace]
DEPARTAMENTUL DE CALCULATOARE\\}
\newcommand{\DiplomaRO}{PROIECT DE DIPLOMA}
\newcommand{\AdvisorRO}{Coordonator științific:}
\newcommand{\BucRO}{BUCUREȘTI}
 
\newcommand{\UniTextEN}{NATIONAL UNIVERSITY OF SCIENCE AND TECHNOLOGY POLITEHNICA BUCHAREST \\[\HeaderLineSpace]
FACULTY OF AUTOMATIC CONTROL AND COMPUTERS \\[\HeaderLineSpace]
COMPUTER SCIENCE AND ENGINEERING DEPARTMENT\\}
\newcommand{\DiplomaEN}{DIPLOMA PROJECT}
\newcommand{\AdvisorEN}{Thesis advisor:}
\newcommand{\BucEN}{BUCHAREST}
\newcommand{\ProjectTitleRO}{Orientare in Spatiu folosind ORB-SLAM}
\newcommand{\ProjectTitleEN}{Spatial Orientation using ORB-SLAM}

\newcommand{\frontPage}[6]{
\begin{titlepage}
\begin{center}
{\Large #1}  
\vspace{50pt}
\vspace{105pt}
{\Huge #2}\\  
\vspace{40pt}
{\Large #3}\\ \vspace{0pt}  
{\Large #4}\\     
\vspace{40pt}
{\LARGE Alfred Andrei Pietraru}\\   
\end{center}
\vspace{60pt}
\begin{tabular*}{\textwidth}{@{\extracolsep{\fill}}p{6cm}r}
&{\large\textbf{#5}}\vspace{10pt}\\      
&{\large Prof.\ dr.\ ing.\ Anca Morar}   
\end{tabular*}
\vspace{20pt}
\begin{center}
{\large\textbf{#6}}\\ 
\vspace{0pt}
{\normalsize 2025}
\end{center}
\end{titlepage}
}

\newcommand{\frontPageRO}{\frontPage{\UniTextRO}{\DiplomaRO}{\ProjectTitleRO}{\BucRO}{\AdvisorRO}}
\newcommand{\frontPageEN}{\frontPage{\UniTextEN}{\DiplomaEN}{\ProjectTitleEN}{\BucEN}{\AdvisorEN}}

\linespread{1.5}
\setlength\parindent{0pt}
\setlength\parskip{8pt}
\include{1_abstract}
\begin{document}
\frontPageRO{}
\frontPageEN{}
\begingroup
\linespread{1}
\tableofcontents
\endgroup
\AbstractPage{}
\chapter{Introducere}
\section{Context}
SLAM, Simultanious Localization and Mapping reprezintă o clasă de algoritmi de planificare 
și control a mișcării unui agent prin mediu pentru a construi un model al spațiului cât
mai apropiat de realitate. Aceste clase au câștigat atenția publicului în ultimii ani, lucru care
a condus la dezvoltarea numeroaselor variante prezente la momentul curent pe piață, 
fiecare adaptat pentru mediul și tipul de
senzori folosiți. O atenție deosebită a fost acordată sistemelor de tip Visual SLAM din mai
multe motive: camerele video sunt unul dintre cele mai comune tipuri de senzori, există o 
multitudine de tehnici de Computer Vision pentru procesarea imaginilor, iar filozofia pe 
care acești algoritmi o urmează este asemănătoare cu modul în care creierul uman 
interpretează mediul înconjurator. Astfel, sunt alese un set de puncte din spațiu care vor fi 
considerate referințe, iar unghiul din care acestea sunt observate poate oferi informații
despre poziția agentului în mediu. Astăzi, cei mai populari algoritmi 
de tip Visual SLAM îmbină domenii precum Machine Learning, Computer Vision, Robotică și 
Matematică, pentru a crea sisteme robuste, capabile să îndeplinească o varietate de sarcini.   

\section{Problemă}
Creați un sistem capabil să exploreze un mediu necunoscut din interior. Acesta trebuie să
creeze o hartă a zonei parcurse, să reconstruiască traseul estimând poziția camerei pentru 
fiecare cadru citit, să fie tolerant la erori și să poată opera pentru perioade de timp îndelungate.    
\section{Obiective}
Obiectivele principale ale lucrării sunt:
\begin{itemize}
    \item studierea, configurarea și implementarea unui sistem de tip ORB-SLAM2\cite{orbslam2} 
    \item testarea performanțelor folosind seturi de date consacrate, utilizarea unui video
    realizat de mine pentru etapa de evaluare  
    \item testarea unei implementări în care o cameră tip RGBD să fie înlocuită cu o cameră
    monoculară tradițională și o rețea neurală, această sarcină presupunând alegerea unei
    arhitecturi potrivite și compararea celor două metode
    \item prezentarea problemelor întâlnite în etapa de dezvoltare și a unor direcții de îmbunătățire  
\end{itemize}

\section{Soluția propusă}
Lucrarea propune implementarea algoritmului ORB-SLAM2 adaptat pentru camerele de tip RGBD
și testarea acestuia pe setul de date TUM RGBD\cite{tum}. Se vor analiza aspecte precum acuratețea 
traiectoriei și îndeplinirea condițiilor de funcționare în timp real. De asemenea, se va 
încerca înlocuirea matricei de adâncime creată de camera RGBD, cu o hartă de distanțe 
calculată de către rețeaua neurală FastDepth\cite{fastdepth}.  

\section{Rezultatele obținute}
Rezultatele au arătat că implementarea algoritmului ORB-SLAM2, folosind o cameră tip RGBD 
prezintă rezultate bune în medii indoor și că poate funcționa în timp real cu viteze de aproximativ
10{-}15 cadre pe secundă. Utilizarea unei rețele neurale pentru estimarea distanței nu a 
avut rezultatele dorite, algoritmul fiind capabil să funcționeze pentru cel mult 50 de cadre.  

\section{Structura lucrării}
Lucrarea este structurată în mai multe capitole. Introducerea oferă contextul lucrării și definește problema abordată, obiectivele și soluția propusă. Capitolul 2 descrie 
cerințele funcționale, nonfuncționale și motivația.
Capitolul 3 realizează un studiu de piață asupra metodelor curente, separându-le in trei
categorii. Capitolul 4 descrie la nivel conceptual soluția propusă: algoritmii folosiți și componentele logice.
Capitolul 5 detaliază modul în care este realizată evaluarea și rezultatele obținute.
Ultimul capitol prezintă impresii personale despre acest proiect, lucruri pe 
care le-aș putea îmbunătăți, problemele întâlnite și direcțiile viitoare. 

\chapter{Cerințe și Motivație}
\section{Motivație}
Filmele, cărțile și jocurile pe calculator ne arată un viitor al omenirii în care
roboți inteligenți îndeplinesc sarcini complexe, sunt capabili să discute cu noi și 
să se adapteze mediului înconjurător. Deși în momentul de față suntem departe de a
crea un framework suficient de complex pentru un asemenea agent, consider că algoritmii din categoria 
Visual SLAM sunt un pas spre această direcție. Îmi este greu să îmi imaginez un robot care
să poată simula compartamentul uman și să nu fie capabil să se deplaseze și să înțeleagă mediul
în care se află. Pentru noi aceste lucruri sunt adânc înrădăcinate în modul în care funcționeaza creierul, dar pentru
un calculator, a fost nevoie de aproape 30 de ani de cercetare pentru a crea algoritmi 
suficient de complecși pentru a îndeplini sarcini minimale de orientare, cum ar fi 
capacitatea de învățare a mediului și de poziționare a agentului în spațiu. Prima
dată, conceptul de SLAM a fost definit în anul 1995 în această lucrare\cite{durrant1996slam}, 
iar de atunci a avut parte de o dezvoltare continuă. În ziua de azi, această categorie 
de algoritmi are numeroase aplicații practice: 
\begin{itemize}
    \item realizarea sarcinilor din viața de zi cu zi: cazul roboților de curățenie
sau a celor care transportă obiecte în interiorul unei clădiri 
    \item în aplicații medicale, ca de exemplu asistența pentru persoanele nevăzătoare
    \item în aplicații militare: cartografierea zonelor necunoscute
    \item în interiorul clădirilor sau în medii ostile unde nu este acces la un sistem de coordonate globale
cum ar fi poziția dată de un sistem GPS  
    \item în aplicații industriale, inspecții asupra instalației sau depozitelor, 
detectarea unor erori și raportarea zonei în care au fost observate    
\end{itemize}
Există numeroase aplicații pentru sistemele SLAM, întrucât toate pleacă de la faptul 
că agentul trebuie să creeze o hartă a mediului și să înțeleagă care este poziția acestuia.
Pe măsură ce aceste sisteme vor evolua, va crește și complexitatea sarcinilor pe 
care le pot îndeplini.   

\section{Cerințe Funcționale și Nonfuncționale}
Din punct de vedere al cerințelor funcționale, algoritmul ORB-SLAM va primi 
un video realizat cu o cameră de tip RGBD și va returna 2 fișiere text. Unul va conține
estimarea poziției pentru fiecare cadru în parte, iar celălalt va avea salvată harta 
mediului înconjurator, alcătuită dintr-un nor de puncte în spațiu și cadrele cheie
asociate acestora. Algoritmul va avea o interfață grafică minimală, reprezentată din 2 
ferestre. Cea din stânga va conține o reprezentare pentru cadrul curent procesat. 
Acesta va avea culoarea albastru, iar celelalte cadre cheie salvate în harta vor avea cu 
verde și cu roșu punctele din spațiu. Toate acestea vor alcătui harta mediului
înconjurător. În fereastra din dreapta va fi afișat fiecare cadru în format alb negru,
iar cu roșu vor fi marcate feature-urile detectate de algoritmul ORB.\@ Cadrele cheie 
consecutive sunt conectate între ele prin intermediul unei drepte de culoare neagră.
Aceste segmente concatenate vor genera traseul realizat de camera în video. 
\begin{figure}[htbp] 
  \centering
  \includegraphics[width=1.0\textwidth]{./images/interfata_grafica.png}
  \caption{Interfața grafică, stânga reprezentarea hărții, dreapta extragere feature-uri cu ORB}
  \label{fig:exemplu_imagine}
\end{figure}  \\         
Ca cerințe nonfuncționale, algoritmul trebuie să meargă în timp real, să 
proceseze între 10{-}15 cadre pe secundă și să poată fi folosit în sisteme embedded
în care unitatea centrală de procesare are cel mult 2 core-uri și nu poate folosi GPU-ul.
Sistemul trebuie să fie
rezistent la erorile de estimare pentru fiecare imagine primită, să optimizeze atât
harta, cât și traseul realizat și să aibă capacitatea de relocalizare în situația în care
urmărirea cadru cu cadru eșuează. Mediul în care poate să opereze este unul static, de
mici dimensiuni și nu poate accesa coordonatele globale ale poziției sale prin intermediul
tehnologiilor precum GPS.\@

\chapter{Studiu de piață}
\section{State of the art Visual SLAM}
Cei mai noi algoritmi de Visual SLAM funcționează acum folosind tehnici
de deep learning. În continuare vor fi prezentate lucrările care au reprezentat SOA, până la începutul 
anului 2025, grupate pe categorii în funcție de modul în care sunt folosite tehnicile de deep learning.
Am considerat potrivită împărțirea algoritmilor pe 3 nivele, în funcție de gradul de utilizare 
a tehnicilor de deep learning pentru realizarea operațiilor specifice sistemelor SLAM:\@
\begin{enumerate}
    \item Algoritmi care se bazează fundamental pe tehnici de deep learning pentru a funcționa, 
DPV-SLAM\cite{lipson2024deep}, ESLAM\cite{Wei2024RealTimeDV}.\@
    \item Algoritmi care sunt la granița dintre metodele clasice și cele deep learning, în care 
doar anumite componente sunt îmbunătățite cu ajutorul rețelelor neurale: Light-SLAM\cite{Zhao2024LightSLAMAR}, HFNet-SLAM\cite{Liu2023HFNetSLAMAA}.\@
    \item Algoritmii clasici care nu folosesc deloc rețele neurale: ORB-SLAM3\cite{9440682}, SVO\cite{Forster2014SVOFS}.\@ 
\end{enumerate}   
Deep Patch Visual SLAM (DPV-SLAM), este un sistem SLAM care folosește deep neural networks.
Acesta împarte operațiile care trebuie realizate în 2 categorii: frontend-ul care realizează 
sarcina de visual odometry cu ajutorul unui sistem derivat din Deep Patch Visual Odometry (DPVO)\cite{Teed2022DeepPV}
și partea de backend alcătuită din 2 metode de loop closure: proximity loop closure și classical
loop closure. Algoritmul are nevoie între 5{-}6 GB de memorie pe GPU pentru a putea rula. 
O altă problemă o reprezintă proximity loop closure. Aceasta funcționează cu ajutorul unei hărți
foarte dense, de feature-uri obținute cu ajutorul metodei de optical flow, fiind imposibil de folosit 
în timp real fără a folosi GPU.\@ \\

ESLAM sau Efficient Dense Visual SLAM using Neural Implicit Maps este un sistem de SLAM 
monocameră RGB-D care folosește o combinație între o hartă densă 3D, reprezentată de o rețea 
neurală implicită și un backend optimizat geometric pentru estimarea matricei de poziție a camerei.
Avantajele acestei implementări sunt faptul că produce o hartă densă și detaliată, poate reconstrui 
detalii chiar și în zone parțial observate. Problema acestei implementări este că necesită un 
GPU și resurse mari de calcul, nefiind potrivită pentru dispozitivele embedded. \\


Implementarea Light-SLAM folosește ORB-SLAM2 la bază. Partea de backend,
alcătuită din local mapping și loop closure, folosește în continuare metode clasice. Local mapping
este responsabil de optimizarea hărții, iar loop closure de recunoașterea
zonelor prin care a mai trecut algoritmul și de închiderea buclelor traiectoriei. Acestea sunt 
realizate folosind metode clasice. Extragerea de keypoint-uri, descriptori și sarcina de 
matching între descriptorii a două imagini consecutive este realizată de 2 rețele neurale.
Acest sistem poate funcționa în timp real dacă se poate folosi un GPU, dar cea mai mare 
problemă o reprezintă faptul că rețelele neurale nu sunt capabile să găsească feature-uri cu
acuratețe suficient de bună în zone care nu seamănă cu ceea ce au întâlnit în setul de date
 de antrenare. Astfel, algoritmul nu are garanția că va funcționa în situații critice.\\

HFNet-SLAM este o metodă construită pe baza ORB-SLAM3 și se folosește de arhitectura HF-Net, 
având straturile de convoluție separate în depthwise convolution și pointwise convolution, 
asemănător cu modul în care este gândit Mobile\_Net. În loc să folosească 2 rețele neurale, 
precum Light-SLAM, aceasta folosește una singură, atât pentru extragerea keypoint-urilor și a
descriptorilor, cât și pentru feature-urile globale, folosite în sarcinile de loop closure. 
Pe lângă problema rețelei neurale care trebuie să ruleze pe GPU și a feature-urilor instabile 
extrase din imagini pentru zone care nu au fost întâlnite în setul de antrenare, algoritmul
calculează pentru fiecare cadru în parte feature-urile sale globale, lucru care adaugă un 
overhead computațional inutil. De asemenea, rețeaua neurală nu extrage keypoint-urile
pe mai multe nivele, astfel, obținându-se prea puține puncte pentru a menține sistemul stabil 
în mediile slab texturate. \\

ORB-SLAM3 este continuarea implementării algoritmului ORB-SLAM2 pe care l-am ales eu.
Acesta a apărut în 2021 și până în acest moment este cea mai complexă și completă metodă de a
estima traiectoria camerei și de a reconstrui o hartă de puncte a mediului înconjurător folosind 
doar metode clasice. În comparație cu precedesorul acestuia, implementarea de ORB-SLAM3 folosește
datele obținute de la Inertial Measurement Unit (IMU) și optimizează rezultatele primite 
folosind tehnica de Maximum a Posteriori Estimation (MAP). Față de versiunea anterioară a algoritmului, cea
curentă lucrează
 cu multiple hărți locale, respectiv noruri
de puncte în spațiu. În momentul în care ORB-SLAM3 pierde orientarea și trebuie să execute 
o relocalizare, sistemul generează o nouă hartă pentru a menține fluidă procesarea cadru 
cu cadru. În situația în care tracker-ul recunoaște zona în care a ajuns, acesta încearcă să unească 
hărțile între ele pentru a reconstrui întreg mediul. Am considerat că zona pe care o parcurge 
agentul nostru este de mici dimensiuni și nu ar avea nevoie de un sistem foarte de 
complex de interconectare a hărților generate, iar utlizarea acestuia ar adăuga un overhead
nejustificat. În plus, nu avem acces la datele ce aparțin componentei IMU.\@ \\

SVO sau Semi-Direct Visual Odometry for Monocular and Multi-Camera Systems este un exemplu 
de algoritm de tip SLAM care folosește doar 2 thread-uri: unul responsabil de 
urmărirea cadru cu cadru și unul pentru optimizarea hărții. Acesta folosește gradienții
pixelilor în imagini pentru a crea feature-uri, nu doar contururile obiectelor. În cazul
algoritmilor din familia ORB-SLAM care încearcă să optimizeze eroarea de proiecție a punctelor
din spațiu, aici se folosesc metode directe și trebuie minimizată eroarea fotometrică a 
pixelilor aflați în apropierea contururilor obiectelor. Este printre cei mai rapizi algoritmi
de SLAM, procesând peste 100 de cadre pe secundă pe un CPU. Totuși, acesta generează o hartă densă, dar cu 
puncte de slabă calitate care nu pot fi refolosite, nu există capacitate de relocalizare și 
este dificil de extins. Așadar, acesta ar putea fi considerat a doua cea mai bună opțiune după ORB-SLAM2.\\

În ciuda faptului că algoritmul ORB-SLAM2 a aparut în 2017, acesta rămâne în continuare 
un exemplu de sistem bine gândit, cu multe posibilități de extindere și capacitatea de a fi 
adaptat la cerințele din zilele noastre. Îndeplinește cu succes toate cerințele funcționale 
și nonfuncționale pe care sistemul ar trebui să le aibă: poate fi folosit în real time,
implementarea procesând aproximativ 15 cadre pe secundă, creează o hartă a mediului înconjurător
pe care o poate optimiza, are capacitate de relocalizare și corectează erorile de estimare
care apar în timp prin mecanismul de loop closure. Acesta poate rula exclusiv pe CPU, fiind
potrivit atât pentru vehicule la sol, cât și pentru drone. Nu are nevoie de o estimare a poziției
globale, putând fi folosit în medii ostile unde terenul este complet necunoscut.  

% \section{State of the art Retea Neurala}
% de adaugat mai multe arhitecturi de retea neurala, de ales care este cel mai potrivit


\chapter{Soluție propusă}
Soluția mea propune implementarea algoritmului ORB-SLAM2. Acesta are 2 scopuri 
fundamentale:
\begin{itemize}
    \item să estimeze pentru fiecare cadru matricea de poziție și orientarea
     camerei, reconstruind astfel traseul parcurs în timpul funcționării algoritmului  
    \item să creeze o hartă locală a mediului înconjurător pentru a memora zonele
prin care a mai trecut și pentru a îmbunătăți estimarea traiectoriei  
\end{itemize}
Matricea de poziție și orientare a camerei (pose matrix) are dimensiuni $ 4 \times 4 $ și are 
formatul prezentat mai jos, unde R reprezintă matricea de rotație $ 3 \times 3 $, iar t este
vectorul coloană de dimensiune 3, reprezentând translația față de punctul de origine
\( (0, 0, 0)\). Aceasta mai este denumită și matricea de conversie din sistemul de 
coordonate global (world space) în sistemul de coordonate al camerei (camera space), fiind
 notată în implementarea mea cu \( T_{cw} \). Inversa acestei matrice
  realizează operația de conversie dintre cele 2 sisteme de coordonate 
în sens opus.

\begin{equation}
T_{cw} = 
\begin{bmatrix}
R & t \\
0 & 1
\end{bmatrix}, \quad{}
T_{wc} = 
\begin{bmatrix}
R & -R^{t}t \\
0 & 1
\end{bmatrix}
\end{equation}

Algoritmul primește ca date de intrare sursa de la care se vor obține imaginile de tip
RGB, matricile de adancime pentru fiecare cadru, parametrii de distorsiune si matricea parametrilor interni ai camerei, 
având dimensiunea $ 3 \times 3 $, notată în mod tradițional cu \(K\). Această matrice trebuie modificată de 
fiecare dată când este schimbată camera cu care se realizează filmarea. Dacă se execută operații de modificare
a dimensiunii imaginilor față de modul în care acestea sunt obținute natural, atât \( K \), cât și parametrii
de distorsiune nu o să mai fie valizi. Matricea parametrilor camerei are următorul format:

\begin{equation}
    K = 
    \begin{bmatrix}
        f_x & 0   & c_x \\
        0   & f_y & c_y \\
        0   & 0   & 1
    \end{bmatrix}
\end{equation}

Algoritmul va returna un fișier text în care se vor afla estimările matricilor 
de poziție împreună cu timestamp-ul asociat pentru fiecare cadru în parte în ordine
cronologică. Rezultatul poate fi comparat cu fișiere care conțin valorile reale și 
care respectă același format pentru a verifica corectitudinea algoritmului.   
Diagrama UML prezintă etapele principale ale algoritmului. O componentă 
reprezintă o funcție care se va executa pentru fiecare cadru procesat.  
În continuare, o să detaliez logica fiecărui bloc din punctul de vedere al algoritmilor folosiți și 
al valorilor de intrare și de ieșire asociate acestora.

% AICI
\begin{figure}[htbp] 
  \centering
  \includegraphics[width=1.0\textwidth]{./images/slam_uml.jpeg}
  \caption{Diagrama UML de activități a întregului sistem}
  \label{fig:aici}
\end{figure}
 
\section{Achiziția datelor}
Scopul acestei componente este citirea imaginii de tip RGB, a matricei de adâncime 
și estimarea poziției curente a camerei pe baza măsurătorilor anterioare. În viitor,
o altă funcție a acestei componente ar putea fi extragerea datelor de la alți senzori 
precum un giroscop sau accelerometru, pentru a obține informații suplimentare cu privire
la orientarea și distanța efectuată de către cameră. Acest lucru ar putea duce la o îmbunătățire
considerabilă a estimării inițiale a poziției. Imaginile pot proveni de la cameră în timp real, 
dintr-un video sau dintr-un set de date. Matricea de adâncime poate fi obținută de la o cameră 
RGBD/Stereo sau prin utilizarea unei rețele neurale pentru estimarea distanțelor. 

\section{Extragerea trăsăturilor} 
Ca date de intrare această componentă primeste doar imaginea de tip RGB și extrage
aproximativ 1000 de trăsături și descriptori asociați acestora. Trăsăturile sunt zone
de interes în imagine care pot fi folosite pentru a detecta obiecte sau găsi asocieri 
între cadrele consecutive. Acestea sunt numite și keypoint-uri în literatura de 
specialitate iar librării precum OpenCV au structuri de date dedicate pentru acestea.
O trăsătură poate fi interpretată matematic ca o zonă în care apare o schimbare 
bruscă a gradientului culorii. De cele mai multe ori, astfel de variații se regăsesc 
în zonele de frontieră dintre obiecte, deoarece apare o diferență
de culoare și implicit una de intensitate luminoasă. Zonele slab texturate, cum ar 
fi cerul sau pereții în interiorul unei clădiri au valori asemanatoare pentru
toti pixeli de pe suprafata. Dacă s-ar folosi un keypoint dintr-o astfel de zonă, 
ar fi dificil de spus cu exactitate de unde a fost extras. Acesta ar putea fi asociat 
mai multe coordonate în imagine. În schimb, o cameră 
complet mobilată ar fi o zonă puternic texturată iar un algoritm de detectie de 
keypoint-uri ar putea să gasească ușor 1000 de trăsături pe care să le folosească.
Dacă algoritmul nu reușește să găsească suficiente keypoint-uri pentru a face urmărirea
intre cadre, de obicei minim 500, operația ar eșua. Din această
cauză algoritmul de ORB-SLAM2 dă rezultate eronate în zonele slab texturate. Dacă algoritmul
de extragere funcționează corect iar traiectoria camerei este una stabilă, fără schimbări
bruște ale direcției de deplasare, trăsături similare ar trebui să fie observate în ambele
imagini. Asocierile dintre ele, ne pot da informații despre modul în care s-a deplasat 
camera între 2 cadre. Problema este că aceste keypoint-uri nu pot fi comparate direct
între ele, din aceasta cauză ne folosim de descriptori. Aceștia sunt vectori de diferite 
dimensiuni care trebuie să sintetizeze informația esențială observată în zona respectivă din
imagine, în mod ideal descriptorii ar trebui să rămână invariabili la operațiile de 
redimensionare și rotație aplicate pe keypoint-uri. 
Algoritmului Oriented Fast and Rotated Brief (ORB)\cite{Rublee2011ORBAE} este folosit
pentru extragerea de 
keypoint-uri și descriptori. A fost creat în anul 2011 ca alternativă pentru alți algoritmi
de extragere de feature-uri precum SIFT\cite{sift} și SURF\cite{bay2006surf}.\@ Motivul pentru
care acesta a ajuns atât de popular se datorează mai multor factori:
\begin{itemize}
    \item Este mult mai rapid decat SIFT și SURF, fiind mult mai potrivit pentru sisteme
în timp real și pentru dispozitive embedded\cite{comparisonsiftsurforb}.
    \item La momentul realizării lucrării științifice ORB-SLAM2, atât SIFT cât 
și SURF se aflau sub protectia drepturilor de autor pe când ORB nu avea o astfel de restrictie.
    \item ORB este invariant din punct de vedere al rotației și are o toleranță bună la variația distanței.   
    \item Folosește descriptori binari, care pot fi ușor de comparat folosind distanța Hamming\cite{hamming1950}, 
\end{itemize}
Implementarea algoritmului ORB poate fi separată în 2 componente, calcularea keypoint-urilor și 
cea a descriptorilor. Pașii pe care îi urmează algoritmul sunt realizați într-o structură for loop.
ORB extrage feature-uri la diferite dimensiuni ale imaginii, pentru a crea trăsături mai
robuste la modificarea distanței. Numărul de execuții ale buclei for, este același cu numărul de 
resize-uri pe care trebuie să le aplice algoritmul. Etapele realizate la fiecare iterație sunt următoarele:   
\begin{enumerate}
    \item Calcularea keypoint-urilor folosind algoritmul FAST-9\cite{fast}.
    \item Selectarea celor mai potrivite keypoint-uri folosind Harris Corner Measure\cite{Harris1988ACC}, Trăsăturile
sunt sortate în ordine descrescătoare și sunt selectate primele N cele mai potrivite
    \item Pentru fiecare keypoint se calculează orientarea acestuia folosind intensitatea centroidului. 
    \item Înainte de a calcula descriptorii, se aplică o operație de smoothing Gaussian pentru fiecare 
zonă selectată de un keypoint, aceasta având o dimensiune prestabilită de $ 31 \times 31 $ de pixeli. Kernel-ul folosit este de $ 5 \times 5 $.
    \item Pentru fiecare keypoint se calculează un descriptor binar de tip BRIEF.\@ 
    \item Fiecare descriptor va fi transformat folosind o matrice de rotație, unghiul fiind dat de orientarea keypoint-ului corespondent.
    În acest fel se obțin descriptorii de tip steer BRIEF\cite{brief}.  
    \item  Se obtine rBRIEF, rotated BRIEF, o variantă optimizată a steer BRIEF, prin alegerea bitilor
despre care se știe ca au varianță mare și grad de corelație scăzut între ei.
\end{enumerate}   

În cazul algoritmul FAST-9, 
cifra 9 vine de la diametrul ferestrei circulare în care se face compararea între valoarea 
intensitații pixelului și centru. Acest algoritm primește ca parametru imaginea și pragul 
pe care trebuie să îl depășească diferența de intensitate între pixeli pentru a fi considerat
un keypoint. De cele mai multe ori, informația dată de keypoint-uri este redundantă, pentru
a selecta un număr restrâns de trăsături, de preferat cele mai expresive, se folosește Harris
Corner Measure. Pentru a calcula orientarea unui keypoint, vom defini noțiunea de centroid  $ C $ 
care este diferit de centrul zonei de $ 31 \times 31 $ pixeli din imagine, $ O $. Vectorul $ \vec{OC} $ va fi cel 
care va da unghiul $ \theta $ al keypoint-ului pe care îl vom obține direct din următoarea 
formulă, unde $ I(x, y) $ reprezintă intensitatea luminoasă a pixelului cu coordonate $ (x, y) $.      
\begin{equation}
m_{pq} = \sum_{x, y} x^p y^q I(x, y), \quad{}
\theta = \text{atan2}(m_{01}, m_{10})
\end{equation}
În etapele 5 și 6 se realizează calcularea descriptorilor: aceștia vor avea formă binară și o 
lungime finală de 256 de biti. Compararea lor se va realiza folosind distanța Hamming. Cu cât 
2 descriptori au o valoarea mai mică a acestei distanțe, cu atat mai similari sunt. Valorile 
fiecărui bit ai descriptorilor sunt asociate pe baza unui test binar în care este comparată intensitatea a 2 
puncte din planul imaginii. Problema este că descriptorii BRIEF sunt sensibili la schimbările de
rotație, din această cauză, prin rotirea coordonatelor pixelilor cu unghiul $ \theta $ al 
orientării se obține steered BRIEF.\@ Pentru a obține rBRIEF, au fost învățate în offline prin
aplicarea unui algoritm de tip Greedy, care teste de verificare a intensității au cea mai mare 
varianță, și primele 256 dintre acestea au fost alese pentru a alcătui descriptorul.       

\section{Harta punctelor din spatiu}
Unul dintre scopurile fundamentale ale algoritmului de ORB-SLAM2, pe langa cel 
de estimare al traseului camerei este cel de creare a hartii locale a mediului
inconjurator. Problema este ca, in comparatie cu versiuni mai avansate ale
acestui algoritm, special modificate pentru o reconstructie cat mai fidela a mediului,
algoritmul nostru trebuie sa functioneze pentru un sistem embedded care nu are 
capacitate de procesare suficient de mare, fiind nevoit astfel sa simuleze mediul 
printr-un nor de puncte cu o densitate redusa (sparse). Cele 2 sarcini sunt 
dependente una de cealalta, fiecare element din norul de puncte actioneaza ca
o referinta, o caracteristica a mediului care ar trebui sa fie observata de fiecare 
data cand punctul se afla in frustum-ul camerei. De exemplu: presupunem ca avem
o imagine in care este observata in totalitate o masa in interiorul unei incaperi. 
ORB va identifica aproape instantaneu feature-urile (colturile mesei) si teoretic,
indiferent de modul in care ne-am roti in jurul mesei, aceleasi feature-uri
ar trebui sa fie observate de fiecare data, mai mult de atat, considerand ca mediul 
este static, acestea sunt mereu asociate cu acelasi punct din spatiu, devenind astfel 
o referinta pe baza careia putem estima modul in care s-ar deplasa camera. 
In literatura de specialitate, aceste puncte din spatiu sunt  denumite MapPoint-uri 
iar functionalitatea corecta a algoritmului depinde strict de 
modul in care aceste MapPoint-uri sunt observate cadru cu cadru. Un astfel de punct 
in spatiu este creat dintr-un keypoint, dar nu vom avea nevoie de toate punctele din 
regiunea respectiva si vom considera ca centrul este punctul cel mai semnificativ,  
avand coordonate \(x\) si \(y\), si distanta fata de camera fiind estimata ca fiind 
\(d\). Ne vom folosi de matricea transformarii din coordonatele camerei
in coordonatele globale si de parametrii interni ai camerei \(f_x\), \(f_y\) 
distanta focala, si \(c_x\), \(c_y\) coordonatele centrului imaginii.
Vectorul coloana cu 3 dimensiuni reprezinta pozitia in spatiu a feature-ului gasit
in cadrul curent, astfel am creat primul MapPoint. Ca alternativa,
pentru a nu lucra cu matrici de dimensiuni $ 4 \times 4 $ putem folosi \(R_{wc}\) reprezentand
matricea de rotatie si \(t_{wc}\) vectorul de translatie.

\begin{equation} 
\begin{bmatrix}
X \\
Y \\
Z \\
1
\end{bmatrix} = T_{wc} *  
\begin{bmatrix}
\frac{x - c_x}{f_x} * d \\
\frac{y- c_y}{f_y} * d \\
d \\
1
\end{bmatrix}, \quad{}
\begin{bmatrix}
    X \\
    Y \\
    Z 
\end{bmatrix} = R_{wc} *
\begin{bmatrix}
    \frac{x - c_x}{f_x} * d \\
    \frac{y- c_y}{f_y} * d \\
    d
    \end{bmatrix} + t_{wc}
\end{equation}

In literatura de specialitate MapPoint-urile sunt considerate ca fiind 
niste ancore (landmark) pozitionate dinamic de catre algoritm, acestea sunt asociate
cu un anumit cadru cheie si ne vor ajuta in optimizarea matricei de pozitie dar si
pentru sarcina de relocalizare si de memorare a zonelor cunoscute.

\section{Asociere puncte din spatiu cu feature-uri ORB}
Ca date de intrare avem feature-urile si descriptorii extrasi din imagine, matricea
de adancime si harta de MapPoint-uri. Scopul acestei componente este sa gaseasca 
cat mai multe asocieri de 1:1 intre feature-uri si MapPoint-uri. Intr-un caz ideal
fiecare feature gasit ar trebui sa aiba asociat un MapPoint, dar in realitate nu se 
poate intampla acest lucru din 2 motive: 
imperfectiuni ale algoritmului ORB de detectie ale feature-urilor: acesta nu garanteaza 
ca acelasi feature va fi gasit de fiecare data pentru cadre consecutive si faptul ca 
modelul isi schimba orientarea, facand ca MapPoint-urile aflate la limita campului 
vizual al camerei sa nu mai poata fi observate. Un MapPoint este un feature 
al unui cadru anterior, proiectat in spatiu. In final, aceasta componenta realizeaza 
tot o comparare de feature-uri intre cadrul curent, si multiple cadre anterioare.
Aceasta operatie de comparare se realizeaza prin intermediul distantei Hamming dintre
descriptori, cu cat valoarea obtinuta este mai mica, cu atat cele 2 feature-uri sunt 
mai asemanatoare. Exista mai multe tipuri de algoritmi folositi pentru feature 
matching, dar cel folosit in implementarea curenta este Brute Force Feature Matching
optimizat. Acest algoritm primeste ca date de intrare 2 seturi de feature-uri si 
incearca sa gaseasca asocieri intre ele. Asocierele sunt facute cu ajutorul descriptorilor,
se calculeaza distanta Hamming iar daca valoarea obtinuta este minima, perechea 
respectiva de feature-uri se considera ca a fost corect asociata. Pentru ORB-SLAM2 
o potrivire intre 2 keypoint-uri arata ca ele se refera la exact acelasi punct din spatiu,
observat din 2 imagini diferite.
Daca \(N\) este numarul de feature-uri din primul set, \(M\) numarul de feature-uri din 
al doilea set si \(D\),dimensiunea descriptorului, in cazul nostru ORB este 32, complexitatea
algoritmului devine \(O(N * M * D)\). Destul de costisitor de folosit
pentru un sistem in timp real, mai mult de atat este predispus la erori, compararea 
feature-urilor nu tine cont de locatia acestora in imagine, obtinandu-se astfel asocieri
care matematic par corecte, dar ele nu au sens din punct de vedere logic. Pentru a rezolva
aceasta problema si a reduce complexitatea temporala se stabileste o fereastra circulara de 
dimensiune prestabilita in jurul punctului de proiectie unde se pot cauta feature-uri.
O data ce 2 keypoint-uri au fost considerate ca facand referinta la acelasi punct din spatiu,
cadrului curent ii este asociat un nou MapPoint. 

\section{Optimizare Estimare Pozitie Initiala}
Aceasta componenta primeste ca data de intrare estimarea pozitiei curente a camerei
\(T_{cw}\), si o asociere bijectiva intre feature-urile gasite in imagine si punctele 
care exista la momentul respectiv in spatiu. Ca date de iesire vom avea doar matricea 
pozitiei curente a camerei optimizata. Daca asocierile intre feature-uri si MapPoint-uri
sunt perfecte, ar trebui ca proiectia punctului din spatiu pe imagine sa se suprapuna pe 
centrul keypoint-ului. Rareori se petrece acest lucru in practica, iar distanta dintre
proiectia unui MapPoint si coordonotale centrului feature-ului reprezinta eroarea de 
asociere. Pentru a minimiza aceasta eroare, exista 2 optimizari care se pot face:
prima este modificarea valorilor matricei de pozitiei, iar cea de-a doua este modificarea 
coordonatelor din spatiu ale MapPoint-ului. Inainte de a prezenta algoritmul de optimizare
folosit, voi arata modul in care se proiecteaza un MapPoint in plan.
\subsection{Proiectarea MapPoint in planul imaginii}
Aceasta operatie de proiectie poate fi vazuta ca aplicarea unui functii $ \pi(\cdot) $
ce primeste ca date de intrare coordonatele globale ale punctului, iar ca rezultat va
returna coordonatele omogene in planul imaginii. Aceasta transformare se petrece in 
2 etape:
\begin{enumerate}
    \item conversia din sistemul de coordonate globale in sistemul de coordonate al camerei
    \item conversia din sistemul de coordonate al camerei in sistemul de coordonate al imaginii
\end{enumerate}
In prima etapa putem folosi coordonatele omogene, pentru a face conversia in mod direct.
Alternativ, putem extrage din matricea de pozitie \(T_{cw}\) atat matricea de 
rotatie \(R_{cw}\) cat si vectorul coloana de translatie \(t_{cw}\).

\begin{equation}
\mathbf{X}_{camera} = \mathbf{T}_{cw} \cdot 
\begin{bmatrix}
\mathbf{X}_w \\
1
\end{bmatrix}, \quad
\mathbf{T}_{cw} =
\begin{bmatrix}
\mathbf{R_{cw}} & \mathbf{t_{cw}} \\
\mathbf{0}^T & 1
\end{bmatrix}, \quad
\mathbf{X}_{camera} = \mathbf{R_{cw}} \cdot \mathbf{X_w} + \mathbf{t_{cw}}
\end{equation}

Matricea \(T_{cw}\) este utilizata atat pentru a descrie pozitia si orientarea in spatiu 
cat si pentru a schimba din sistemul de coordonate global in cel al camerei.
In sistemul de coordonate global, un punct se afla la exact aceeasi valoare indiferent 
de pozitia camerei care il priveste, in sistemul de coordonate al camerei, pozitia unui 
MapPoint o sa difere de fiecare data.
In etapa a doua MapPoint-ul este in sistemul de referinta al camerei, 
coordonatele fiind reprezentate prin vectorul coloana \(X_{camera}\). Vom considera 
a 3-a valoare a acestui vector \(Z_c\). Aceasta reprezinta distanta dintre planul camerei
si punctul pe care il analizam. \(Z_c\) ne spune daca punctul respectiv poate fi observat
in imagine. Daca valoarea \(Z_c\) este mai mica sau egala cu 0, inseamna ca punctul 
se proiecteaza in spatele camerei, facandu-l invalid. In situatia in care \(Z_c\) este 
mai mare decat 0, vom realiza conversia in coordonatele omogene ale imaginii cu ajutorul
urmatoarei formule, \(u\) fiind asociat axei x si \(v\) fiind asociat axei y. Daca 
valorile \(u\) si \( v \) au valori mai mari ca 0 si mai mici decat dimensiunea imaginii,
vectorul coloana \([u, v, 1]\) este rezultatul cautat. 

\begin{equation}
    \begin{bmatrix}
        u \\
        v \\
        1
        \end{bmatrix}
        =
        \mathbf{K} \cdot
        \begin{bmatrix}
        \frac{X_c}{Z_c} \\
        \frac{Y_c}{Z_c} \\
        1 
        \end{bmatrix}, \quad{}
        \mathbf{K} =
        \begin{bmatrix}
        f_x & 0 & c_x \\
        0 & f_y & c_y \\
        0 & 0 & 1
        \end{bmatrix}
\end{equation} 


\subsection{Motion Only Bundle Adjustment}
Algoritmul folosit in aceasta etapa se numeste Motion Only Bundle Adjustment\cite{bundleAdjustment}. Acesta 
modifica doar matricea pozitiei curente a camerei. Coordonatele punctelor din spatiu sunt
considerate ca fiind constante. Algoritmul este unul iterativ, minimizand o functie de cost.
Forma generala a functiei de cost este suma erorilor de proiectie pentru toate perechile
(feature, MapPoint). Iar formula generala este aceasta.
 \begin{equation}
    \mathbf{R}_{cw}, \mathbf{t_{cw}} = \min_{\mathbf{R}_{cw}, t_{cw}} \sum_{i=1}^{N} \rho\left( \left\| \mathbf{x}_i - K \cdot \left( \mathbf{R}_{cw} \cdot \mathbf{X}_i + \mathbf{t_{cw}} \right) \right\|^2 \right)
\end{equation}

In aceasta formula, \(x_i\) reprezinta coordonatele omogene feature-ului in sistemul de 
coordonatele al imaginii
iar \(X_i\) reprezinta coordonatele globale ale MapPoint-ului pentru care calculam eroarea
de proiectie. Simbolul $ \rho(\cdot) $ reprezinta functia Huber\cite{huber1964} pentru scalarea valorilor de eroare.
Daca o asociere intre un feature si un MapPoint nu este potrivita, diferenta dintre centrul feature-ului
si proiectia MapPoint-ului este mai mare decat un prag prestabilit. Aceasta diferenta, lasata nemodificata,
ar destabiliza algoritmul. Iar o astfel de problema este usor de observat, daca modificarea matricei
de pozitie duce la variatii enorme a orientarii sau a translatiei intre 2 cadre consecutive, 
atunci cel mai probabil asocierile intre feature-uri si MapPoint-uri aveau valori eronate. 
Termenul de \(outlier\) este folosit pentru a descrie o pereche incorecta. Functia de pierdere Huber reduce 
valoarea acestor outlier-ere permitandu-le in acelasi timp sa faca parte din algoritmul de 
optimizare. In acest fel algoritmul devine mai robust si capabil ajunga la o valoare optima 
in mai putine iteratii. Mai jos este prezentata formula matematica a functiei Huber Loss, 
unde $ \delta $ reprezinta un numar real pozitiv, toleranta a erorii de proiectie. 

\begin{equation}
\rho(s) =
\begin{cases}
\frac{1}{2}s^2 & \text{if } |s| \leq \delta \\
\delta (|s| - \frac{1}{2}\delta) & \text{if } |s| > \delta
\end{cases}
\end{equation}

In urma executiei algoritmului obtinem matricea de pozitie optimizata, mai mult de atat,
stim care dintre perechile (feature, MapPoint) au avut statutul de outlier si le putem 
elimina pentru a nu influenta in mod negativ functionalitatea algoritmului.

\section{Crearea unui cadru cheie}
Aceasta componenta primeste ca date de intrare absolut toate informatiile procesate de 
pana acum pentru cadrul curent: imaginea de tip rgb, matricea de adancime, punctele cheie,
descriptorii, asocierile (feature, MapPoint) si matricea estimarii pozitiei. 
Toate acestea impreuna vor alcatui un cadru cheie care va fi salvat in memorie.
Salvarea pozitiilor cadrelor anterioare ne poate ajuta in 2 feluri. Putem folosi doar 2 
cadre anterioare pentru a estima pozitia celui care urmeaza bazandu-ne pe legea inertiei. 
Consider ca o data inceputa deplasarea camerei intr-o anumita directie, este foarte 
probabil ca aceea miscare sa fie mentinuta si la urmatorul cadru. Fie \(T_{cw}\) matricea de 
pozitie pentru cadrul la care vrem sa estimam deplasarea, iar \(T_{cw1}, T_{cw2} \) matricile 
de pozitie a celor 2 cadre imediat predecesoare. Formula de estimare a pozitiei curente este:

\begin{equation}
\mathbf{T_{cw}} = \mathbf{T_{cw1}} \cdot \left( \mathbf{T_{cw2}}^{-1} \cdot \mathbf{T_{cw1}} \right)
\end{equation} 

Al doilea motiv pentru care avem nevoie de cadre cheie este recreearea mediului si a traseului parcus.
Incercam sa salvam numarul minim de cadre necesare pentru a reproduce harta de MapPoint-uri a mediului
inconjurator. Un cadru cheie nou (KeyFrame) aduce cu sine MapPoint-uri noi, extrase din feature-urile
gasite in imaginea respectiva. Functionarea corecta a urmarii cadru cu cadru, este determinata de numarul de 
MapPoint-uri gasite in imaginea curenta in comparatie cu un cadru de referinta. In momentul 
in care numarul de puncte cheie gasite in imaginea curenta scade sub un anumit prag, stim ca este
necesar un nou cadru cheie care: sa stabilizeze urmarirea, sa introduca noi MapPoint-uri, si sa 
ajute la optimizarea intregii harti a mediului. 

\subsection{Optimizare harta locala}
Harta locala este alcatuita din KeyFrame-uri si MapPoint-uri. Pentru a optimiza
harta trebuie sa adaugam noi puncte de tip MapPoint si noi KeyFrame-uri in ea.
Pentru a valida conexiunile care deja exista. Modul in care sunt create si sterse 
punctele urmeaza o abordare numita survival of the fittest. La fiecare nou KeyFrame 
adaugat, sunt create in aproximativ 100 de noi MapPoint-uri si inserate in harta. Acestea
vor fi supuse unui test care sa evalueze cat de usor sunt observate feature-urile pe care
le reprezinta.  Din punct de vedere matematic, un MapPoint este observat de un KeyFrame
daca proiectia acestuia in imagine este un vector valid in sistemul de coordonate 
al KeyFrame-ului respectiv. Cu cat mai multe Keyframe-uri observa acelasi MapPoint,
cu atat mai stabil este punctul respectiv din spatiu. Intr-un caz ideal, ar trebui ca
orice MapPoint creat sa fie stabil. De cele mai multe ori nu se intampla acest lucru din
cauza erorilor de feature matching. Astfel, doar cele mai evidente feature-uri raman 
salvate in harta pana la finalul algoritmului. Scopul acestei componente este adaugarea 
KeyFrame-urilor noi, eliminarea celor redundante si testarea stabilitatii tuturor MapPoint-urilor
create. Pentru a salva Keyframe-ul curent in harta urmatoarele operatii trebuie urmate:
\begin{enumerate}
    \item Folosind harta de adancime, sunt selectate cele mai apropiate \( N \) feature-uri 
    care nu au un MapPoint asociat si au valoarea adancimii mai mare ca 0.
    Coordonatele acestor feature-uri sunt proiectate in spatiu pentru a obtine noi MapPoint-uri.      
    \item KeyFrame-ul curent este comparat cu alte cadre cheie, pentru a vedea cu cine imparte 
    cele mai multe puncte comune. Keyframe-urile sunt stocate in harta intr-o structura de tip 
    graf neorientat unde nodurile sunt cadrele cheie iar arcele sunt numarul de MapPoint-uri comune 
    dintre ele.
    \item sunt eliminate punctele cheie redundante sau care au fost observate in prea putine 
    cadre pentru a fi luate in considerare.
    \item se executa un algoritm numit Local Bundle Adjustment in care KeyFrame-urile 
    care au cele mai multe puncte comune cu cadrul curent analizat vor avea matricea de
    pozitie optimizata si coordonatele globale ale MapPoint-urilor asociate acestora.
\end{enumerate} 
 

Local Bundle Adjustment este similar cu Motion Only Bundle Adjustment. In continuare vorbim de un
algoritm iterativ care incearca sa minimizeze o functie de cost, folosind metoda scaderii 
gradientului. Diferenta este ca optimizarea se aplica pe mai mult de un cadru cheie: atat
matricea poztiei si punctele din spatiu asociate (MapPoint-urile) vor fi optimizate.
Avem urmatoarele etape:
\begin{enumerate}
    \item Se creeaza lista de cadre mobile. Plecand de la cadrul curent, se vor selecta
    toti vecinii de gradul 1 si 2 din graful neorientat stocat in harta. Aceste Keyframe-uri
    sunt considerate \(mobile\) deoarece matricea lor de pozitie se va modifica.  
    \item Se creeaza lista de MapPoint-uri ale caror coordonate vor fi optimizate. Fiecare
Keyframe din multimea cadrelor mobile observa un numar de puncte in spatiu, toate aceste
puncte vor fi folosite de catre algoritmul de optimizare.
    \item Se creeaza lista de cadre fixe pentru care matricea de pozitie nu se va 
modifica. Pentru fiecare punct din lista de MapPoint-uri ce vor fi optimizate se va itera 
prin lista de KeyFrame-uri care observa acel MapPoint. Daca un KeyFrame apartine multimii 
de cadre mobile va fi ignorat, iar daca nu, va fi adaugat in lista de
cadre fixe. Acestea sunt incluse in algoritm pentru a garanta ca modificarea coordonatelor
unui MapPoint nu va strica asociera (feature, MapPoint) in cadrele care nu vor avea matricea
de pozitie modificata. 
\end{enumerate}

Lucrarea stiintifica care sta la baza ORB-SLAM2, implementeaza deja functia de 
cost pe care algoritmul de Local Bundle Adjustment o foloseste. Pentru a intelege mai usor formula 
matematica, aceasta trebuie privita de la dreapta la stanga. \(E_{kj}\) reprezinta eroarea
de proiectie a unui MapPoint pe feature-ul asociat. Indicele \(k\) apartine KeyFrame-ului,
\(j\) reprezinta ordinul perechii (feature, MapPoint) pentru care calculam eroarea. Simbolul 
$ \rho(\cdot) $ este asociat functiei Huber, folosita pentru a ameliora efectele perechilor de tip 
outlier. \(X_k\) este o notatie pentru multimea tuturor asocierilor (feature, MapPoint) pentru un 
Keyframe \(k\). Suma erorilor tuturor perechilor este calculata pentru toate cadrele fixe si mobile.
Parametrii care vor fi optimizati sunt: coordonatele MapPoint-urilor selectate de catre
algoritm \(X_i\) cat si matricile de pozitie pentru cadrele mobile \( K_l \). Algoritmul 
optimizeaza valorile pana cand ajunge la o valoare de minim sau pentru un numar de iteratii. 

\begin{equation}
\{\mathbf{X}_i, \mathbf{R}_l, \mathbf{t}_l \mid i \in \mathcal{P}_L, l \in \mathcal{K}_L\} = \arg \min_{\mathbf{X}_i, \mathbf{R}_l, \mathbf{t}_l} \sum_{k \in \mathcal{K}_L \cup \mathcal{K}_F} \sum_{j \in \mathcal{X}_k} \rho(E_{kj})
\end{equation}
\begin{equation}  
E_{kj} = \left\| \mathbf{x}_{j} - K \cdot \left(\mathbf{R}_k \mathbf{X}_j + \mathbf{t}_k\right) \right\|
\end{equation}

\subsection{Reteaua Neurala FastDepth}
Retele Neurale Artificiale sunt o tehnica des intalnita in Machine Learning pentru a rezolva 
sarcini complexe pentru care nu exista solutii algoritmice clar definite sau implementarea acestora 
este mult prea costisitoare. Conform \cite{GURESEN2011426} si \cite{prince2023understanding} retele neurale definesc o functie nonliniara care gaseste o 
corespondenta intre un set multivariat de date de intrare \(x\) si un set multivariat de 
date de iesire \(y\), modificand un set de parametrii $ \phi $, $ f(x, \phi) = y $. Aceasta este
alcatuita dintr-un numar enorm de elemente de procesare care contin parametrii functiei $ \phi $,
conectate intre ele intr-o structura de tip graf si dispuse pe straturi. Cele mai importante fiind: stratul de intrare
si de iesire, unde se stabileste forma generala pe care trebuie o sa respecte datele care vor
parcurge reteaua si modul in care va arata rezultatul obtinut. Celelalte nivele sunt denumite 
straturi ascunse. Aceastea fac prelucrarea informatiei primite de la straturile anterioare 
si o transmit mai departe. Spunem ca o retea neurala invata din datele primite, daca isi 
modifica parametrii $ \phi $ astfel incat sa reprezinte cu mai multa acuratete corespondenta
intre datele de intrare \(x\) si cele de iesire \(y\).
FastDepth\cite{fast} este o arhitectura de retea neurala folosita pentru estima adancimii in imagini. 
Aceasta primeste o imagine de tip RGB al interiorului unei incaperi si returneaza o matrice
cu valori in intervalul $0m - 10m$ estimand pentru fiecare pixel in parte distanta de la
planul de proiectie al imaginii pana la punctul din spatiu surprins de fotografie. Scopul nostru
este antrenarea unei retele neurale care sa produca o matrice de adancime cu valori cat mai 
apropiate de distanta reala la care se afla obiectele fata de camera. ORB-SLAM2
foloseste o camera tip RGBD / Stereo care descrie cu foarte mare acuratete distanta pana intr-un
anumit punct din spatiu, dar creeaza o matrice rara de valori, suprafetele lucioase sau 
cele car nu au putut fi clar observate vor avea adancimea 0 pentru a arata ca distanta nu a putut 
fi corect estimata in pixelii respectivi. Un motiv pentru care retelele
neurale sunt o alternativa buna este ca ele vor avea o estimare pentru fiecare pixel din imagine.
Reteaua FastDepth pare sa realizeze o pseudosegmentare a zonelor din imagine, identifica
conturul obiectelor si atribuie valori ale distantei asemanatoare pentru pixelii ce apartin aceleasi
entitati. Arhitecturile de mari dimensiuni nu pot fi folosite in timp real 
fara a utiliza un GPU, dar nu este cazul si pentru arhitectura FastDepth care poate procesa 
aproximativ 100 de cadre pe secunda. De asemenea consuma o cantitate redusa de memorie,  
parametrii retelei pot fi stocati intr-un fisier ONNX ce ocupa mai putin de 8 MB, fiind usor
de integrat intr-un dispozitiv embedded. \\
Un posibil dezavantaj al acestei arhitecturi este limitarea de 10m, fiind nepotrivit
de folosit afara, dar ideal pentru un spatiu inchis de mici dimensiuni. Un alt dezavantaj
este ca valorile aproximate vor avea o acuratete mai slaba decat cele obtinute de camerele
Stereo/RGBD.\\
Exista mai multe filozii cand vine vorba de modul in care ar trebui sa
arate arhitectura retelelor neurale si operatiile pe care ar trebui sa le realizeze
fiecare strat. Feed forward neural network a fost printre primele arhitecturi definite. 
Elementele de procesare sunt dispuse pe straturi, si fiecare strat primeste input-ul de la stratul 
precedent si transmitea output-ul la stratul imediat urmator. Informatia circula liniar, de la 
intrarea in retea pana la finalul acesteia. Abordarea s-a dovedit eficienta in situatiile 
in care era nevoie de retele neurale de mici dimensiuni, cu un numar redus de straturi si 
parametrii. In momentul in care crestea complexitatea, abordarea de feed forward neural network 
devenea greu de antrenat si dadea rezultate mai slabe\cite{GURESEN2011426}. Pentru a rezolva aceasta problema 
au aparut arhitecturile de tip residual network. Acestea au aplicatii in procesarea imaginilor\cite{he2015deepresiduallearningimage},
unde datele de intrare au dimensiuni mari si este nevoie de multe nivele pentru a extrage
suficiente informatii. Principiul de functionare este utilizarea unor straturi reziduale 
denumite si skip connections, in care rezultatul unui strat este salvat si transmis ca data de
intrare la un alt nivel decat la cel imediat urmator. Abordarea aceasta pastreaza din informatiile initiale ale 
datelor de intrare in straturile viitoare stabilizand antrenarea. O alta arhitectura des 
intalnita este cea de encoder-decoder folosita in numeroase aplicatii practice in ceea ce 
priveste imaginile: sarcini de colorare a imaginilor gri\cite{imageColorization}, reconstructie a imaginilor care 
contin parti lipsa si de generare de imagini: un exemplu fiind Variational Auto Encoder\cite{Islam2023FastAE}. \\
Pe langa tipurile de arhitecturi propuse, exista mai multe categorii de straturi in retele neurale.
Primele folosite erau cele fully connected unde fiecare element de procesare era conectat 
cu toate celelalte elemente de procesare din stratul urmator. Matematic, operatia poate fi vazuta 
ca o inmultire de matrici, o operatie costisitoare, iar utilizarea exclusiva a straturilor complet 
conectate creea o retea neurala incapabila sa reprezinte functii nonliniare, scazand capacitatea de
generalizare. De cele mai multe ori straturile liniare sunt folosite impreuna cu functii de
activare nonliniare precum ReLU sau Sigmoid dar in continuare ramane problema numarului mare 
de parametrii care trebuie antrenati. Din aceasta cauza au fost create straturile convolutionale
care  folosesc mai putini parametrii si au aplicabilitate in procesarea imaginilor. Principiul
teoretic pe care se bazeaza este ca pixelii alaturati in imagine au aceeasi semnificatie,
reprezentand acelasi feature. Operatia de convolutie trebuie realizata pe o zona a imaginii iar
modificarea parametrilor afecteaza output-ul generat de mai multi pixeli. Se stabileste un kernel,
o matrice de  mici dimensiuni, in FastDepth folosindu-se kernel-uri de (3, 3), acestea vor stoca
parametrii \(w_{mn}\) pe care reteaua neurala ii va antrena pentru stratul 
convolutional. In formula \(h_{ij}\) reprezinta intensitatea pixelului dupa calculul
operatiei de convolutie, iar \(x_{ij}\) este valoarea intensitatii pixelului de pe coloana i si linia j.  
Litera \(a\) reprezinta functia de activare folosita iar \(\beta\) este o valoare numerica
denumita bias. Acesta poate fi modificat in timpul antrenarii si creste
capacitatea de generalizare a functiei de convolutie\cite{prince2023understanding}.  
\begin{equation}
h_{ij} = a \left[ \beta + \sum_{m=1}^{3} \sum_{n=1}^{3} \omega_{mn} x_{i+m-2, j+n-2} \right]
\end{equation}
Stratul de convolutie este in continuare prea costisitor pentru a crea o arhitectura in timp
real de mari dimensiuni. Presupunem ca avem un vector de intrare pentru un strat de convolutie
cu dimensiunile \([d_{in}, h, w]\) unde \(d_{in}\) este numarul de canale,  \(h \) inaltimea si \(w\) latimea vectorului.
Kernelul folosit are dimensiunile \([k, k, d_{in}, d_{out}]\),
unde \(d_{out}\) este numarul de canale rezultate in urma convolutiei.
In total se vor executa $ h \cdot w \cdot d_{in} \cdot d_{out} \cdot k \cdot k$ operatii.
Pentru  a rezolva aceasta problema a fost creat un strat numit Depthwise Separable Convolutions\cite{sifre2014rigidmotionscatteringtextureclassification}, 
obtinut prin compunerea a 2 straturi de convolutie, unul numit depthwise convolution,
iar celalalt pointwise convolution. Aceasta abordare creste viteza de procesare si imbunatateste
acuratetea in sarcini de clasificare pentru seturi de date precum ImageNet ILSVRC2012. In cazul 
depthwise convolution, fiecare canal al datelor de intrare este procesat de un singur kernel
al stratului de convolutie. Pointwise convolution uneste printr-o combinatie liniara 
rezultatul procesarii fiecarui canal. Complexitatea temporala obtinuta astfel este de:
$ h \cdot w \cdot d_{in} \cdot (k^2 + d_{out})$\cite{fast}. \\ 

FastDepth foloseste tehnica de skip connections, straturile finale primind ca date de intrare
valorile calculate de straturile aflate la inceput, si urmeaza o arhitectura encoder-decoder. 
Encoder-ul transforma datele de intrare intr-o forma mai compacta asemeni unei operatii de arhivare.
Aceasta este realizata folosind o alta retea neurala numita Mobile\_Net\cite{howard2017mobilenetsefficientconvolutionalneural} si 
ulterior Mobile\_Netv2\cite{sandler2019mobilenetv2invertedresidualslinear} 
care reduce numarul de parametrii si creste viteza de procesare fara a impacta acuratetea. 
Partea de decoder este alcatuit din 5 straturi de tip depthwise convolution fiecare urmate de o 
interpolare liniara care dubleaza dimensiunea rezultatului, ultimul strat fiind un 
pointwise convolution care uneste canalele obtinute si returneaza matricea de adancime. Pentru
Mobile\_Netv2 exista parametrii preantrenati in libraria Pytorch\cite{paszke2017automatic} pe setul de date ImageNet
fiind un motiv in plus de a folosi aceasta arhitectura in dezvoltarea FastDepth. 
Mobile\_Netv2 foloseste atat depthwise convolution cat si pointwise convolution intr-un
strat numit Inverted Residual, acesta fiind alcatuit din urmatoarele componente unde $ t $ 
este factorul de multiplicare al numarului de canale, $ s $ este parametrul de stride, determina 
daca se micsoreaza numarul de canale, $ h, w, d $ sunt dimensiunile matricei de intrare:
\begin{figure}[htbp] 
  \centering
  \includegraphics[width=1.0\textwidth]{./images/bottleneck_residual_block.png}
  \caption{Straturile unui bloc de tip Inverted Residual, preluat din\cite{sandler2019mobilenetv2invertedresidualslinear}}
  \label{fig:bottleneck_residual_block}
\end{figure}
% continuare detaliere mobileNet_v2
% adaugare poze


\chapter{Detalii de implementare}
\section{Limbaje de programare si librarii folosite}
Implementarea este realizata in C++17. Pentru management-ul librariilor si al codului folosesc
CMake 3.28.3. Acesta imi permite sa grupez in foldere codul scris de mine si face operatia de 
linking automat cu binarele pachetelor folosite. Librariile principale sunt OpenCV\cite{itseez2015opencv} 4.9.0, Ceres\cite{CeresSolver} 2.2.0, Eigen 3.4.0, DBoW2\cite{bagofwordslibrary} si ultima versiune de Sophus
pana la data de ianuarie 2025. In comparatie cu alte librarii care inca mai trec 
prin diverse update-uri, Sophus a intrat intr-o etapa de mentenanta, dezvoltarea efectiva a 
acestuia fiind finalizata din iunie 2024. O prima problema pe care am intalnit-o a fost gasirea 
unei versiuni compatibile de C++ cu toate aceste pachete. Am incercat mai multe variante printre 
care C++11, C++14, C++17 si C++20. Preferinta mea ar fi fost sa folosesc o versiune cat mai noua
cu putinta, dar care sa poata fi compatibila cu toate librariile mentionate. C++11 si C++14 nu 
erau compatibile cu Ceres, versiunea minima pentru aceasta librarie era C++17. C++20  si C++23 nu 
era compatibil cu Sophus si cu Eigen, iar ambele librarii sunt fundamentale deoarece 
implementeaza metode puternic optimizate de a lucra cu matrici iar API-ul lor era mai simplu decat
cel din OpenCV.\@ Singura optiune ramasa a fost C++17 care era incompatibila cu DBoW2. Libraria 
folosea o versiune mai veche a functiei throw pentru erori. In momentul in care am eliminat aceasta
directiva, am putut recompila codul ca librarie. Bibliotecile utilizate sunt urmatoarele: \\
OpenCV este o librarie de computer vision. Contine implementari ale algoritmilor de extragere de
trasaturi precum FAST, ORB, SIFT, SURF, API-uri pentru procesare video: citirea unui video cadru
cu cadru, procesarea de imagini: aplicarea de filtre, transformarea in grayscale, eliminarea distorsiunii
cauzata de camera. Foarte importante sunt structurile ce abstractizeaza matricile si parametrii 
prin care se indentifica trasaturile: cv::Mat si cv::KeyPoint. Structura KeyPoint este 
fundamentala pentru implementarea algoritmului deoarece stocheaza numeroase informatii despre zona
pe care o reprezinta: orientarea acesteia, coordonatele centrului si nivelul la care a fost observat, 
parametrii de care am avut nevoie in fiecare etapa de procesare a cadrelor. Pe langa aceste 
lucruri, OpenCV are un modul dedicat pentru citirea parametrilor retelelor neurale din fisierele care 
urmeaza un format de tip ONNX, fiind o alternativa potrivita daca vreau sa utilizez un model doar 
pentru sarcini de inferenta.  \\
Ca librarie de optimizare am avut de ales intre Ceres si g2o\cite{g2oLibrary}. In implementarea oficiala 
g2o era cel folosit. Motivul principal fiind ca permite abstractizarea parametrilor care 
trebuie optimizati si a relatiilor dintre acestia sub forma unui graf neorientat. 
API-ul de g2o permite activarea si dezactivarea anumitor noduri, pentru a face implementarea 
mai robusta impotriva perechilor de tip outlier, si pentru a putea reintroduce noduri eliminate
temporar in graful de optimizare. Ceres din pacate nu permite acest lucru. O data create conditiile initiale acestea 
pot fi dezactivate si nu mai este permisa reutilizarea lor in aceeasi problema de optimizare. La
finalizarea algoritmului memoria folosita de catre noduri este eliberata. Cu toate acestea, Ceres
are un API usor de utilizat si are o viteza comparativa cu cel din g2o.\\
Folosesc libraria Eigen deoarece este mai simplu API-ul de calcul cu matrici decat 
cel din OpenCV.\@ Pentru a accesa elementele unei matrici in OpenCV se foloseste o referinta la 
vectorul de date facand accesarea elementelor mult mai nesigura iar verificarea indicelui este 
facuta la runtime. In cazul matricilor din Eigen, accesarea elementelor si operatiile
cu matrici sunt verificate la compile time, prevenind astfel erorile inainte de a rula programul. \\
Sophus este o librarie care imi permite sa lucrez cu algebra de tip Lie. In loc de a vedea
estimarile pozitiei ca pe niste matrici de $ 4 \times 4 $, le pot vedea ca pe un vector alcatuit din 7 
elemente. Primii 4 parametrii alcatuiesc un quaternion, aceasta fiind o exprimare vectoriala 
a unei matrici $ 3 \times 3 $ de rotatie, iar ultimii 3 parametrii reprezinta un vector de translatie.
Biblioteca implementeaza operatii care imi permit sa lucrez cu acesti vectori, care fac parte 
dintr-un grup numit \(se(3)\) si garanteaza ca rezultatul obtinut este scalat corespunzator 
pentru a face parte in continuare din aceeasi categorie. \\
DBoW2 este o metoda de tip bag of words pentru compararea imaginilor intre ele. Este utilizat
pentru operatii precum feature matching intre imagini consecutive, relocalizari si recunoastera
zonelor prin care a trecut pentru a inchide buclele create de mai multe cadre cheie salvate
in harta. Acesta este alcatuit dintr-o structura de tip arbore. Fiecare nivel este obtinut
din realizarea unui algoritm de clusterizare a descriptorilor de tip ORB ca de exemplu 
kmeans++, separarea tuturor descriptorilor in functie de centroizi si reluarea aceleasi
operatii in fiecare dintre clusterele nou create. Nodurile de tip frunza sunt alcatuite 
dintr-un singur descriptor. Construirea arborelui se realizeaza intr-o etapa offline. In cazul
librariei DBoW2, setul de date folosit a fost Bovisa 2008{-}09{-}01. Au fost alese 10K imagini
iar pentru fiecare cadru in parte extrasi 1000 descriptori ORB.\@ Acestia au fost folositi 
pentru a crea un arbore de adancimea 6 iar numarul de clustere pe care le creeaza fiecare 
iteratie a algoritmului kmeans++ este de 10. Pe ultimul strat exista un milion de frunze, si 
tot aceeasi lungime o va avea si vectorul de feature-uri care va reprezenta o imagine. Fiecare
descriptor va primi o valoare numerica numita greutate, invers proportionala cu frecventa pe 
care o are acesta. Cu cat este mai rar un anumit descriptor, cu atat este mai util pentru a 
diferenta o imagine de multe altele. Scopul principal al librariei este sa primeasca ca data
de intrare descriptorii ORB ai unei imagini si sa calculeze vectorul sau bag-of-words.
Vectorul bag-of-words este alcatuit in principal din valori de 0. Fiecare descriptor al 
imaginii parcurge arborele de la radacina spre frunze, parcurgerea realizandu-se prin 
calcularea distantei Hamming dintre descriptor si toate nodurile de pe un anumit nivel, 
si alegerea nodului cu distanta Hamming minima. Nodul frunza la care va ajunge va avea asociat
un index, in cazul de fata cu valori de la 0 la un milion. La acelasi index va fi modificata
valoarea din vectorul bag-of-words in valoarea greutatii descriptorului stocat in arbore.
Apelul de biblioteca returneaza de asemenea un vector de feature-uri in care fiecare element
este o pereche de forma $ (int, vector\_descriptori) $, primul element este indexul 
clusterului de la nivelul 4 al arborelui DBOW2, iar cel de-al doilea element reprezinta un 
vector de descriptori din imaginea curenta care se potrivesc in acelasi cluster. Doua imagini pot
fi comparate intre ele prin intermediul acestui vector de feature-uri, lucru care va fi detaliat
in descriera clasei OrbMatcher. In implementarea ORB-SLAM2, nu este practica creearea unui vector
de tip bag of words cu un milion de elemente, mai ales ca majoritatea valorilor sunt 0, asa ca o
reducere a dimensionalitatii vectorului ar creste viteza de calcul a sistemului. Din aceasta 
cauza, compararea descriptorilor se realizeaza doar pana la nivelul 4 in arbore, vectorul bow 
avand doar 1000 de elemente, iar cel de feature-uri avand acelasi numar de elemente cu numarul de 
descriptori. 

\section{Mediu de lucru si principalele clase}
Structura de fisiere este una simpla, in folderul radacina se regaseste fisierul de 
CmakeLists.txt care va fi interpretat de utilitarul cmake pentru a genera automat Makefile-ul. 
Acest Makefile va contine regulile de build si de clean pentru proiect. In fisierul main.cpp
vor fi initializate componentele si se va putea selecta pe care dintre cele 2 seturi
de date se va aplica algoritmul. Aceste seturi de date contin de fapt cadrele dintr-un video 
facut cu o camera RGBD Microsoft Kinetic impreuna cu matricile de adancime si pozitiile acestora
in spatiu pentru fiecare cadru in parte. Aceste seturi de date sunt suficient de complexe pentru 
a permite evaluarea functionarii algoritmului de ORB-SLAM2. Tot in main.cpp, se va realiza citirea
fisierului ORBvoc.txt, acesta contine datele pe care le va folosi clasa ORBVocabulary pentru a
initializa arborele folosit de libraria DBOW2. \\
Tot in folderul radacina se regaseste si fisierul fast\_depth.onnx, in care este stocata 
arhitectura si parametrii retelei neurale FastDepth pentru estimarea adancimii. In folderul de 
include se afla antetele claselor pe care le voi implementa si in folderul de  src se regaseste 
codul de C++ ce implementeaza logica programului.
Am observat ca separarea codului in acest fel este o practica des intalnita in proiectele de mari
dimensiuni si garanteaza flexibilitate in includirea dependintelor intre fisiere. Algoritmul 
ORB-SLAM2 este unul complex, depinzand de o multitudine de parametrii care pot influenta acuratetea.
Cei mai importanti sunt cei corelati cu camera. In fisierul config.yaml se regaseste matricea \(K\), 
parametrii de distorsiune ai imaginii si alte constante pe care le-am considerat ca fiind niste
hiperparametrii ai algoritmului. Acestia vor trebui modificati in functie de mediul in care va 
rula ORB-SLAM2 pentru a garanta functionarea corecta. \\

Clasa TumDatasetReader este responsabila de achizitia de date si de scrierea 
in fisier a traiectoriei pe care o estimeaza algoritmul cadru cu cadru. Achizitia de date 
presupune citirea din memorie a matricei RGB, convertirea acesteia in grayscale pentru 
o procesare mai rapida de catre algoritmul ORB si de obtinerea hartii de adancime pentru 
cadrul respectiv. Acest lucru poate fi realizat in 2 feluri: matricea de distante este citita
din setul de date TUM RGBD si a fost inregistrata cu o camera RGBD tip Microsoft Kinetic,
sau se foloseste reteaua neurala FastDepth care estimeaza in timp real distanta pentru
fiecare pixel din cadrul curent. Imaginea RGB si harta de adancime
vor fi transmise ca parametrii clasei Tracker. TumDatasetReader stocheaza estimarile 
pozitiilor camerei pentru fiecare cadru in parte. Cadrele cheie, cele salvate 
in clasa Map, vor avea matricea de pozitie stocata nealterat in memorie, ele sunt deja relative 
fata de primul cadru citit. Pentru celelalte cadre, matricea de pozitie salvata in clasa
TumDatasetReader este relativa la un cadru cheie, de preferat ultimul cadru cheie creat pana
la citirea imaginii curente. Motivul pentru care se realizeaza salvarea pozitiilor in acest 
fel, este ca doar cadrele din clasa Map sunt salvate in memorie si pot fi optimizate 
de catre algoritmul Bundle Adjustment asa ca doar acestea ar trebui sa aiba valoarea lor 
salvata explicit. \\

Clasa MapPoint este fundamentala pentru buna functionare a algoritmului ORB-SLAM2. Aceasta este 
formata cu ajutorul unui KeyPoint si al unui KeyFrame asociat acestuia. In etapa anterioara, 
am prezentat modul in care se face proiectia coordonatelor unui punct cheie in spatiu, acestea 
devenind coordonatele globale ale MapPoint-ului pe care il creem. Punctului din spatiu 
i se asociaza de asemenea descriptorul acelui keypoint care l-a creat, pentru compararea ulterioara 
cu alte KeyPoint-uri din alte imagini. Un MapPoint are nevoie de un vector de orientare, acesta 
ajuta in verificarea proprietatii unui MapPoint de a fi sau nu vizibil dintr-un KeyFrame. Pentru a 
calcula acest vector de orientare prima data se determina coordonatele globale ale centrului camerei
pentru cadru cheie care a creat acel KeyPoint, acest lucru se realizeaza in felul urmator:
\begin{equation}
    \begin{bmatrix}
        X_{c} \\
        Y_{c} \\
        Z_{c} \\
        \end{bmatrix} = -R_{wc}^t * t_{wc}, \quad{}  
        T_{wc} =     
     \begin{bmatrix}
            R_{wc} & t_{wc} \\
            0 & 1
        \end{bmatrix}
\end{equation} 
Normalizarea diferentei intre coordonatele globale ale centrului camerei si ale MapPoint-ului creeaza
vectorul de orientare. Acesta poate fi modificat, daca se constata ca mai multe cadre observa acelasi
punct. In situatia respectiva, vectorul de orientare final va fi media aritmetica a celorlalti vectori 
de orientare individuali.

Clasa Feature, aceasta componenta nu exista in implementarea oficiala a ORB-SLAM, dar am considerat 
ca utilizarea acesteia ar simplifica codul. Extinde clasa KeyPoint fara a o mosteni explicit,
are asociata distanta, extrasa din matricea de adancime, descriptorul
si o valoare de tip boolean care arata daca punctul este monocular sau stereo. Aceasta clasificare
 se obtine prin compararea adancimii cu o valoare foarte apropiata
de 0. In cazul implementarii mele, daca distanta este mai mica decat $ 1e-2 $, consider ca valoarea
estimata de camera RGBD ori de reteaua neurala nu este corecta si ca punctul respectiv este
monocular, altfel consider ca este stereo. Pentru fiecare KeyPoint extras,
se va crea o instanta a clasei Feature care va fi stocata direct in KeyFrame. Fiecare
Feature are setat pe null la initilializare o referinta la un obiect 
de tip MapPoint. Pentru a garanta functionarea in real time a algoritmilor, asocierea (Feature,
 MapPoint) trebuie sa poata fi accesata in $ O(1) $. Vectorul de elemente Feature, impreuna cu o
structura de tip dictionar unde cheia va fi MapPoint si valoarea de tip Feature, vor face acest
lucru posibil. Singura problema este ca cele 2 structuri incearca sa 
reprezinte aceleasi corelatii, in cazul vectorului am indicele unui Feature drept cheie si incerc
sa accesez MapPoint-ul asociat, iar in cazul dictionarului, am referinta unui MapPoint si incerc sa
obtin adresa unui Feature. Ambele structuri trebuie sa contina aceleasi perechi, altfel 
comportamentul algoritmului devine nedefinit. \\

Clasa KeyFrame, contine mai multe elemente legate de cadrul curent. Pentru a mentine 
functionarea sistemului in timp real, trebuie sa stocam in memorie rezultatele calculelor noastre.
In aceasta clasa se vor regasi matricea de adancime, vectorul de instante ale clasei Feature,
vectorul de trasaturi calculat de metoda bag-of-words implementata in DBOW2 si cadrul initial, 
convertit in format grayscale ce va fi folosit ulterior pentru afisarea in timp real a performantelor
algoritmului. In interiorul constructorului acestei clase sunt mai multe operatii realizate,
majoritatea necesare pentru a creste eficienta accesarii datelor. De exemplu: vectorul de tip Feature
in medie contine 1000 de elemente care nu sunt sortate. In situatia in care proiectam un MapPoint in 
plan ar trebui sa comparam coordonatele proiectiei cu pozitia fiecarui Feature in parte pentru a 
stabili care este cel mai apropiat. O modalitate de a rezolva acest lucru este segmentarea suprafetei
in \(K\) zone, in cazul meu am ales $ K = 100$, fiecare reprezentand o portiune din imaginea initiala,
avand asociate referintele Feature-urilor care se gasesc pe suprafata respectiva. In acest fel,
in functie de zona in care este proiectat un MapPoint, vom stii ce Feature are o posibilitate
mare de a corespunde, reducand astfel numarul de comparatii. 
Considerand ca toate valorile de tip Feature sunt dispuse in mod egal pe suprafata imagini atunci
complexitatea devine, $ O(N / K) $ unde \(N\) reprezinta numarul de Feature-uri iar \(K\) reprezinta
numarul de zone in care a fost impartita imaginea. Constructorul este responsabil de initializarea
structurilor de tip Feature, partitionarea lor in functie de coordonatele in imagine, si de memorarea
estimarii curente a pozitie camerei si a centrului camerei in coordonate globale. Tot in aceasta 
clasa se regaseste structura de tip dictionar (MapPoint, Feature), care va fi adaptata pe tot 
parcursul algoritmului. Alta metoda importanta este: $ get\_vector\_keypoints\_after\_reprojection $.
Aceasta primeste ca date de intrare coordonatele proiectiei unui MapPoint, valoarea ferestrei 
de proiectie, si octava minima si maxima. Octavele reprezinta nivelul la care a fost observat 
un Keypoint in imagine si o estimare grosiera a distantei dintre camera si punctul din spatiu 
observat. Acesta poate sa aiba valori inte 0 si 7 inclusiv si ne spune de cate ori 
s-a facut resize la imagine pentru a surprinde o anumita trasatura. De exemplu: daca 
un Keypoint are valoarea octavei 0, inseamna ca algoritmul a detectat-o in imaginea nemodificata.
Daca ar fi 1, atunci dimensiunea imagini a fost redusa o singura data cu 0.8 din valoarea initiala 
si asa mai departe. Feature-ul care are asociat un MapPoint trebuie sa aiba valori ale octavei
apropiate intre ele. Daca aceasta situatie nu s-ar respecta, ar introduce erori de estimare 
a distantei, ne asteptam ca Feature-uri corespondente, sa fie aproximate la aceeasi distanta.
Altfel ar putea inseamna ca cele 2 puncte din spatiu sunt diferite. Daca mediul are o structura 
simetrica, de exemplu: o sala de clasa cu bancile aliniate una in fata celeilalte, algoritmul 
ar putea observa 2 colturi ce apartin de 2 mese diferite, daca nu ar avea aceasta separare 
pe baza octavei, urmatorul cadru care observa aceleasi mese ar putea sa asocieze eronat punctele 
intre ele, afectand estimarea pozitiei. Fereastra de proiectie reprezinta cat de departe poate 
sa fie Feature-ul de coordonatele punctului de proiectie ale unui MapPoint pentru a fi considerata
corecta o asociere intre cele 2 elemente. In functie de dimensiunea ferestrei aceasta poate intersecta 1, 2 sau 4 subsectiuni din cele 100 in care 
este impartita imaginea. Problema cea mai mare pe care am avut-o cu clasele KeyFrame si MapPoint
era dependenta circulara. MapPoint-urile aveau nevoie de un KeyFrame si un Feature pentru a fi 
create si trebuia sa mentina o lista a KeyFrame-urilor care observa MapPoint-ul respectiv.
In cazul KeyFrame-ului, acesta trebuie sa pastreze referinte asupra tuturor MapPoint-urilor pe 
care le observa. Pentru a rezolva aceasta problema am folosit o clasa aditionala care face 
operatii cu cele 2 structuri si am folosit forward declaration.  \\  

Clasa Map implementeaza harta pe care o foloseste algoritmul ORB-SLAM2. Aceasta este responsabila
de stocarea corecta a KeyFrame-urilor, a MapPoint-urilor si rezolva problema dependentei circulare
a celor 2 clase. Aici am implementate metodele de adaugare/stergere a unui MapPoint dintr-un KeyFrame.
De asemenea, clasa MapPoint contine referinte la toate KeyFrame-urile care o observa. Aceste 
referinte sunt adaugate / sterse de catre 2 metode care se regasesc aici. Clasa Map creaza o 
structura de tip graf ponderat neorientat, in care nodurile sunt reprezentate de KeyFrame-uri. 
Arcele arata daca exista mai mult de 15 puncte comune intre 2 KeyFrame-uri iar ponderea lor este 
determinata de numarul de MapPoint-uri comune. Clasa Map realizeaza operatii pe graful de KeyFrame-uri,
adauga/sterge noduri si face interogari pentru a afla vecinii directi sau cei pe nivel 2. Am ales 
sa implementez aceasta structura folosind $ std::unordered\_map $. Drept cheie va avea KeyFrame-ul 
curent iar valoarea returnata de structura de tip dictionar va fi un alt $ std::unordered\_map $, 
ce va contine toate celelalte KeyFrame-uri cu care este direct conectata dar si ponderea conexiunii.
In acest fel accesarea vecinilor de ordinul 1 va fi o operatie ce se poate realiza in timp constant.
Functia $ track\_local\_map $ este folosita de catre clasa Tracking. Aceasta primeste ca date 
de intrare cadrul curent si ultimul cadru cheie salvat. Nu returneaza nimic, doar incearca 
sa gaseasca cate un MapPoint pentru Feature-urile care inca nu au fost corelate cu un punct 
din spatiu. Aceasta operatie este costisitoare si functioneza in felul urmator:
\begin{enumerate}
    \item sunt cautate toate KeyFrame-urile vecine de gradul 1 si 2 cu ultimul KeyFrame adaugat
    \item din aceste KeyFrame-uri sunt extrase toate MapPoint-urile observate de catre ele 
    \item MapPoint-urile sunt proiectate si sunt cautate potriviri pentru Feature-urile
care inca nu au MapPoint-uri asociate.
\end{enumerate}
Pentru a nu fi necesar sa calculam de fiecare data KeyFrame-urile vecine si  
harta locala de MapPoint-uri, le stochez ca variabile in interiorul clasei Map. Acestea vor 
fi modificate in momentul in care un KeyFrame este adaugat in harta. Intr-un caz ideal, ar 
trebui ca pentru fiecare Feature adaugat sa se gaseasca un MapPoint, dar acest lucru rareori 
se intampla. In situatia in care s-au gasit mai putin de 30 de puncte din spatiu care s-au 
proiectat corect in imagine, se considera ca a aparut o eroare de urmarire si algoritmul incepe o
etapa de relocalizare. \\

Clasa OrbMatcher este responsabila de realizarea urmarii feature-urilor asemanatoare intre 
cadre consecutive. Inainte de a incepe prezentarea metodelor implementate, voi descrie
pipeline-ul de procesare al unui punct din spatiu pentru a fi considerat observabil de 
catre camera. Avem o instanta a obiectului MapPoint $ mp $, daca una dintre operatiile
prezentate esuaeaza, punctul respectiv este ignorat de catre KeyFrame-ul curent.
\begin{enumerate}
    \item Se proiecteaza coordonatele globale ale $ mp $ in planul imaginii folosind matricea 
de estimare a pozitiei $ T_{cw} $ si matricea parametrilor camerei $ K $. Se verifica daca 
coordonatele proiectiei sunt valide pentru imagine.     
    \item Se calculeaza distanta $ d $ de la centrul camerei la $ mp $. In functie de valoarea 
octavei stocata in acest MapPoint, se pot estima o limita minima si maxima pentru $ d $. 
Daca valoarea obtinuta nu se incadreaza in acest interval se considera ca punctul este invalid.
    \item Cu ajutorul geometriei analitice se obtine ecuatia dreptei care uneste $ mp $ 
si centrulul camerei. Aceasta dreapta si vectorul de directie al MapPoint-ului,
trebuie sa creeze un unghi cu o valoare mai mare de 60 de grade pentru a fi considerat $ mp $
observabil.
\end{enumerate} 
Daca aceste 3 verificari au fost realizate cu succes se considera ca punctul poate fi observat
de catre camera. Exista 2 functii responsabile de asocierile intre cadrele curente, scopul acestora 
este ca gaseasca corespondente intre Feature-urile din cadrul curent si MapPoint-urile 
din spatiu. Pentru a se gasi perechea Feature $ f $ si MapPoint $ mp $, trebuie ca
$ mp $ sa se proiecteze in vecinatatea $ f $ iar descriptorii asociati atat Feature-ului 
cat si al MapPoint-ului sa aiba distanta Hamming sub un prag, setat in aceasta implementare la 50.     
O metoda ar fi compararea tuturor Feature-urilor din spatiu, cu totalitatea MapPoint-urilor 
observate de cadrul curent. Dar aceasta metoda ar fi ineficienta. O alta abordare ar fi separarea
Feature-urilor in clustere in functie de distanta Hamming a descriptorilor, abordare stabila dar 
lenta si preferabil de utilizat cand nu ne putem baza pe estimarea matricei de pozitie a cadrului anterior. 
Iar cealalta abordare o reprezinta clusterizarea in functie de coordonatele in imagine 
ale Feature-ului. \\ 
Functie $ match\_frame\_reference\_frame $ implementeaza prima metoda. Aceasta primeste ca 
parametru 2 vectori de feature-uri calculati de libraria DBoW2, unul asociat cadrului curent, pentru     
care estimam matricea de pozitie si unul asociat cadrului anterior, pentru care cunoastem deja matricea de pozitie
si asocierile de tip (Feature, MapPoint). Elementele acestor vectori sunt de tip $ (int, vector\_descriptori) $.
Daca 2 astfel de perechi au prima valoare egala intre ele, inseamna ca cei 2 vectori de descriptori fac parte din 
acelasi cluster, conform arborelui din libraria DBOW2. Fiecare descriptor are asociata o instanta a clasei 
Feature. In cadrul anterior, instanta poate avea sau nu, un MapPoint corespondent. Daca exista acel
MapPoint se poate proiecta in imagine. Descriptorul intern al MapPoint-ului este comparat cu ceilalti 
descriptori din acelasi cluster din cadrul curent si se aplica testul de proportionalitate Lowe pentru a
garanta ca descriptorul cu distanta Hamming minima este cel mai bun.   
Functia $ match\_consecutive\_frames $ este mai simpla si implementeaza a doua metoda. MapPoint-ul din 
spatiu este proiectat in imagine si toate Feature-urile dintr-o zona circulara de raza de variabila
sunt considerati posibili candidati pentru a crea o asociere (Feature, MapPoint). Se calculeaza distanta 
Hamming intre descriptorul MapPoint-ului si cel al Feature-ului. Iar descriptorul cu distanta minima si 
mai mica decat un prag setat la 100 este considerat ca fiind cel mai potrivit. Feature-ul asociat acelui
descriptor, va pastra o referinta a MapPoint-ului. \\

Clasa MotionOnlyBA implementeaza in Ceres algoritmul Motion Only Bundle Adjustment,
primeste ca date de intrare KeyFrame-ul curent si returneaza matricea de pozitie optimizata.
Libraria lucreaza cu o notiune din C++ numita functori. Acestea sunt clase/structuri pentru
care s-a facut overload la operatorul $ () $. Clasa BundleError se afla din aceeasi categorie
si implementeaza functia de eroare obtinuta din proiectarea unui MapPoint si asocierea 
acestuia cu un Feature. Pentru a crea problema de optimizare, clasa ceres::Problem trebuie 
sa stie care parametrii trebuie optimizati si functia de eroare pe care trebuie sa o minimizeze.
In cazul acestui algoritm, singurul lucru care va fi modificat este matricea de pozitie a KeyFrame-ului
pe care o voi converti in forma $ se(3) $, transformand-o intr-un vector de 7 elemente. Iar pentru 
functia de eroare, nu voi scrie explicit ca este suma erorilor de proiectie. In schimb, voi initializa pentru
fiecare asociere de tip (Feature, MapPoint) cate un element al clasei BundleError. Algoritmul de 
optimizare implementat de libraria Ceres, va incerca in mod independent sa reduca valoarea erorii 
pentru fiecare pereche in parte, modificand pe rand vectorul pozitiei. Exista un motiv pentru 
care schimb modul in care este exprimata pozitia camerei, matricea de pozitie contine 2 componente:
matricea de rotatie $ R $ si un vector de translatie $ t $. Pentru $ t $ nu exista restrictii 
de modificare atata timp cat aceasta nu aduce modificari mari intre pozitiile a doua cadre consecutive,
orice mod in care ar varia parametrii acestui vector, in continuare semnificatia
lui de vector de translatie ramane nealterata, in aceasta situatie putem spune ca parametrii 
sunt alterati de catre libraria Ceres folosind $ EuclidianManifold $, mici modificari bazate pe
calcularea derivatelor partiale ale acestora din functia de eroare definita in BundleError, asemanator 
modului in care sunt modificati parametrii in retele neurale. Pentru matricea de rotatie $ R $ nu se 
mai poate aplica aceeasi logica. Aceasta trebuie sa faca parte din structura de tip grup numit 
$ SO(3) $, adica sa respecte egalitatea $ R * R^t = R^t * R = I $ si trebuie sa reprezinte o rotatie 
reala pe cele 3 axe. Alterarea aleatorie a parametrilor ar duce la o matrice invalida. Din aceasta
cauza, modificarea rotatiei trebuie facuta cu un anumit unghi
iar acest lucru se poate realiza printr-o inmultire de 2 matrici de rotatie valide. Din pacate nu 
exista implementare in forma matriceala pentru schimbarea unghiului de rotatie, dar este pentru
Quaternioni. Din aceasta cauza fac conversia din matrice de pozitie in vector din 
categoria $ se(3) $, iar pentru primii 4 parametrii asociati rotatiei, optimizarea lor se realizeaza
folosind $ QuaternionManifold $. Aceasta abordare rezolva problema instabilitatii numerice si 
garanteaza ca rezultatul operatiei de optimizare este un element valid in $ se(3) $, ce poate fi 
ulterior convertit in forma matriceala. In functie de categoria din care face parte Feature-ul, 
acesta este considerat monocular sau stereo. Functia de eroare implementata de clasa BundleError 
este identica pentru ambele, cu exceptia ca pentru punctele stereo, este verificata si distanta 
la care se afla punctul fata de valoarea la care a fost estimata de camera RGBD.\@ Pentru a preveni
instabilitatea cauzata de punctele de tip outlier, functia Huber descrisa in capitolul anterior este
folosita in calcularea finala a erorii de proiectie. In implementara oficiala realizata de g2o, 
agloritmul de optimizare este rulat de 4 ori, si dupa fiecare executie sunt eliminate punctele de tip 
outlier. Experimental, am observat ca etapa de optimizare cadru cu cadru este cea mai costisitoare 
operatie pe care o realizeaza algoritmul de ORB-SLAM2, executia acesteia de 4 ori, nu creste 
semnificativ acuratetea si reduce viteza de prelucrarea la aproximativ 5 cadre pe secunda, facandu-l 
nepotrivit pentru un sistem in timp real. Am observat ca obtin rezultate foarte bune, ruland o singura
data Motion Only Bundle Adjustment, urmat apoi de o etapa de eliminare a corelatiilor 
(Feature, MapPoint) de tip outlier. Daca mai putin de 3 asocieri raman, se considera ca algoritmul
a acumulat prea multe erori in urmarirea cadru cu cadru si trece intr-o stare de relocalizare. \\

Clasa Tracker realizeaza urmarirea traiectoriei cadru cu cadru. Aceasta integreaza 
fiecare dintre componentele definite anterior, si este responsabila de captarea cadrului curent,
transformarea acestuia in KeyFrame si luarea deciziei daca va fi salvat in Map pentru a completa 
harta mediului inconjurator. Pasii urmatori se executa pentru fiecare cadru in parte:
\begin{enumerate}
    \item Se creeaza KeyFrame-ul curent.
    \item Se estimeaza matricea de pozitie pe baza legii de miscare.
    \item Se realizeaza asocierea intre Feature-urile (puncte 2D) din cadru curent si 
MapPoint-urile observate de cadru anterior (puncte 3D)
    \item Pe baza asocierilor respective realizate anterior, se optimiza matricea de pozitie
a KeyFrame-ului curent, sunt eliminate asocierile de tip outlier
    \item Este proiectata harta locala pe cadrul curent, si se gasesc noi asocieri 
(Feature, MapPoint), se executa din nou aceeasi operatie de optimizare Motion Only Bundle Adjustment
    \item Este evaluat KeyFrame-ul curent, se verifica daca trebuie salvat in clasa Map. 
\end{enumerate}
Cadrul curent si matricea de adancime sunt citite de TumDatasetReader. In imaginea RGB 
se foloseste ORB pentru a extrage un vector de KeyPoint-uri si un vector de descriptori. 
Acestea sunt folosite pentru a initializa un obiect de tip KeyFrame. Pentru algoritmul ORB se 
foloseste o versiune modificata implementata in clasa ORBextractor si este conceputa sa extraga 
aproximativ 1000 de puncte cheie, acestea fiind distribuite cat mai egal pe suprafata imaginii. 
Daca un numar foarte mare de keypoint-uri s-ar obtine din aceeasi zona, acuratetea estimarii ar 
avea de suferit, pixelii din zonele aflate mai aproape de camera se misca cu o viteza mai mare 
decat cei aflati in departare, daca am considera doar punctele dintr-o anumita in zona in realizarea
estimarii, am obtinute variatii in miscare prea bruste / lente depenzind de locul unde s-au gasit 
majoritatea punctelor.  Parametrii setati pentru algoritmul ORB sunt urmatorii: 1000 de feature-uri,
factorul de scalare al imaginii este 1.2, exista maxim 8 nivele, si algoritmul FAST care face 
extragerea initiala de KeyPoint-uri sa ia in considerare zona respectiva daca diferenta de intensitate
intre pixeli este de la 20 in sus. Daca in schimb, zona este slab texturata atunci poate sa seteze
aceasta diferenta la 7, pentru a garanta ca vor fi gasite feature-uri chiar si in cele mai 
dezavantajoase zone din imagine. De-a lungul duratei de viata a algoritmului, clasa 
Tracker pastreaza 4 referinte de tip KeyFrame: cadrul curent care este analizat, 2 cadre imediat 
anterioare care vor fi folosite la estimarea pozitiei si ultimul cadru referinta care a fost creat.
Cadrul referinta este ultimul KeyFrame adaugat in Map si indica aproximativ in ce 
zona se afla camera si care MapPoint-uri ar trebui sa fie vizibile. Cadrele mai vechi care nu au 
fost salvate in Map au fost sterse pentru a reduce cantitatea de memorie folosita. Pentru ultimul
KeyFrame creat urmeaza etapa de estimare a pozitiei curente, aceasta se face pe baza legii 
de miscare, iar valorile matricii vor fi calculate folosindu-ne de cele 2 cadre salvate in Tracker.
Important aici de observat ca pentru primul cadru citit, pozitia acestuia este matricea identitate
4*4, acest lucru sugerand ca dispozitivul care inregistreaza mediul considera ca primul KeyFrame
este chiar originea sistemului de coordonate, iar toate matricile de pozitie viitoare sunt de fapt
transformari relative fata de origine. Primul KeyFrame va fi salvat intotdeauna in clasa Map si 
este utilizat pentru a initializa primele puncte de tip MapPoint: pentru toate Feature-urile de 
tip stereo din imagine, se vor creea puncte in spatiu. Din cauza acestui mod de initializare, 
ORB-SLAM2 este sensibil pana la aparitia urmatorului cadru cheie, estimarile facute de acesta 
in prima etapa fiind predispuse la erori. Uneori algoritmul isi pierde orientarea cu totul, fiind
necesara o etapa de relocalizarea, sau de reluare a executiei acestuia. ORB-SLAM3, implementeaza 
o metoda mult mai robusta de initializare, generand mai multe harti locale in situatia in care
urmarirea cadru cu cadru esueaza si le uneste intre ele in momentul in care recunoaste o zona pe 
care a vizitat-o deja. Dupa ce a fost create KeyFrame-ul si a fost facuta estimarea initiala a 
pozitiei, clasa OrbMatcher este folosita pentru a gasit corelatii intre Feature-uri si MapPoint-uri.
Alegerea metodei care indeplineste acest lucru fiind determinata de numarul de KeyFrame-uri 
create de la ultima relocalizare sau de la adaugarea unui nou cadru cheie in Map. Daca nu se 
vor gasi minim 15 asocieri, se va considera ca algoritmul si-a pierdut orientarea, altfel, 
asocierile respective vor fi utlizate de catre MotionOnlyBA pentru a realiza optimizarea pozitiei.
Perechile de tip outlier vor fi eliminate si noua pozitie a KeyFrame-ului va fi returnata. Daca 
vor ramane mai putin de 3 asocieri se va considera, din nou, ca algoritmul si-a pierdut orientarea.
In final, se foloseste clasa Map pentru a proiecta toate punctele din harta locala pe cadrul curent
iar asocierile gasite vor trece din nou printr-un proces de optimizare. Daca nu se gasesc minim
50 de perechi (Feature, MapPoint) inseamna ca urmarirea cadrului curent a esuat. Altfel se 
trece la etapa urmatoare si se va decide daca vom stoca in Map KeyFrame-ul curent. Acest lucru 
se va intampla daca urmatoarele conditii vor avea loc simultan.    
\begin{enumerate}
    \item au trecut mai mult de 30 de cadre de la ultimul KeyFrame adaugat in Map
    \item numarul de MapPoint-uri in cadrul curent este 25\% din numarul urmarit de cadrul de referinta
    \item cadrul curent are cel putin 70 de Feature-uri de tip stereo, cu distanta dintre centrul
camerei si punct este mai mica de 3.2 metri si urmareste cel putin 100 de MapPoint-uri
\end{enumerate}

Clasa LocalMapping este responsabila de optimizarea hartii algoritmului. Aceasta sterge/adauga KeyFrame-uri
si MapPoint-uri, iar la fiecare cadru cheie nou, realizeaza operatia de Local Bundle 
Adjustment. Aceasta metoda optimizeaza matricile de pozitie si toate MapPoint-urile vecinilor 
directi si cei de categoria a doua pentru KeyFrame-ul abia adaugat. In momentul in care 
thread-ul de Tracking considera ca un nou cadru cheie trebuie de adaugat in harta, se executa
metoda principala $ local\_map $, aceasta indeplineste urmatoarele operatii:
\begin{enumerate}
    \item Creeaza noi MapPoint-uri din primele 100 de Feature-uri de tip stereo, 
sortate in ordine crescatoare dupa distanta la care se afla acestea de centrul camerei  
    \item Adauga cadrul curent in graful de KeyFrame-uri stabilind vecinii directi ai
acestuia 
    \item Noile MapPoint-uri create sunt adaugate intr-o lista numita $ recently\_added $,
pentru a iesi din aceasta lista, punctele trebuie sa treaca un test care dovedeste ca 
nu sunt rezultatul unui Feature eronat detectat de catre algoritmul ORB, si ca pot fi 
folosite cu incredere
    \item Se executa operatia de $ culling $, punctele sunt verificate daca sunt valide iar
daca nu, memoria lor este eliberata.
    \item Se foloseste operatia de triangulare pentru a crea noi MapPoint-uri din 
Feature-urile care se potrivesc intre ele si fac parte din cadre cheie diferite.
    \item Se detecteaza entitatile de tip MapPoint care reprezinta acelasi punct
din spatiu, iar una dintre referinte este stearsa pentru creste coorenta hartii si 
a creste ponderea conexiunii dintre KeyFrame-urile adiacente    
    \item Se executa operatia de KeyFrame culling, se verifica daca informatiile pe 
care le detine un KeyFrame, adica totalitatea valorilor de tip MapPoint pe care le 
detine, sunt observate si din alte cadre. Daca peste 90\% din punctele observate de un 
anumit cadru sunt vizibile si din alte cadre, KeyFrame-ul analizat este considerat redundant
si memoria lui este eliberata. Acest lucru garanteaza ca graful clasei Map, contine 
doar cadre esentiale pentru reprezentarea norului de puncte. 
\end{enumerate}
In etapa a 4-a se executa operatia de $ culling $, aceasta elimina punctele care nu sunt
de incredere. Singurele puncte care nu vor trece prin aceasta etapa de verificare sunt
cele generate de primul KeyFrame, tot primul KeyFrame nu poate fi sters deoarece 
ar da peste cap sistemul de coordonate local sub care lucreaza ORB-SLAM2. Un punct 
este considerat de incredere daca din momentul in care a fost creat, el a fost 
observat in 3 cadre cheie consecutive si daca a fost observat in cel putin 25\% din 
numarul total de cadre care au trecut de la creearea acestuia. Ambele conditii 
trebuie sa fie respectate simultan in momentul in care se face verificarea punctului
respectiv. Politica pe care o urmeaza familia de algoritmi ORB-SLAM este sa genereze 
multe puncte, fara a impune restrictii, pe care apoi le va supune acestui test de 
relevanta. \\

Ultima clasa este cea de MapDrawer pe care o folosesc pentru a afisa norul de MapPoint-uri,
cadrul curent analizat si pozitiile cadrelor cheie observate. Folosesc libraria Pangolin si 
OpenGL pentru desenarea fiecarei structuri, camera urmareste cadrul curent. Interfata grafica 
scade viteza de procesare a cadrelor dar este o modalitate eficienta de a intelege vizual ce 
se petrece in algoritm. Implementarea pentru interfata grafica am realizat-o spre final, cand 
aveam celelalte componente finalizate, lucru care a ingreunat procesul de dezvoltare deoarece 
lucram cu valori numerice in terminal. Acum daca as reincepe implementarea, interfata grafica ar fi
printre primele lucruri pe care le-as realiza. Datorita acestei clase am reusit sa gasesc erori in
modul de constructie al grafului ponderat din clasa Map si al modului in care proiectam punctele 
in spatiu.

\section{Pipeline antrenare FastDepth}

Pentru reteaua Neurala FastDepth pipeline-ul de antrenare a fost scris folosind libraria Pytorch
iar pentru operatiile de preprocesare folosesc libraria Albumentations. Setul de date pe care 
am facut antrenarea se numeste Nyu Depthv2 Dataset\cite{nyudataset} si l-am obtinut de pe Kaggle. Rezultatul 
acestui pipeline trebuie sa fie un fisier de tip ONNX cu valorile parametrilor retelei FastDepth 
in urma antrenarii pe setul de date. O problema pe care am observat-o la setul de date este ca 
pentru imaginile de antrenament, adancimile sunt exprimate ca fiind in intervalul $ [0, 255] $, pe cand in setul 
de date de validare, acestea se afla intre $ [0, 10000] $ reprezentand valorile in milimetrii ale
distantelor. O limitarea acestui set de date este ca nu poate detecta distante mai mari de 10 metri.
Dar considerand ca algoritmul trebuie sa functioneze pentru incaperi de mici dimensiuni, consider ca 
aceasta distanta maxima nu ar trebui sa reprezinte o problema. Pentru antrenare am ales sa urmez lucrarea
stiintifica\cite{fastdepth} si am setat hiperparametrii:
\begin{itemize}
    \item Optimizatorul folosit a fost implementarea din Pytorch pentru Stochastic Gradient Descent, 
torch.SGD, avand un learning rate de $ 1e-3 $, o valoare a momentumului de $ \beta=0.9 $ si
$ weight\_decay=1e-4 $.
    \item antrenarea s-a realizat pentru 50 de epoci iar durata antrenarii a fost de aproximativ 
6 ore jumatate. Laptopul pe care am antrenat este un Asus TUF Gaming A15, avand un procesor AMD Ryzen 7 cu o frecventa 
de 4.2 GHz si placa video NVIDIA GeForce RTX 2060, cu o memorie de 6GB.\@
    \item Imaginile in setul de date au o dimensiune de $ (3, 460, 640) $. Pentru a creste viteza de procesare
am modificat dimensiunile la $ (3, 256, 320) $ si am aplicat o functie de normalizare de tip min\_max. Ambele 
transformari sunt aplicate atat pe setul de date de antrenare cat si pe cel de test.
    \item un batch de date are dimensiune de 8
    \item In lucrarea FastDepth functia de pierdere folosita este L1Loss, aceasta fiind suma diferentelor dintre valoarea 
reala si cea determinata de reteaua neurala in modul. In implementarea mea am ales sa folosesc o functie de pierdere 
mai robusta conform acestei lucrari stiintifice\cite{lossfunctionidea}. 
\end{itemize}
Acuratetea a fost verificata  prin compararea diferentei relative intre valorile obtinute prin inferenta si cele 
reale cu un factor $ RELATIVE\_ERROR=0.15 $. Aceasta operatie a fost realizata pentru fiecare pixel in parte, iar
acuratetea reprezinta procentul de pixeli cu o valoare care se incadreaza in limita impusa de RELATIVE\_ERROR.\@
Pentru a preveni antrenarea pentru intervale lungi fara a obtine rezultate, am avut 2 metode pe care le-am implementat:
o strategia de early stopping: in situatia in care valoarea acuratetii nu ar fi crescut pentru 5 epoci antrenarea 
ar fi fost oprita si o strategie pentru modificarea learning rate-ului in timpul antrenarii. Daca acuratetea nu crestea 
pentru 3 epoci valoarea parametrului sa fie redusa la 0.3 din valoarea initiala. In practica am observat ca reteaua 
converge aproape monoton catre o valoare optima. Functia de pierdere primeste ca date de intrare matricea de adancime 
obtinuta de catre reteaua neurala si matricea cu valori reala din setul de date, denumita groundtruth, si returneaza o valoarea 
numerica de tip double care exprima cat de departe se afla estimarea noastra de realitate. Ideea antrenarii unei retele 
neurale este minimizarea acestei valori. Functia de eroare este alcatuita dintr-o combinatie liniara a 
3 componente diferite\cite{lossfunctionidea}: L1Loss, GradientEdgeLoss si Structural Similarity Loss, formula matematica este:
\begin{equation}
    loss = 0.6 \cdot L1Loss + 0.2 \cdot GradientEdgeLoss + StructuralSimilarityLoss 
\end{equation}  
Structural Similary Loss se asigura ca media si distributia standard pe care o urmeaza valorile estimate, se apropie
de media si distributia standard a matricei groundtruth. In comparatie cu celelalte 2 componente ale functiei de 
pierdere care sunt aplicate la nivel de pixel, aceasta abstractizeaza rezultatele ca fiind 2 distributii Normale 
cu parametrii $ \mathcal{N}(\mu, \sigma^2) $ care trebuie sa se suprapuna.                     
Principiul de functionare pentru GradientEdgeLoss este ca pixeli din regiuni apropiate trebuie sa aiba cam aceleasi 
valori de estimare ale distantei si ca diferenta intre pixeli adiacenti pe axele x si y, ar trebui sa fie identica
cu cea din imaginea groundtruth. Aceasta poate fi scrisa in felul urmator, unde $ N $ reprezinta numarul de pixeli din 
imagine, iar derivata valorilor pixelilor in raport cu axa de coordonate reprezinta diferenta intre matricea imaginii
initiale si aceeasi matrice avand un rand shiftat la dreapta  pentru axa X notata $ \frac{\partial I}{\partial x} $
si un rand shiftat vertical pentru axa Y notata $ \frac{\partial I}{\partial y} $.  
\begin{equation}
    L_{\text{edges}} = \frac{1}{N} \sum_{i=1}^{N} \left( 
    \left| \frac{\partial I_{\text{pred}}}{\partial x} - \frac{\partial I_{\text{true}}}{\partial x} \right| + 
    \left| \frac{\partial I_{\text{pred}}}{\partial y} - \frac{\partial I_{\text{true}}}{\partial y} \right|
    \right)
\end{equation}

\chapter{Evaluare}
\section{Setul de date TUM RGBD Dataset}
Setul de date utilizat pentru a realiza evaluarea se numeste TUM RGBD Dataset\cite{tum}. 
Acesta contine numeroase subseturi, fiecare verificand un aspect diferit al implementarii, 
ajutand la creearea unui imagini de ansamblu cu privire la robustetea algoritmului in functie 
de mediu in care se lucreaza si de traiectoria pe care o urmeaza. Cele 2 subseturi pe 
care le-am considerat potrivite pentru implementarea sunt:
\begin{itemize}
    \item Subsetul rgbd\_dataset\_freiburg1\_xyz contine cadrele unui video de 35 de secunde, in care
traiectoria este in principal alcatuita din translatii, exista foarte putine rotatii fiind
ideal pentru a verifica daca estimarea pozitiei in spatiu este corect realizata.    
    \item  Subsetul rgbd\_dataset\_freiburg1\_rpy are 27 de secunde si contine 
foarte putine translatii. Exista in schimb numeroase schimbari bruste de rotatie care reduc
acuratetea imaginii captate, testand la maxim capacitatea algoritmului ORB de a extrage 
feature-uri. Sistemul isi schimba orientarea pe toate cele 3 axe, fiind unul dintre cele
mai dificile subseturi de date pe care se poate face antrenarea. Algoritmul 
ORB-SLAM2 este sensibil la operatiile de rotatie, mai ales atunci cand camera isi schimba 
orientarea catre o zona necunoscuta. Pentru a crea harta zonei respective sunt generate 
numeroase KeyFrame-uri si MapPoint-uri, pe care algoritmul trebuie sa le filtreze in clasa
de LocalMapping, lucru care creste complexitatea temporala si spatiala si scade acuratetea
sistemului.  
\end{itemize}   
Videourile sunt realizate 
cu ajutorul unei camere RGBD Microsoft Kinetic, avand frecventa de 30 de cadre pe secunda, 
setul de date contine imaginile de tip RGB, hartile de adancime pentru fiecare cadru in parte,
vectorii de pozitie in forma $ se(3) $, primii 3 parametrii fiind pozitia in spatiu 
$ (tx, ty, tz) $ iar urmatorii 4 parametrii sunt asociati matricei de rotatie, scrisa sub
forma de Quaternion, $ (qw, qx, qy, qz) $ si timestamp-urile asociate momentului in care 
au fost inregistrate fiecare din valorile din setul de date. Cu ajutorul acestor timestammp-uri
putem crea asocieri de tip (imagine RGB, matrice de adancime, pozitie) pe care le putem 
transmite algoritmul ORB-SLAM2. Pozitia este considerata ca fiind valoarea ideala, groundtruth,
si va fi comparata cu rezultatele obtinute. Clasa TumDatasetReader este responsabila de citirea
datelor si stocarea matricilor de pozitie obtinute pentru fiecare cadru. Dupa parcurgerea 
intregului set de date, valorile estimate sunt salvate intr-un fisier de tip text unde vor 
fi comparate cu cele reale.
\section{Metrici utilizate}
Algoritmul ORB-SLAM2 scrie intr-un fisier estimarile matricilor de pozitie pentru fiecare 
cadru in parte. Pentru a realiza comparatia cu datele de tip groundtruth din setul de 
date, folosesc un pachet din python numit $ evo $, acesta este capabil sa creeze un grafic
al traiectoriei, permitand astfel o reprezentare vizuala a rezultatelor si o separare a 
acestora in functie de ceea ce vreau sa evaluez: viteza, translatia sau orientarea. De
exemplu, figura de mai jos reprezinta variatia translatie pe fiecare dintre cele 3 axe. 
Cu albastru este valoarea de tip groundtruth iar cu galben este estimarea realizata de 
implementarea mea pentru algoritmul ORB-SLAM.\@ Rezultatele sunt obtinute pentru subsetul 
de date rgbd\_dataset\_freiburg1\_xyz, acesta fiind special conceput pentru a testa 
corectitudinea estimarii translatiei intre cadre.

\begin{figure}[htbp] 
  \centering
  \includegraphics[width=1.0\textwidth]{./images/variate_translatie.png}
  \caption{Graficul translatiei pe fiecare din cele 3 axe, groundtruth si estimare ORB-SLAM2}
  \label{fig:exemplu_imagine}
\end{figure}
Consider ca o reprezentare grafica in care traiectoria groundtruth se suprapune exact cu
ceeea ce a obtinut estimarea ORB-SLAM2 poate fi considerata in mod neoficial, o metrica
pe baza careia sa putem spune daca algoritmul functioneaza corect. Pentru exactitate 
se poate folosi: APE (Absolute Pose Error), aceasta metrica măsoară distanța euclidiană 
dintre pozițiile estimate și cele reale, la fiecare moment de timp.
In general valorile scorurilor APE obtinute pentru ambele seturi de date sunt pana in 0.05,
sugerand ca acuratetea este buna. Alte metrici pe care le folosesc pentru implementarea 
mea a algoritmului ORB-SLAM2 este numarul de secunde necesar pentru parcurgerea setului de 
date sau numarul de cadre pe secunda. Numarul de KeyFrame-uri create, in momentul in care 
sistemul nu mai poate realiza urmarirea cadru cu cadru, acesta insereaza un nou KeyFrame,
cu cat sunt mai putine KeyFrame-uri noi adaugate, se poate considera ca traiectoria este 
usor de interpretat si ca sistemul este stabil. O alta metrica este legata de numarul 
de relocalizari pe care a trebuit sa le faca algoritmul pentru a parcurge setul de date.
Relocalizarea apare in situatia in care urmarirea cadru cu cadru esueaza si este cautat 
KeyFrame-ul care seamana cel mai bine cu cadrul curent folosind vectorul de feature-uri 
calculat de metoda bag-of-words. Ideal, numarul necesar de relocalizari ar trebui sa fie 0.\\
In cazul rularii implementarii mele pe setul de date rgbd\_dataset\_freiburg1\_xyz acesta 
dureaza in medie 67 de secude, functionand la aproximativ 15 cadre pe secunda, in medie 
este nevoie intre 5{-}7 Keyframe-uri noi pentru parcurgerea setului de date. Logica 
de relocalizare nu este deloc folosita, algoritmul fiind capabil sa realizeze urmarirea
cadru cu cadru.
Pe setul de date rgbd\_dataset\_freiburg1\_rpy a durat 73 de secunde, videoul avand 
27 de secunde, reprezinta aproximativ 10{-}11 cadre pe secunde. Acest lucru se datoreaza 
numeroaselor operatii de optimizare a hartii pe care trebuie sa le faca algoritmul deoarece 
sunt adaugate intre 17{-}19 KeyFrame-uri pentru a parcurge intreg setul de date, din cauza 
miscarilor bruste ale camerei care reduc considerabil claritatea imaginilor extrase.
In continuare, numarul de relocalizari este 0. Mai jos, am atasat graficul care compara 
estimarea orientarii pentru fiecare cadru, estimarile fiind descompuse dupa cele 3 dimensiuni
ale rotatiei. Graficul realizat de algoritmul ORB-SLAM2 pare sa fie shiftat in timp fata 
de cel real, dar sa aiba aproximativ aceeasi forma cu cel al valorilor de tip groundtruth.
Consider ca problema poate sa porneasca de la modul in care sunt atasate timestamp-urile 
pentru fiecare cadru in parte, lucru care nu are legatura directa cu modul in care este
realizata implementarea, ci cu modul in care setul de date creeaza perechile (imagine RGB, 
vector pozitie, matrice de adancime).      
\begin{figure}[htbp] 
  \centering
  \includegraphics[width=1.0\textwidth]{./images/rpy_variatie_orientare.png}
  \caption{Graficul orientarii pentru setul de date rgbd\_dataset\_freiburg1\_rpy}
  \label{fig:exemplu_imagine}
\end{figure}
Am incercat utilizarea retelei neurale FastDepth pentru a estima adancimea in loc de a
folosi matricea de distante a setului de date TUM RGBD.\@ Pentru a testa arhitectura voi
lua doar imaginea si vectorul de pozitie simuland astfel o situatie in care sistemul ar avea
doar o camera RGB.\@ Problema cu aceasta implementare este ca reteaua neurala nu este suficient
de exacta iar estimarile distantelor intre cadre consecutive in continuare variaza mult. Pentru 
subsetul de date rgbd\_dataset\_freiburg1\_xyz implementarea intra in etapa de relocalizare
in medie dupa primele 50{-}60 de cadre procesate. Sistemul este mult prea instabil pentru
a realiza in mod corect urmarirea cadru cu cadru. \\
In etapele intiale ale dezvoltarii algoritmului am incercat utilizarea unui video realizat 
folosind camera telefonului pentru testarea implementarii. Au fost 3 probleme pe care le-am 
intalnit: nu puteam determina parametrii corecti ai camerei telefonului. Aveam nevoie 
de distanta focala, de coordonatele centrului imaginii si de parametrii de distorsiune.
A doua problema o reprezenta lipsa unei matrici de adancime pentru fiecare cadru iar cea de-a
treia, era lipsa un vector de pozitie pentru fiecare imagine. Chiar 
si in situatia in care obtineam parametrii camerei telefonului folosind algoritmi implementati
in OpenCV, in continuare nu puteam fi sigur daca estimarile realizate 
de mine sunt cele corecte. Din cauza acestor multe probleme, am ajuns la concluzia ca un set 
de date ar fi o varianta mai potrivita.

\chapter{Concluzii}
O problema pe care am avut-o in testarea algoritmului a fost ca nu am putut face 
functionala implementarea initiala a ORB-SLAM2, exista conflicte intre versiunile de
biblioteci Eigen si g2o. Versiunea de Eigen folosita la momentul respectiv nu mai 
exista acum in repo-ul oficial si eu nu am reusit sa o accesez pentru a testa.\\ 
Libraria Ceres este usor de folosit pentru problemele de optimizare non-liniare, 
consider ca cel mai mare plus pe care il aduce lucrarea mea este ca am cea mai 
completa implementare a algoritmului Bundle Adjustment in aceasta librarie la 
care se adauga o logica de filtrare a punctelor de tip outlier si are caz 
separat de folosire atat pentru punctele monoculare cat si pentru cele stereo.   
In plus am adus optimizari la codul oficial pentru ORB-SLAM2: implementarea lor 
nu elibereaza absolut deloc memoria pentru MapPoint-urile si KeyFrame-urile 
considerate invalide, eu am rezolvat aceasta problema, extinzand durata pentru 
care poate rula algoritmul si facandu-l potrivit pentru sistemele embedded.
De asemenea, in implementarea oficiala nu este folosita nicio structura de tip
dictionar, clasele principale folosite au parametrii de stare care isi modifica
valoarea la fiecare cadru, functiile avand efecte laterale care genereaza erori 
greu de urmarit si corectat. In total am scris 4030 de linii de cod in C++ si 
am modificat codul pentru numeroase functii, facandu-l mai usor de inteles si 
mentinut. ORB-SLAM2 indeplineste 3 functii: urmarirea cadru cu cadru, corectarea 
erorilor, si inchiderea buclelor. Etapa de inchidere a buclelor nu am reusit sa o 
implementez din cauza apropierii termenului de predare, cu toate acestea, algoritmul
obtine in continuare rezultate bune pentru subseturile de date alese. \\
In ciuda faptului ca utilizarea unei retele neurale pentru a inlocui o camera 
RGBD nu a functionat asa cum am crezut initial, algoritmul devine instabil si 
isi pierde complet orientarea dupa primele 50 de cadre, consider ca o retea 
neurala precum FastDepth poate fi folosita in estimarea adancimii pentru 
punctele care au avut atribuita distanta 0 de catre camera tip RGBD.\@   
O directie viitoare ar fi utilizarea unui sistem care combina cele 2 
abordari. Separarea straturilor convolutionale in depthwise si pointwise s-a dovedit
a fi o tehnica buna pentru a creste viteza arhitecturii astfel incat sa poata fi
folosita in timp real. \\
Un lucru pe care il regret este ca nu am utilizat o interfata grafica inca de la 
primele etape, pentru a detecta erorile in timpul dezvoltarii algoritmului. Desi scorul
calculat de APE este o metrica buna pentru a vedea cat de bine se potrivesc 2 
estimari ale pozitiei unui cadru, o simpla valoare numerica nu este suficient
pentru a avea o privire generala asupra modului in care functioneaza implementarea. \\
Ca directii viitoare, ma gandeam sa utilizez arhitecturi de retele neurale pentru
sarcini bine delimitate, cum ar fi extragerea de feature-uri sau gasirea de corelatii
de tip (Feature, MapPoint), adaugarea unui modul de object detection si a unui 
algoritm de planificare de trasee, pentru a indeplini sarcini simple de gasire 
a unor obiecte de mici dimensiuni. Pana acum algoritmul a folosit doar metode
clasice, ORB pentru extragere de feature-uri, Brute Force pentru 
matching, bag-of-words pentru relocalizare. Acum, consider ca directia pe care ar
trebui sa o urmeze aceasta clasa de algoritmi de tip SLAM este una in care 
tehnici de Machine Learning sunt folosite pentru a creste viteza si poate 
acuratetea operatiilor. Aceasta fiind si directia incurajata de lucrarile ce 
fac parte din state of the art pana la momentul curent.

\printbibliography
 


\end{document}