\documentclass[12pt,a4paper]{report}
% chktex-file 44
\usepackage[utf8]{inputenc} % pentru suport diacritice
\usepackage[romanian]{babel} % setări pentru limba română 
\renewcommand\familydefault{\sfdefault} % sans serif

\usepackage[margin=2.54cm]{geometry}	% dimensiuni pagină și margini
\usepackage{graphicx} % support the \includegraphics command and options
\usepackage{amsmath}

% formatting sections and subsections
\usepackage{textcase}
\usepackage[titletoc, title]{appendix}
\usepackage{titlesec}
\titleformat{\chapter}{\large\bfseries\MakeUppercase}{\thechapter}{2ex}{}[\vspace*{-1.5cm}]
\titleformat*{\section}{\large\bfseries}
\titleformat*{\subsection}{\large\bfseries}
\titleformat*{\subsubsection}{\large\bfseries}

\usepackage{chngcntr}
\counterwithout{figure}{chapter} % no chapter number in figure labels
\counterwithout{table}{chapter} % no chapter number in table labels
\counterwithout{equation}{chapter} % no chapter number in equation labels

\usepackage{booktabs} % for much better looking tables
\usepackage{url} % Useful for inserting web links nicely
\usepackage[bookmarks,unicode,hidelinks]{hyperref}

\usepackage{array} % for better arrays (eg matrices) in maths
\usepackage{paralist} % very flexible & customisable lists (eg. enumerate/itemize, etc.)
\usepackage{verbatim} % adds environment for commenting out blocks of text & for better verbatim
\usepackage{subfig} % make it possible to include more than one captioned figure/table in a single float
\usepackage{enumitem}
\setlist{noitemsep}

%%% HEADERS & FOOTERS
\usepackage{fancyhdr}
\pagestyle{empty}
\renewcommand{\headrulewidth}{0pt}
\renewcommand{\footrulewidth}{0pt}
\lhead{}\chead{}\rhead{}
\lfoot{}\cfoot{\thepage}\rfoot{}


\newcommand{\HeaderLineSpace}{-0.25cm}
\newcommand{\UniTextRO}{UNIVERSITATEA NATIONALA DE STIINTA SI TEHNOLOGIE POLITEHNICA BUCURESTI \\[\HeaderLineSpace] 
FACULTATEA DE AUTOMATICĂ ȘI CALCULATOARE \\[\HeaderLineSpace]
DEPARTAMENTUL DE CALCULATOARE\\}
\newcommand{\DiplomaRO}{PROIECT DE DIPLOMA}
\newcommand{\AdvisorRO}{Coordonator științific:}
\newcommand{\BucRO}{BUCUREȘTI}

\newcommand{\UniTextEN}{NATIONAL UNIVERSITY OF SCIENCE AND TECHNOLOGY POLITEHNICA BUCHAREST \\[\HeaderLineSpace]
FACULTY OF AUTOMATIC CONTROL AND COMPUTERS \\[\HeaderLineSpace]
COMPUTER SCIENCE AND ENGINEERING DEPARTMENT\\}
\newcommand{\DiplomaEN}{DIPLOMA PROJECT}
\newcommand{\AdvisorEN}{Thesis advisor:}
\newcommand{\BucEN}{BUCHAREST}
\newcommand{\ProjectTitleRO}{Orientare in Spatiu folosind ORB-SLAM}
\newcommand{\ProjectTitleEN}{Spatial Orientation using ORB-SLAM}

\newcommand{\frontPage}[6]{
\begin{titlepage}
\begin{center}
{\Large #1}  % header (university, faculty, department)
\vspace{50pt}
\vspace{105pt}
{\Huge #2}\\  % diploma project text
\vspace{40pt}
{\Large #3}\\ \vspace{0pt}  % project title
{\Large #4}\\     % project subtitle
\vspace{40pt}
{\LARGE Alfred Andrei Pietraru}\\   % student name
\end{center}
\vspace{60pt}
\begin{tabular*}{\textwidth}{@{\extracolsep{\fill}}p{6cm}r}
&{\large\textbf{#5}}\vspace{10pt}\\      % scientific advisor
&{\large Prof.\ dr.\ ing.\ Anca Morar}   % advisor name
\end{tabular*}
\vspace{20pt}
\begin{center}
{\large\textbf{#6}}\\    % bucharest
\vspace{0pt}
{\normalsize 2025}
\end{center}
\end{titlepage}
}

\newcommand{\frontPageRO}{\frontPage{\UniTextRO}{\DiplomaRO}{\ProjectTitleRO}{\BucRO}{\AdvisorRO}}
\newcommand{\frontPageEN}{\frontPage{\UniTextEN}{\DiplomaEN}{\ProjectTitleEN}{\BucEN}{\AdvisorEN}}

\linespread{1.5}
\setlength\parindent{0pt}
\setlength\parskip{8pt}
\include{1_abstract}
%%%%%%%%%%%%%%%%%%%%%%%%%%%%%%%%%%%%%%%%%%%%%%%%%%   
%%
%%          End of template definitions
%%   
%%%%%%%%%%%%%%%%%%%%%%%%%%%%%%%%%%%%%%%%%%%%%%%%%%
\begin{document}
\frontPageRO{}
\frontPageEN{}
% \begingroup
% \linespread{1}
% \tableofcontents
% \endgroup
\AbstractPage{}
\chapter{Introducere}
SLAM, Simultanious Localization and Mapping reprezinta o clasa de algoritmi de planificare si control 
al miscarii unui agent prin mediu pentru a construi un model al spatiului cat mai apropiat de realitate.
In timpul functionarii sunt indeplinite 3 functii: agentul trebuie sa creeze o harta, sa isi 
cunoasca la orice moment de timp pozitia in spatiu si sa fie capabil sa controleze modul in care se 
deplaseaza. In literatura de specialitate exista 2 categorii mari pentru sistemele de tip SLAM, acestea 
sunt active SLAM, in care robotul ia decizii online, alegand sa se deplaseze astfel incat sa maximizeze 
acuratetea hartii pe care o creeaza pentru mediul inconjurator si SLAM pasiv, in care sistemul doar 
observa mediul si creaza harta pe baza acestor observatii, acesta nu ia parte in procesul de planificare
al traseului pe care il va parcurge agentul.

\chapter{Cerinte si Motivatie}
\section{Motivatie}
Filmele, cartile si jocurile pe calculator ne prezinta un viitor al omenirii in care
roboti inteligenti indeplinesc sarcinile plictisitoare din viata de zi cu zi, sau cele 
care ar putea pune in pericol siguranta omului. Exista zeci de filme care descriu 
acest fenomen complex de robot inteligent, capabil sa se adapteze la mediu si sa 
interactioneze cu omul. Desi in momentul de fata suntem departe de a crea un framework 
suficient de complex pentru un asemenea agent, consider ca suntem pe drumul cel bun. 
Imi este greu sa imi imaginez un robot care sa poata simula compartamentul uman si sa 
nu fie capabil sa se deplaseze si sa inteleaga mediul in care se afla. Pentru noi, 
aceste lucruri sunt adanc inradacinate in modul in care functioneaza creierul, dar pentru
un calculator, a fost nevoie de aproape 20 de ani de cercetare pentru a creea algoritmi 
suficienti de complexi pentru a indeplini niste sarcini minimale de orientare cum ar 
capacitatea de invatare a mediului si de pozitionare a agentului in spatiu, pe cand 
noi realizam aceste lucruri intuitiv. Chiar si cu algoritmii de tip SLAM dezvoltati 
pana in acest moment, exista numeroase aplicatii practice: 
\begin{itemize}
    \item pentru sarcini din viata de zi cu zi, cazul robotilor de curatenie sau a celor
care transporta obiecte in interiorul unei cladiri 
    \item in aplicatii medicale, ca de exemplu asistenta pentru persoanele nevazatoare
    \item in aplicatii militare: cartografierea zonelor necunoscute: in interiorul
cladirilor sau medii ostile in care nu exista suport pentru un sistem de coordonate
global cum ar fi GPS  
    \item in aplicatii industriale, inspectii asupra instalatiei sau depozitelor, 
detectarea unor erori si raportarea zonei in care au fost observate    
\end{itemize}
Exista numeroase aplicatii pentru sistemele SLAM, doar toate pleaca de la prinicipiul 
ca agentul trebuie sa creeze o harta a mediului si sa inteleaga care este pozitia acestuia.
Pe masura ce aceste sisteme vor evolua si vor incepe sa utilizeze tehnologii din alte 
domenii cum ar fi tehnici de machine learning vor creste si complexitatea sarcinilor pe 
care le pot indeplini.   

\section{Cerinte Functionale}
Algoritmul ORB-SLAM trebuie sa primeasca un video realizat cu o camera de tip RGBD si sa 
returneze un fisier text cu estimarea pozitiei pentru fiecare cadru in parte. De asemenea
acesta va salva in alt fisier text harta mediului inconjurator, aceasta fiind alcatuita 
dintr-un nor de puncte in spatiu si cadrele cheie asociate acestora. Algoritmul va avea 
o interfata grafica minimala alcatuita din 2 ferestre. In prima va fi aratat folosind 
culoarea albastra care este pozitia cadrului curent procesat. Totalitatea 
cadrelor cheie observate vor fi reprezentate folosind verde, si cu rosu punctele din 
spatiu, harta mediului inconjurator. In cea de-a doua fereastra va fi aratat fiecare 
cadru in format alb negru, iar cu rosu vor fi marcate feature-urile detectate de 
algoritmul ORB.\@ Cadrele cheie consecutive sunt conectate intre ele prin intermediul
unei drepte de culoare neagra, acestea fiind folosite impreuna pentru a recompune 
traseul realizat de camera in video.             


cerinte functionale: returneaza harta mediului creata, estimarea pozitiei la fiecare 
momennt de timp, arata grafic ceea ce se petrece   
cerinte non-functionale: viteza, capacitate de corectare erori, optimizare harta, capacitate
de relocalizare  

\chapter{Studiu de piata {-} variante Visual SLAM}
Sistemul nostru nu are senzori si detine doar o camera de tip RGBD / sau RGB, are cel mult 2
core-uri de CPU si nu poate folosi GPU-ul pentru a accelera viteza de procesare. Acesta trebuie sa 
functioneze in timp real, sa poate procesa cel putin 10 {-} 15 cadre pe secunda, sa poata opera in
intr-un mediu indoor de mici dimensiuni, ca de exemplu un apartament si static, elementele care alcatuiesc
mediul nu isi modifica pozitia, sa poata fi integrat intr-un
sistem embedded, sa fie capabil sa memoreze zonele din mediu prin care a trecut, sa construiasca o harta a
mediului minimala, de exemplu printr-un nor de puncte cu o densitate scazuta, sa aiba capacitatea de 
corectie a erorilor si sa poata fi folosit pentru o perioada indelungata. Consideram ca sistemul nu 
are acces la coordonatele globale ale pozitiei sale si nu poate folosi tehnologii precum GPS-ul.
Sistemul nostru trebuie sa faca parte din categoria de Visual SLAM si sa foloseasca o camera de tip 
RGBD sau RGB combinata cu o retea neurala capabila sa fie folosita de CPU pentru sarcini de inferenta 
si sa functioneze in timp real. Cei mai noi algoritmi de visual SLAM functioneaza acum folosind tehnici
de deep learning. In continuare vor fi prezentate lucrarile care au reprezentat SOA, pana la inceputul 
anului 2025, grupate in categorii in functie de modul in care sunt folosite tehnicile de deep learning.
Am considerat potrivita impartirea pe 3 nivele a algoritmilor, in functie gradul de utilizare 
al tehnicilor de deep learning pentru realizarea operatiile specifice sistemelor SLAM:\@
\begin{enumerate}
    \item Algoritmi care se bazeaza fundamental pe tehnici de deep learning pentru a functiona, 
DPV-SLAM, ESLAM.\@
    \item Algoritmi care sunt la granita dintre metodele clasice si cele deep learning, in care 
doar anumite componente sunt imbunatatite cu ajutorul retelelor neurale: Light-SLAM, HFNet-SLAM.\@
    \item Algoritmii clasici, care nu folosesc deloc retele neurale: ORB-SLAM3, SVO.\@ 
\end{enumerate}   
Deep Patch Visual SLAM (DPV-SLAM), este un sistem SLAM care foloseste deep neural networks.
Acesta imparte operatiile care trebuie realizate in 2 categorii: frontend-ul care realizeaza 
sarcina de visual odometry cu ajutorul unui sistem derivat din Deep Patch Visual Odometry (DPVO)
si partea de backend alcatuita din 2 metode de loop closure: proximity loop closure si classical
loop closure. Algoritmul are nevoie intre 5{-}6 GB de memorie pe GPU pentru a putea rula. 
O alta problema o reprezinta proximity loop closure. Este o metoda rapida daca se foloseste un
singur GPU, dar aceasta functioneaza cu ajutorul unei harti foarte dense de feature-uri obtinute
cu ajutorul metodei de optical flow, fiind greu de adaptat la cerintele noastre. \\
ESLAM sau Efficient Dense Visual SLAM using Neural Implicit Maps este un sistem de SLAM 
monocameră RGB-D care folosește o combinație între o harta densa 3D, reprezentata de o retea 
neurala implicita si un backend optimizat geometric pentru estimarea matricei de pozitie a camerei.
Avantajele acestei implementari sunt ca produce o harta densa si detaliata si poate reconstrui 
detalii chiar si in zone partial observate. Problema acestei implementari este ca necesita un 
GPU si resurse mari de calcul si nu este potrivit pentru dispozitivele embedded. \\
Light-SLAM este construit pornind de la aceeasi filozofie ca si ORB-SLAM2, partea de backend
reprezentata de local mapping, adica optimizarea hartii create si loop closure, recunoasterea
zonelor prin care a mai trecut algoritmul si inchiderea buclelor traiectoriei, acestea sunt 
realizate folosind metode clasice. Extragerea de keypoint-uri, descriptori si sarcina de 
matching intre descriptorii a doua imagini consecutive este realizata de 2 retele neurale.
Acest sistem poate functiona in timp real daca se poate folosi un GPU, dar cea mai mare 
problema o reprezinta faptul ca retele neurale nu sunt capabile sa gaseasca feature-uri cu
acuratete suficient de buna in zone care nu seamana cu ceea ce a  intalnit in setul de date
 de antrenare, astfel algoritmul nu are garantia ca va functiona in situatii critice.\\
HFNet-SLAM este o metoda construita pe baza ORB-SLAM3 si folosindu-se de arhitectura HF-Net, 
avand acelasi principiu pe baza caruia a fost facuta ca si Mobile\_Net, avand straturile de convolutie
separate in depthwise convolution si pointwise convolution. In loc sa foloseasca 2 retele neurale 
precum Light-SLAM, aceasta foloseste una singura, atat pentru extragere keypoint-urilor si a
descriptorilor cat si pentru feature-urile globale, folosite in sarcinile de loop closure. 
Pe langa problema retelei neurale care trebuie sa ruleze pe GPU si  a feature-urilor instabile 
extrase din imagini pentru zone care nu au fost intalnite in setul de antrenare, algoritmul
calculeaza pentru fiecare cadru in parte feature-urile sale globale lucru care adauga un 
overhead computational deloc necesar, iar keypoint-urile nu sunt extrase avand o piramida de 
nivele. Din aceasta cauza, acelasi keypoint observat in 2 cadre diferite, dar la dimensiuni 
diferite nu va putea fi asociat observat corespunzator, facand sistemul instabil la deplasarea 
in linie dreapta.  \\
ORB-SLAM3 este continuare implementarii algoritmului ORB-SLAM2 pe care l-am ales eu.
Acesta a aparut in 2021 si pana in acest moment este cea mai complexa si completa metoda de a
estima traiectoria camerei si a reconstrui o harta de puncte a mediului inconjurator folosind 
doar metode clasice. In comparatie cu precedesorul acestuia, implementarea de ORB-SLAM3 foloseste
datele obtinute de la Inertial Measurement Unit (IMU) si optimizeaza rezultatele primite 
folosind tehnica de Maximum a Posteriori Estimation (MAP). Algoritmul lucreaza de asemenea cu un 
sistem in care sunt generate noi harti de fiecare data cand se pierde capacitatea de urmarire 
cadru cu cadru, iar acestea sunt ulterior unite intre ele cand agentul ajunge intr-o zona pe 
care o cunoaste deja. Am considerat ca un mediu de mici dimensiuni in interior nu ar avea 
nevoie de un sistem atat de complex, iar utlizarea acestuia ar adauga un overhead nejustificat.\\
SVO sau Semi-Direct Visual Odometry for Monocular and Multi-Camera Systems este un exemplu 
de algoritm tip SLAM care foloseste doar 2 thread-uri: unul responsabil de 
urmarirea cadru cu cadru si celalalt pentru optimizarea hartii. Acesta foloseste gradientii
pixelilor in imagini pentru a crea feature-uri, nu doar contururile obiectelor. In cazul
algoritmilor din familia ORB-SLAM care incearca sa optimizeze eroarea de proiectie a punctelor
din spatiu, aici se folosesc metode directe, si trebuie minimizata eroarea fotometrica a 
pixelilor aflati in apropierea contururilor obiectelor. Este printre cei mai rapizi algoritmi
de SLAM, procesand peste 100 de cadre pe secunda pe un CPU, dar genereaza o harta cu mult 
prea putine puncte care rareori poate fi refolosite, nu exista capacitate de relocalizare si 
este dificil de extins.

In ciuda faptului ca algoritmul ORB-SLAM2 a aparut in 2017, in continuare ramane
un exemplu de sistem bine gandit, cu multe posibilitati de extindere si capacitate de a fi 
adaptat la cerintele din zilele noastre. Indeplineste cu succes toate criteriile pe care sistemul
dezvoltat ar trebui sa le auba: poate fi folosit in real time, implementarea procesand 15 cadre
pe secunda, creaza harta mediului inconjurator are capacitate de relocalizare si corecteaza 
erorile de estimare care apar in timp prin mecanismul de loop closure. Acesta poate rula exclusiv
pe CPU, fiind potrivit atat pentru vehicule la sol, dar si pentru drone. Nu are nevoie de o 
estimare a pozitiei globale, putand fi folosit in medii ostile in care nu exista acces la GPS 
iar terenul este complet necunoscut.  


\chapter{Solutie propusa}
Solutia mea presupune implementarea algoritmului ORB-SLAM2. Acesta are 2 scopuri 
fundamentale:
\begin{itemize}
    \item sa estimeze pentru fiecare cadru in parte matricea de pozitie si orientare
a camerei, reconstruind astfel traseul parcurs in timpul functionarii algoritmului  
    \item sa creeze o harta locala a mediului inconjurator pentru a memora zonele
prin care a mai trecut si pentru a imbunatatii estimarea traiectoriei  
\end{itemize}
Matricea de pozitie si orientare a camerei (pose matrix) are dimensiuni $ 4 \times 4 $ si are 
formatul prezentat mai jos, unde R reprezinta matricea de rotatie $ 3 \times 3 $, iar t este
vectorul coloana de dimensiune 3, reprezentand translatia fata de punctul de origine
\( (0, 0, 0)\). Aceasta mai este denumita si matricea de conversie din sistemul de 
coordonate global (world space) in sistemul de coordonate al camerei (camera space) si 
este notata in implementarea mea ca \( T_{cw} \). Inversa acestei matrice notata
\(T_{wc} \) realizeaza operatia de conversie dintre cele 2 sisteme de coordonate 
in sens opus.

\begin{equation}
T_{cw} = 
\begin{bmatrix}
R \hspace{0.5cm} t \\
0 \hspace{0.5cm} 1
\end{bmatrix}
\end{equation}

Algoritmul primeste ca date de intrare: sursa de la care va obtine imaginile de tip
RGB pe care va trebui sa le prelucreze, acestea pot sa provina atat de la un video,
set de date consacrat sau chiar in timp real direct de la camera, parametrii de 
distorsiune a imaginii si matricea parametrilor interni ai camerei, avand dimensiunea 
$ 3 \times 3 $ si notata in mod traditional cu \(K\). Aceasta contine 4 constante importante: 
distanta focala a camerei pe axa x si pe y \(f_x\), \(f_y\) si \(c_x\), \(c_y\) \
reprezentand coordonatele centrului imaginii. Aceasta matrice trebuie modificata de 
fiecare data cand este schimbata camera cu care se realizeaza filmarea, sau cand se 
fac operatii de modificare a dimensiunii imaginilor fata de modul in care ar fi 
acestea extrase natural. Matricea are urmatoarea forma:
   
\begin{equation}
    K = 
    \begin{bmatrix}
        f_x & 0   & c_x \\
        0   & f_y & c_y \\
        0   & 0   & 1
    \end{bmatrix}
\end{equation}

Algoritmul va returna un fisier text in care se vor afla estimarile matricilor 
de pozitie impreuna cu timestamp-ul asociat pentru fiecare cadru in parte in ordine
cronologica. Rezultatul poate fi comparat cu fisiere care contin valorile reale si 
care respecta acelasi format pentru a verifica corectitudinea algoritmului.   
Diagrama UML prezinta interactiunea dintre 
principale componente descrise din punct de vedere functional, dar si flow-ul natural
al algoritmului. In continuare voi detalia logica fiecarei componente din punct de 
vedere al algoritmilor folositi, ale valorilor de intrare si de iesire ale acestora. 

\section{Achizitia datelor}
Scopul acestei componente este sa citeasca imaginea de tip RGB de la camera,          
sa extraga matricea de adancime asociata cadrului curent, de exemplu: prin            
intermediul unei camere stereo, a unei camere de tip RGBD sau cu ajutorul unei retele 
neurale si sa creeze o estimare initiala pentru pozitia curenta a camerei pe baza     
masuratorilor anterioare. In viitor, o alta functie a acestei componente ar putea fi  
extragerea datelor de la instrumente de masura precum giroscop sau accelerometru,     
pentru a obtine informatii suplimentare cu privire la orientarea si distanta efectuata
de catre camera care ar putea imbunatatii considerabil estimarea initiala a pozitiei.

\section{Extragere trasaturi} 
Ca date de intrare aceasta componenta primeste doar imaginea de tip RGB, si extrage
aproximativ 1000 de trasaturi si descriptori asociati acestora. Trasaturile sunt zone
de interes in imagine care pot fi folosite pentru a detecta obiecte sau gasi asocieri 
intre cadrele consecutive. Acestea mai sunt numite si keypoint-uri in literatura de 
specialitate iar librarii precum OpenCV au structuri de date dedicate pentru acestea.
O trasatura poate fi interpretata matematic ca o zona in care apare o schimbare 
brusca a gradientului culorii. De cele mai multe ori, astfel de variatii se regasesc 
in zonele de frontiera dintre obiecte, deoarce apare o diferenta
de culoare si implicit una de intensitate luminoasa. Zonele slab texturate, cum ar 
fi cerul sau peretii in interiorul unei cladiri nu au zone care sa poata fi usor de 
comparat, deoarece zonele de pe suprafata respectiva arata similar si este greu de 
estimat de unde a fost extras keypoint-ul respectiv. In schimb, o camera 
complet mobilata ar fi o zona puternic texturata iar un algoritm de detectie de 
keypoint-uri ar putea sa gaseasca usor 1000 de trasaturi pe care sa le foloseasca.
Daca algoritmul nu reuseste sa gaseasca suficiente keypoint-uri pentru a face urmarirea
intre cadre, de obicei minim 500, urmarirea cadru cu cadru nu poate continua. Din aceasta
cauza algoritmul de ORB-SLAM2 da rezultate eronate in zonele slab texturate. Daca algoritmul
de extragere functioneaza corect iar traiectoria camerei este una stabila, fara schimbari
bruste ale directiei de deplasare, trasaturi similare ar trebui sa fie observate in ambele
imagini. Asocierile dintre ele, ne pot da informatii despre modul in care s-a deplasat 
camera intre cele 2 cadre. Problema este ca aceste keypoint-uri nu pot fi comparate direct
intre ele din aceasta cauza ne folosim de descriptori. Acestia sunt vectori de diferite 
dimensiuni care trebuie sa surprinda informatia esentiala observata in zona respectiva din
imagine, in mod ideal descriptorii ar trebui sa ramana invariabili la operatiile de 
redimensionare si rotatie aplicate pe keypoint-uri. 
Algoritmului Oriented Fast and Rotated Brief (ORB) este folosit pentru extragerea de 
keypoint-uri si descriptori. A fost creat in anul 2011 ca alternativa pentru alti algoritmi
de extragere de feature-uri precum SIFT si SURF.\@ Motivul pentru care acesta a ajuns atat de popular 
se datoreaza mai multor factori:
\begin{itemize}
    \item Este mult mai rapid decat SIFT si SURF fiind mult mai potrivit pentru sisteme
in timp real si pentru dispozitive embedded.
    \item la momentul realizarii lucrarii stiintifice despre ORB-SLAM2 atat SIFT cat 
si SURF se aflau sub protectia drepturilor de autor, ORB nu avea vreo astfel de restrictie
    \item ORB este invariant din punct de vedere al rotatiei   
    \item Foloseste descriptori binari, care pot fi usor de comparat folosind distanta Hamming,
    aceasta converteste in biti vectorul de elemente obtinute, 
\end{itemize}
Implementarea algoritmului ORB poate fi separata in 2 componente, calcularea KeyPoint-urilor si 
cea a descriptorilor. Pasii pe care ii urmeaza algoritmul sunt urmatorii, care se repeta 
de cate ori a fost setat ca imaginea initiala sa fie redimensionata:
\begin{enumerate}
    \item Calcularea keypoint-urilor folosind algoritmul FAST-9.
    \item Selectarea celor mai potrivite keypoint-uri folosind Harris Corner Measure, Trasaturile
sunt sortate in ordine descrescatoare si sunt selectate primele N cele mai potrivite
    \item Pentru fiecare keypoint se calculeaza orientarea acestuia folosind intensitatea
centroidului, dupa aceasta operatie avem toate informatiile necesare despre keypoint-uri.
    \item Pentru calcularea descriptorilor vom face o operatie de smoothing pentru fiecare 
zona de $ 31 \times 31 $ de pixeli, folosind un kernel cu o dimensiune de $ 5 \times 5 $.
    \item se calculeaza descriptorii de tip steer BIREF, avand pozitia modificata dupa unghiul
dat de orientare. Operatia de modificare a orientarii duce la o variatie redusa a valorilor 
bitilor din descriptori si la o corelatie puternica intre acestia, facand descriptorii ineficienti
    \item Se obtine rBRIEF o varianta optimizata a algoritmul steer BRIEF, prin alegerea bitilor
despre care se stie ca au varianta mare si grad scazut de corelatie intre ei.
\end{enumerate}   
In cazul algoritmul FAST-9, 
cifra 9 vine de la diametrul ferestrei circulare in care se face compararea intre valoarea 
intensitatii pixelului si centru. Acest algoritm primeste ca parametru imaginea si pragul 
pe care trebuie sa il depaseasca diferenta de intensitate intre pixeli pentru a fi considerat
un keypoint. De cele mai multe ori, informatia data de keypoint-uri este redundanta, pentru
a selecta un numar restrans de trasaturi, de preferat cele mai expresive, se foloseste Harris
Corner Measure.  Pentru a calcula orientarea vom defini notiunea de centroid  $ C $ care este
diferit de centrul zonei din imagine $ O $. Vectorul $ \vec{OC} $ va fi cel 
care va da unghiul $ \theta $ al keypoint-ului pe care il vom obtine direct din urmatoarea 
formula, unde $ I(x, y) $ reprezinta intensitatea luminoasa a pixelului cu coordonate $ (x, y) $.      
\begin{equation}
m_{pq} = \sum_{x, y} x^p y^q I(x, y), \quad{}
\theta = \text{atan2}(m_{01}, m_{10})
\end{equation}
In etapele 5 si 6 se realizeaza calcularea descriptorilor: acestia vor avea forma binara si o 
lungime finala de 256 de biti. Compararea lor se va realiza folosind distanta Hamming, cu cat 
2 descriptori au o valoarea mai mica a acestei distante, cu atat mai similari sunt. Valorile 
descriptorilor sunt asociate pe baza unui test binar in care este comparata intensitatea a 2 
puncte din planul imaginii. Problema este ca descriptorii BRIEF sunt sensibili la schimbarile de
rotatie, din aceasta cauza, prin rotirea coordonatelor pixelilor cu unghiul $ \theta $ al 
orientarii se obtine steered BRIEF.\@ Pentru a obtine rBRIEF, au fost invatate in offline la
creearea algoritmului ORB care teste de verificare a intensitatii au cea mai mare variatie, si
primele 256 dintre acestea au fost alese pentru a alcatui descriptorul.       

\section{Harta punctelor din spatiu}
Unul dintre scopurile fundamentale ale algoritmului de ORB-SLAM2, pe langa cel 
de estimare al traseului camerei este cel de creare a hartii locale a mediului
inconjurator. Problema este ca, in comparatie cu versiuni mai avansate ale
acestui algoritm, special modificate pentru o reconstructie cat mai fidela a mediului,
algoritmul nostru trebuie sa functioneze pentru un sistem embedded care nu are 
capacitate de procesare suficient de mare, fiind nevoit astfel sa simuleze mediul 
printr-un nor de puncte cu o densitate redusa (sparse). Cele 2 sarcini sunt 
dependente una de cealalta, fiecare element din norul de puncte actioneaza ca
o referinta, o caracteristica a mediului care ar trebui sa fie observata de fiecare 
data cand punctul se afla in frustum-ul camerei. De exemplu: presupunem ca avem
o imagine in care este observata in totalitate o masa in interiorul unei incaperi. 
ORB va identifica aproape instantaneu feature-urile (colturile mesei) si teoretic,
indiferent de modul in care ne-am roti in jurul mesei, aceleasi feature-uri
ar trebui sa fie observate de fiecare data, mai mult de atat, considerand ca mediul 
este static, acestea sunt mereu asociate cu acelasi punct din spatiu, devenind astfel 
o referinta pe baza careia putem estima modul in care s-ar deplasa camera. 
In literatura de specialitate aceste puncte din spatiu sunt  denumite MapPoint-uri 
iar functionalitatea corecta a algoritmului depinde strict de 
modul in care aceste MapPoint-uri sunt observate cadru cu cadru. Un astfel de punct 
in spatiu este creat dintr-un keypoint, dar nu vom avea nevoie de toate punctele din 
regiunea respectiva si vom considera ca centrul este punctul cel mai semnificativ,  
avand coordonate \(x\) si \(y\), si distanta fata de camera fiind estimata ca fiind 
\(d\). Mai mult ne vom folosi de matricea transformarii din coordonatele camerei
in coordonatele globale si de parametrii interni ai camerei \(f_x\), \(f_y\) 
distanta focala, si \(c_x\), \(c_y\) coordonatele centrului imaginii.
Vectorul coloana cu 3 dimensiuni reprezinta pozitia in spatiu a feature-ului gasit
in cadrul curent pe care il analizam, astfel am creat primul MapPoint. Ca alternativa,
pentru a nu lucra cu matrici de dimensiuni $ 4 \times 4 $ putem folosi \(R_{wc}\) reprezentand
matricea de rotatie si \(t_{wc}\) vectorul de translatie.

\begin{equation} 
\begin{bmatrix}
X \\
Y \\
Z \\
1
\end{bmatrix} = T_{wc} *  
\begin{bmatrix}
\frac{x - c_x}{f_x} * d \\
\frac{y- c_y}{f_y} * d \\
d \\
1
\end{bmatrix}, \quad{}
\begin{bmatrix}
    X \\
    Y \\
    Z 
\end{bmatrix} = R_{wc} *
\begin{bmatrix}
    \frac{x - c_x}{f_x} * d \\
    \frac{y- c_y}{f_y} * d \\
    d
    \end{bmatrix} + t_{wc}
\end{equation}

In literatura de specialitate MapPoint-urile sunt considerate ca fiind 
niste ancore (landmark) pozitionate dinamic de catre algoritm, acestea sunt asociate
cu un anumit cadru cheie si ne vor ajuta in optimizarea matricei de pozitie dar si
pentru sarcina de relocalizare si de memorare a zonelor cunoscute.

\section{Asociere puncte din spatiu cu feature-uri ORB}
Ca date de intrare avem feature-urile si descriptorii extrasi din imagine, matricea
de adancime si harta de MapPoint-uri. Scopul acestei componente este sa gaseasca 
cat mai multe asocieri de 1:1 intre feature-uri si MapPoint-uri. Intr-un caz ideal
fiecare feature gasit ar trebui sa aiba asociat un MapPoint, dar in realitate nu se 
poate intampla acest lucru din 2 motive: 
imperfectiuni ale algoritmului ORB de detectie ale feature-urilor: acesta nu garanteaza 
ca acelasi feature va fi gasit de fiecare data pentru cadre consecutive si faptul ca 
modelul isi schimba orientarea, facand ca MapPoint-urile aflate la limita campului 
vizual al camerei sa nu mai poata fi observate. Un MapPoint este un feature 
al unui cadru anterior, proiectat in spatiu. In final, aceasta componenta realizeaza 
tot o comparare de feature-uri intre cadrul curent, si multiple cadre anterioare.
Aceasta operatie de comparare se realizeaza prin intermediul distantei Hamming dintre
descriptori, cu cat valoarea obtinuta este mai mica, cu atat cele 2 feature-uri sunt 
mai asemanatoare. Exista mai multe tipuri de algoritmi folositi pentru feature 
matching, dar cel folosit in implementarea curenta este Brute Force Feature Matching
optimizat. Acest algoritm primeste ca date de intrare 2 seturi de feature-uri si 
incearca sa gaseasca asocieri intre ele. Asocierele sunt facute cu ajutorul descriptorilor,
se calculeaza distanta Hamming iar daca valoarea obtinuta este minima, perechea 
respectiva de feature-uri se considera ca a fost corect asociata, pentru ORB-SLAM2 
acest lucru reprezinta ca am gasit exact acelasi punct din spatiu, in 2 imagini diferite.
Daca \(N\) este numarul de feature-uri din primul set, \(M\) numarul de feature-uri din 
al doilea set si \(D\) fiind dimensiunea descriptorului, in cazul nostru fiind 32, complexitatea
algoritmului devine \(O(N * M * D)\). Facand-ul un algoritm destul de costisitor de folosit
pentru un sistem in timp real, mai mult de atat, este predispus la erori, compararea 
feature-urilor nu tine cont de locatia acestora in imagine, obtinandu-se astfel asocieri
care matematic par corecte, dar ele nu au sens din punct de vedere logic. Pentru a rezolva
aceasta problema si a reduce complexitatea temporala se stabileste o fereastra patrata de 
lungime prestabilita in jurul punctului de proiectie unde se pot cauta feature-uri.
In final se obtin asocierile intre feature-uri, sunt cautate MapPoint-urile 
corespunzatoare feature-urilor din cadrele anterioare si altfel se obtin asocierele 
(feature, MapPoint) de care are nevoie algoritmul.      

\section{Optimizare Estimare Pozitie Initiala}
Aceasta componenta primeste ca data de intrare estimarea pozitiei curente a camerei
\(T_{cw}\), si o asociere bijectiva intre feature-urile gasite in imagine si punctele 
care exista la momentul respectiv in spatiu. Ca date de iesire vom avea doar matricea 
pozitiei curente a camerei optimizata. Daca asocierile intre feature-uri si MapPoint-uri
sunt perfecte, ar trebui ca proiectia punctului din spatiu pe imagine sa se suprapuna pe 
centrul keypoint-ului. Rareori se petrece acest lucru in practica, iar distanta dintre
proiectia unui MapPoint si coordonotale centrului feature-ului reprezinta eroarea de 
asociere. Pentru a minimiza aceasta eroare, exista 2 optimizari care se pot face:
prima este modificarea valorilor matricei de pozitiei, iar cea de-a doua este modificarea 
coordonatelor din spatiu ale MapPoint-ului. Inainte de a prezenta algoritmul de optimizare
folosit, voi arata modul in care se proiecteaza un MapPoint in plan.
\subsection{Proiectarea MapPoint in planul imaginii}
Aceasta operatie de proiectie poate fi vazuta ca aplicarea unui functii $ \pi(\cdot) $
ce primeste ca date de intrare coordonatele globale ale punctului, iar ca rezultat va
returna coordonatele omogene in planul imaginii. Aceasta transformare se petrece in 
2 etape:
\begin{enumerate}
    \item conversia din sistemul de coordonate globale in sistemul de coordonate al camerei
    \item conversia din sistemul de coordonate al camerei in sistemul de coordonate al imaginii
\end{enumerate}
In prima etapa putem folosi coordonatele omogene, pentru a face conversia in mod direct.
Alternativ, putem extrage din matricea de pozitie \(T_{cw}\) atat matricea de 
rotatie \(R_{cw}\) cat si vectorul coloana de translatie \(t_{cw}\).

\begin{equation}
\mathbf{X}_{camera} = \mathbf{T}_{cw} \cdot 
\begin{bmatrix}
\mathbf{X}_w \\
1
\end{bmatrix}, \quad
\mathbf{T}_{cw} =
\begin{bmatrix}
\mathbf{R_{cw}} & \mathbf{t_{cw}} \\
\mathbf{0}^T & 1
\end{bmatrix}, \quad
\mathbf{X}_{camera} = \mathbf{R_{cw}} \cdot \mathbf{X_w} + \mathbf{t_{cw}}
\end{equation}

Matricea \(T_{cw}\) este utilizata atat pentru a descrie pozitia si orientarea in spatiu 
cat si pentru a schimba din sistemul de coordonate global in cel al camerei.
In sistemul de coordonate global, un punct se afla la exact aceeasi valoare indiferent 
de pozitia camerei care il priveste, in sistemul de coordonate al camerei, pozitia unui 
MapPoint o sa difere de fiecare data.
In etapa a doua MapPoint-ul respectiv este in sistemul de referinta al camerei, 
coordonatele fiind reprezentate prin vectorul coloana \(X_{camera}\). Vom considera 
a 3-a valoare a acestui vector \(Z_c\). Aceasta reprezinta distanta dintre planul camerei
si punctul pe care il analizam. \(Z_c\) ne spune daca punctul respectiv poate fi observat
in imagine. Daca valoarea \(Z_c\) este mai mica sau egala cu 0, inseamna ca punctul 
se proiecteaza in spatele camerei, facandu-l invalid. In situatia in care \(Z_c\) este 
mai mare decat 0, vom realiza conversia in coordonatele omogene ale imaginii cu ajutorul
urmatoarei formule, \(u\) fiind asociat axei x si \(v\) fiind asociat axei y. Daca 
valorile \(u\) si \(\v\) au valori mai mari ca 0, si mai mici decat dimensiunea imaginii.
Vectorul coloana \([u, v, 1]\) este rezultatul cautat. 

\begin{equation}
    \begin{bmatrix}
        u \\
        v \\
        1
        \end{bmatrix}
        =
        \mathbf{K} \cdot
        \begin{bmatrix}
        \frac{X_c}{Z_c} \\
        \frac{Y_c}{Z_c} \\
        1 
        \end{bmatrix}, \quad{}
        \mathbf{K} =
        \begin{bmatrix}
        f_x & 0 & c_x \\
        0 & f_y & c_y \\
        0 & 0 & 1
        \end{bmatrix}
\end{equation} 


\subsection{Motion Only Bundle Adjustment}
Algoritmul folosit in aceasta etapa se numeste Motion Only Bundle Adjustment. Acesta 
modifica doar matricea pozitiei curente a camerei. Coordonatele punctelor din spatiu sunt
considerate ca fiind constante. Algoritmul este unul iterativ, minimizand o functie de cost.
Forma generala a functiei de cost este suma erorilor de proiectie pentru toate perechile
(feature, MapPoint). Iar formula generala este aceasta.
 \begin{equation}
    \mathbf{R}_{cw}, \mathbf{t_{cw}} = \min_{\mathbf{R}_{cw}, t_{cw}} \sum_{i=1}^{N} \rho\left( \left\| \mathbf{x}_i - K \cdot \left( \mathbf{R}_{cw} \cdot \mathbf{X}_i + \mathbf{t_{cw}} \right) \right\|^2 \right)
\end{equation}

In aceasta formula, \(x_i\) reprezinta coordonatele omogene feature-ului in coordonatele imagininii,
iar \(X_i\) reprezinta coordonatele globale ale MapPoint-ului pentru care calculam eroarea
de proiectie. Simbolul $ \rho(\cdot) $ reprezinta functia Huber pentru scalarea valorilor de eroare.
Daca o asociere intre un feature si un MapPoint nu este potrivita, diferenta dintre centrul feature-ului
si proiectia MapPoint-ului este mai mare decat un prag prestabilit. Aceasta diferenta, lasata nemodificata,
ar destabiliza algoritmul. Iar o astfel de problema este usor de observat, daca modificarea matricei
de pozitie duce la variatii enorme a orientarii sau a translatiei intre 2 cadre consecutive, 
atunci cel mai probabil asocierile intre feature-uri si MapPoint-uri aveau valori eronate, 
termenul de \(outlier\) fiind folosit in aceasta situatie. Functia Huber Loss reduce 
valoarea acestor outlier-ere permitandu-le in acelasi timp sa faca parte din algoritmul de 
optimizare. In acest fel algoritmul devine mai robust si capabil ajunga la o valoare optima 
in mai putine iteratii. Mai jos este prezentata formula matematica a functiei Huber Loss, 
unde $ \delta $ reprezinta un numar real pozitiv, toleranta a erorii de proiectie. 

\begin{equation}
\rho(s) =
\begin{cases}
\frac{1}{2}s^2 & \text{if } |s| \leq \delta \\
\delta (|s| - \frac{1}{2}\delta) & \text{if } |s| > \delta
\end{cases}
\end{equation}

In urma executiei algoritmului obtinem matricea de pozitie optimizata, mai mult de atat,
stim care dintre perechile (feature, MapPoint) au avut statutul de outlier si le putem 
elimina pentru a nu influenta in mod negativ functionalitatea algoritmului.

\section{Crearea unui cadru cheie}
Aceasta componenta primeste ca date de intrare absolut toate informatiile procesate de 
pana acum pentru cadrul curent: imaginea de tip rgb, matricea de adancime, punctele cheie,
descriptorii, asocierile (feature, MapPoint) si estimarea matricii pozitiei curente a camerei.
Toate aceste data impreuna vor alcatui un cadru cheie care va fi salvat in memorie, pentru 
termen scurt sau lung. Exista 2 motive principale pentru care dorim sa facem aceasta operatie
In primul rand ne ajuta sa estimam pozitia urmatorului cadru bazandu-ne pe legea inertiei. 
Consider ca o data inceputa deplasarea camerei intr-o anumita directie, este foarte 
probabil ca aceea miscare sa fie mentinuta si la urmatorul cadru. Fie \(T_{cw}\) matricea de 
pozitie pentru cadrul la care vrem sa estimam deplasarea, iar \(T_{cw1}, T_{cw2} \) matricile 
de pozitie a celor 2 cadre imediat predecesoare. Formula de estimare a pozitiei curente este:

\begin{equation}
\mathbf{T_{cw}} = \mathbf{T_{cw1}} \cdot \left( \mathbf{T_{cw2}}^{-1} \cdot \mathbf{T_{cw1}} \right)
\end{equation} 

Al doilea motiv pentru care avem nevoie de cadre cheie este recreearea mediului. Incercam sa salvam
numarul minim de cadre necesare pentru a reproduce harta de MapPoint-uri a mediului inconjurator.
Un cadru cheie nou (KeyFrame) aduce cu sine MapPoint-uri noi, extrase din feature-urile gasite in 
imaginea respectiva. Functionarea corecta a urmarii cadru cu cadru, este determinata de numarul de 
MapPoint-uri gasite in imaginea curenta in comparatie cu un cadru de referinta. In momentul 
in care numarul de puncte cheie gasite in imaginea curenta scade sub un anumit prag, stim ca este
necesar un nou cadru cheie care: sa stabilizeze urmarirea, sa introduca noi puncte cheie, si sa 
ajute la optimizarea intregii harti a mediului. 

\subsection{Optimizare harta locala}
Harta locala este alcatuita din KeyFrame-uri si MapPoint-uri, avand intre ele o relatie de many-to-many.
Se considera ca un KeyFrame si un MapPoint sunt conectate intre ele daca un MapPoint dat poate sa 
fie observat dintr-un KeyFrame. Matematic se traduce ca proiectia punctului respectiv pe imaginea
stocata in KeyFrame poate fi asociat cu un feature valid. Avem acelasi exemplu cu masa dintr-o incapere. 
Mai multe cadre consecutive observa acelasi colt al piesei de mobilier. Acel feature are
asociat un MapPoint, ceeea ce inseamna ca MapPoint-ul respectiv este observat din mai multe 
KeyFrame-uri. Cu cat mai multe Keyframe-uri observa acelasi MapPoint, cu atat mai stabil este 
punctul respectiv din spatiu. Intr-un caz ideal, ar trebui ca orice MapPoint creat sa fie 
stabil. De cele mai multe ori nu se intampla acest lucru din cauza erorilor de feature matching.
Astfel, doar cele mai evidente feature-uri raman salvate in harta pana la finalul algoritmului.
Pentru a salva Keyframe-ul curent in harta trebuie sa se petreaca in ordine urmatoarele operatii:
\begin{enumerate}
    \item Exista feature-uri in cadrul curent care nu au fost asociate cu un MapPoint.
    Folosindu-ne de harta de adancime si de coordonatele fiecarui punct cheie din imagine selectam 
    cele mai apropiate n astfel de puncte si le proiectam in spatiu pentru a obtine noi MapPoint-uri.      
    \item KeyFrame-ul curent este comparat cu alte cadre cheie, pentru a vedea cu cine imparte 
    cele mai multe puncte comune. Keyframe-urile sunt stocate in harta intr-o structura de tip 
    graf neorientat unde nodurile sunt cadrele cheie iar arcele sunt numarul de MapPoint-uri comune 
    dintre ele.
    \item sunt eliminate punctele cheie redundante sau care au fost observate in prea putine 
    cadre pentru a fi luate in considerare.
    \item se executa un algoritm numit Bundle Adjustment pentru a optimiza atat 
    pozitiile punctelor din spatiu dar si matricea de pozitie a cadrelor cheie
\end{enumerate} 
 
Bundle Adjustment este similar cu Motion Only Bundle Adjustment. In continuare vorbim de un
algoritm iterativ care incearca sa minimizeze o functie de cost, folosind metoda scaderii 
gradientului. Diferenta in aceasta situatie este ca Bundle Adjustment se aplica pe mai mult 
de un cadru si atat matricea poztiei si punctele din spatiu (MapPoint-urile) vor fi optimizate.
Avem urmatoarele etape:
\begin{enumerate}
    \item Se creeaza lista de cadre mobile. Plecand de la cadrul curent, se vor selecta
    toti vecinii de gradul 1 si 2 din graful neorientat stocat in harta. Aceste Keyframe-uri
    sunt considerate ca fiind \(mobile\) deoarece matricea lor de pozitie se va modifica.  
    \item Se creeaza lista de MapPoint-uri care vor fi optimizate. Fiecare Keyframe din multimea 
    cadrelor mobile observa un numar de puncte in spatiu, toate aceste puncte vor fi folosite 
    de catre algoritmul de optimizare.
    \item Se creeaza lista de cadre fixe, pentru acestea, matricea de pozitie nu se va modifica.
    Avand lista de MapPoint-uri ce vor fi optimizate, pentru fiecare punct din spatiu
    vom itera prin lista de KeyFrame-uri care observa acel MapPoint. Daca un KeyFrame apartine 
    multimii de cadre mobile in vom ignora, iar daca nu, vom considera ca face parte din lista de
    cadre fixe. Acestea sunt incluse in algoritm pentru a garanta ca modificarea coordonatelor
    MapPoint-ului nu va strica asociera (feature, MapPoint) in cadrele care nu vor avea matricea
    de pozitie modificata. 
\end{enumerate}

Lucrarea stiintifica care sta la baza ORB-SLAM2, implementeaza deja functia de 
cost pe care algoritmul de Bundle Adjustment o foloseste. Pentru a intelege mai usor formula 
matematica, aceasta trebuie privita de la dreapta la stanga. \(E_{kj}\) reprezinta eroarea
de proiectie a MapPoint-ului pe feature-ul asociat. Indicele \(k\) este asociat KeyFrame-ului,
iar \(j\) este indicele perechii (feature, MapPoint) pentru care calculam eroarea. Simbolul 
$ \rho(\cdot) $ este asociat functiei Huber, folosita pentru a ameliora efectele perechilor de tip 
outlier. \(X_k\) reprezinta multimea tuturor asocierilor (feature, MapPoint) pentru un 
Keyframe \(k\). Aceasta suma de erori este calculata pentru fiecare cadru, atat cele fixe cat 
si cele mobile. Parametrii care trebuiesc optimizati sunt: coordonatele MapPoint-urilor 
selectate de catre algoritm \(X_i\) cat si matricile de pozitie pentru cadrele mobile. Algoritmul 
returneaza noile valori ale parametrilor care trebuie optimizati.  

\begin{equation}
\{\mathbf{X}_i, \mathbf{R}_l, \mathbf{t}_l \mid i \in \mathcal{P}_L, l \in \mathcal{K}_L\} = \arg \min_{\mathbf{X}^i, \mathbf{R}_l, \mathbf{t}_l} \sum_{k \in \mathcal{K}_L \cup \mathcal{K}_F} \sum_{j \in \mathcal{X}_k} \rho(E_{kj})
\end{equation}
\begin{equation}  
E_{kj} = \left\| \mathbf{x}_{j} - K \cdot \left(\mathbf{R}_k \mathbf{X}_j + \mathbf{t}_k\right) \right\|
\end{equation}

\subsection{Reteaua Neurala FastDepth}
Retele Neurale Artificiale sunt o tehnica des intalnita in Machine Learning pentru a rezolva 
sarcini complexe pentru care nu exista solutii algoritmice clar definite sau implementarea acestora 
este mult prea costisitoare. Retele neurale definesc o functie nonliniara care gaseste o 
corespondenta intre un set multivariat de date de intrare \(x\) si un set multivariat de 
date de iesire \(y\), modificand un set de parametrii $ \phi $, $ f(x, \phi) = y $. Aceasta este
alcatuita dintr-un numar enorm de elemente de procesare care contin parametrii functiei $ \phi $,
conectate intre ele intr-o structura de tip graf si dispuse pe straturi. Cele mai importante fiind: stratul de intrare
si de iesire, unde se stabileste forma generala pe care trebuie o sa respecte datele care vor
parcurge reteaua si modul in care va arata rezultatul obtinut. Celelalte nivele sunt denumite 
straturi ascunse. Aceastea fac prelucrarea informatiei primite de la straturile anterioare 
si o transmit mai departe. Spunem ca o retea neurala invata din datele primite, daca isi 
modifica parametrii $ \phi $ astfel incat sa reprezinte cu mai multa acuratete corespondenta
intre datele de intrare \(x\) si cele de iesire \(y\).
FastDepth este o arhitectura de retea neurala folosita pentru estima adancimii in imagini. 
Aceasta primeste o imagine de tip RGB al interiorului unei incaperi si returneaza o matrice
cu valori in intervalul $0m - 10m$ estimand pentru fiecare pixel in parte distanta de la
planul de proiectie al imaginii pana la punctul din spatiu surprins de fotografie. Scopul nostru
este antrenarea unei retele neurale care sa produca o matrice de adancime cu valori cat mai 
apropiate de distanta reala la care se afla obiectele fata de camera. In ciuda 
faptului ca algoritmul de tip ORB-SLAM2 functioneaza fara a folosi orice tehnica de Machine Learning. 
Consider ca adaugarea unor retele neurale care sa indeplineasca anumite sarcini clar definite:
cum ar fi extragerea feature-urilor sau pentru estimarea adancimii in imagini ar fi un pas in fata
pentru a creste eficienta algoritmului. Revenind la sarcina de estimare a adancimii, ORB-SLAM2
foloseste o camera tip RGBD / Stereo care descrie cu foarte mare acuratete distanta pana intr-un
anumit punct din spatiu, dar creeaza o matrice rara de valori, majoritatea avand valori de 0,
sugerand incapacitatea de a estima distanta in zonele respective. Un motiv pentru care retelele
neurale sunt o alternativa buna este ca ele vor avea o estimare pentru fiecare pixel din imagine.
In plus, reteaua FastDepth pare sa realizeze o pseudosegmentare a zonelor din imagine, reuseste
sa identifice conturul obiectelor si atribuie valori ale distantei asemanatoare pentru pixelii
ce apartin aceleasi entitati. \\
Un posibil dezavantaj al acestei arhitecturi este limitarea de 10m, fiind nepotrivit
de folosit afara, dar ideal pentru un spatiu inchis de mici dimensiuni. Cu toate acestea exista 2 
probleme pentru care o retea neurala s-ar putea sa nu fie optiunea potrivita: valorile aproximate 
au o acuratete mai slaba decat cele obtinute de camerele Stereo/RGBD iar viteza acestora de 
procesare a cadrelor nu este suficient de mare pentru a functiona in timp real.
Un motiv pentru care am ales arhitectura FastDepth este ca rezolva in totalitatea problema vitezei procesand 
aproximativ 130 de cadre pe secunda.\@ De asemenea consuma o cantitate redusa de memorie, totalitatea 
parametrilor folositi in antrenare ocupand pana in 40 de MB, fiind usor de integrat intr-un 
dispozitiv embedded. \\ Exista mai multe filozii cand vine vorba de modul in care ar trebui sa
arate arhitectura unei astfel de retele neurale si operatiile pe care ar trebui sa le realizeze
fiecare strat in parte. Feed forward neural network a fost printre primele arhitecturi definite. 
Elementele de procesare sunt dispuse pe straturi, si fiecare strat primeste input-ul de la stratul 
precedent si transmitea output-ul la stratul imediat urmator. Informatia circula liniar, de la 
intrarea in retea pana la finalul acesteia. Abordarea s-a dovedit eficienta in situatiile 
in care era nevoie de retele neurale de mici dimensiuni, cu un numar redus de straturi si 
parametrii. In momentul in care crestea complexitatea, abordarea de feed forward neural network 
devenea greu de antrenat si dadea rezultate mai slabe, chiar daca din punct de vedere teoretic o
retea cu dimensiuni mai mari ar trebui sa fie capabila sa generalizeze mai bine.    
Pentru a rezolva aceasta problema au aparut arhitecturile de tip residual network. Acestea au 
aplicatii numeroase in clasificarea imaginilor, unde datele de intrare au dimensiuni mari
si este nevoie de multe nivele pentru a extrage suficiente informatii pentru a realiza 
clasificarea. Principiul de functionare este utilizarea unor straturi reziduale denumite si skip
connections, in care rezultatul unui strat este salvat si transmis ca data de intrare la un alt
nivel decat la cel imediat urmator. Abordarea aceasta pastreaza din informatiile initiale ale 
datelor de intrare in straturile viitoare stabilizand antrenarea. O alta arhitectura des 
intalnita este cea de encoder-decoder folosita in numeroase aplicatii practice in ceea ce 
priveste imaginile: sarcini de colorare a imaginilor gri, reconstructie a imaginilor care 
contin parti lipsa si de generare de imagini: un exemplu fiind Variational Auto Encoder.
FastDepth foloseste tehnica de skip connections, straturile finale primind ca date de intrare
valorile calculate de straturile aflate la inceput, pentru a pastra informatia initiala a 
imaginii si urmeaza o arhitectura encoder-decoder. Encoder-ul transforma datele de intrare
intr-o forma mai compacta asemeni unei operatii de arhivare, iar partea de decoder dubleaza
dimensiunea datelor pana cand le aduce la forma initiala si compune canalele printr-o operatie
liniara pentru a forma matricea de adancime. \\
Pe langa tipurile de arhitecturi propuse, exista mai multe categorii de straturi in retele neurale.
Primele folosite erau cele fully connected, in care fiecare element de procesare era conectat 
cu toate celelalte elemente de procesare din stratul urmator. Matematic, operatia poate fi vazuta 
ca o inmultire de matrici, o operatie costisitoare, iar utilizarea exclusiva a straturilor complet 
conectate creea o retea neurala incapabila sa reprezinte functii nonliniare, scazand capacitatea de
generalizare. De cele mai multe ori straturile liniare sunt folosite impreuna cu functii de
activare nonliniare precum ReLU sau Sigmoid dar in continuare ramane problema numarului mare 
de parametrii care trebuie antrenati. Din aceasta cauza au fost create straturile convolutionale
care  folosesc mai putini parametrii si au aplicabilitate in procesarea imaginilor. Principiul
teoretic pe care se bazeaza este ca pixelii alaturati in imagine au aceeasi semnificatie,
reprezentand acelasi feature. Operatia de convolutie trebuie realizata pe o zona a imaginii iar
modificarea parametrilor afecteaza output-ul generat de mai multi pixeli. Aceasta filozofie
diferita fata de cea a straturilor complet conectate, in care valoarea fiecarui parametru este
modificata individual. Pentru a realiza operatia de convolutie, se stabileste un kernel, o
matrice de  mici dimensiuni, in FastDepth folosindu-se kernel-uri de (3, 3), acestea vor stoca
parametrii pe care reteaua neurala ii va antrena pentru stratul convolutional \(w_{mn}\). In 
formula \(h_{ij}\) reprezinta intensitatea pixelului dupa calculul operatiei de convolutie, 
iar \(x_{ij}\) este valoarea intensitatii pixelului de pe coloana i si linia j.  
Litera \(a\) reprezinta aici functia de activare folosita iar \(\beta\) este o valoare numerica
denumita bias. Acesta poate fi modificat in procesul de antrenare al retelei neurale si creste
capacitatea de generalizare a functiei de convolutie.  
\begin{equation}
h_{ij} = a \left[ \beta + \sum_{m=1}^{3} \sum_{n=1}^{3} \omega_{mn} x_{i+m-2, j+n-2} \right]
\end{equation}
Stratul de convolutie este in continuare prea costisitor pentru a fi folosit de suficient de 
multe ori pentru a crea o arhitectura in timp real de mari dimensiuni. Presupunem ca avem 
un vector de intrare pentru un strat de convolutie \(x\) cu dimensiunile \([d, h, w]\) unde
\([h,w]\) reprezinta inaltimea si latimea vectorului, iar \(d\) reprezinta numarul de canale,
pentru o imagine de tip RGB, valoarea parametrului \(d_{in}\) este 3, fiind 3 canale de culoare pentru
rosu, verde si albastru. De asemenea presupunem ca avem un kernel de dimensiuni \([k, k, d_{in}, d_{out}]\),
unde \(d_{out}\) este numarul de canale care rezulta in urma operatiei de convolutie, atunci
numarul total de operatii va fi: $ h \cdot w \cdot d_{in} \cdot d_{out} \cdot k \cdot k$. Pentru 
a rezolva aceasta problema a fost creat un strat numit Depthwise Separable Convolutions, 
obtinut prin compunerea a 2 straturi de convolutie, unul numit depthwise convolution,
iar celalalt pointwise convolution. Aceasta abordare creste viteza de procesare si imbunatateste
acuratetea in sarcini de clasificare pentru seturi de date precum ImageNet ILSVRC2012. In cazul 
depthwise convolution, fiecare canal al datelor de intrare este procesat de un singur kernel
al stratulului de convolutie. Pointwise convolution uneste printr-o combinatie liniara 
rezultatul procesarii fiecarui canal. Complexitatea temporala obtinuta astfel este de:
$ h \cdot w \cdot d_{in} \cdot (k^2   + d_{out})$. \\ 
FastDepth foloseste o arhitectura encoder-decoder. Partea de encoder este reprezentata de o alta 
retea numita Mobile\_Net si ulterior Mobile\_Netv2 care reduce numarul de parametrii si creste 
viteza de procesare fara a impacta acuratetea. Partea de decoder este alcatuit din 5 straturi de 
tip depthwise convolution fiecare urmate de o interpolare liniara care dubleaza dimensiunea 
rezultatului, ultimul strat fiind un pointwise convolution care uneste canalele obtinute  
si returneaza matricea de adancime. Pentru Mobile\_Netv2 exista parametrii preantrenati in libraria 
Pytorch pe setul de date ImageNet fiind un motiv in plus de a folosi aceasta arhitectura 
in dezvoltarea FastDepth. Mobile\_Netv2 foloseste atat depthwise convolution cat 
si pointwise convolution intr-un strat numit Inverted Residual, acesta fiind alcatuit din 
urmatoarele componente unde $ t $ este factorul de multiplicare al numarului de canale,
$ s $ este parametrul de stride, determina daca se micsoreaza numarul de canale, $ h, w, d $
sunt dimensiunile matricei de intrare.   \\
\textbf{DE INSERAT AICI IMAGINEA}

% continuare detaliere mobileNet_v2
% detaliere fastDEpth
% adaugare poze


\chapter{Detalii de implementare}
Implementarea este realizata in C++17. Pentru management-ul librariilor si al codului folosesc
CMake 3.28.3. Acesta imi permite sa grupez in foldere codul scris de mine si face operatia de 
link-are automat cu binarele librariilor pe care le folosesc. Alte tehnologii pe care le 
mai folosesc sunt OpenCV 4.9.0, Ceres 2.2.0, Eigen 3.4.0, DBoW2 si ultima versiune de Sophus
pana la data de ianuarie 2025. In comparatie cu alte librarii care inca mai trec 
prin diverse update-uri, Sophus a intrat intr-o etapa de mentenanta, dezvoltarea efectiva a 
acestuia fiind finalizata din iunie 2024. O prima problema pe care am intalnit-o a fost gasirea 
unei versiuni compatibile de C++ cu toate aceste pachete. Am incercat mai multe variante printre 
care C++11, C++14, C++17 si C++20. Preferinta mea ar fi fost sa folosesc o versiune cat mai noua
cu putinta, dar care sa poata fi compatibila cu toate librariile mentionate. C++11 si C++14 nu 
erau compatibile cu Ceres, libraria fiind mult prea complexa pentru a intra si face modificari in
codul sursa. C++20 nu era compatibil cu Sophus si cu Eigen, iar ambele librarii sunt fundamentale 
deoarece implementeaza metode puternic optimizate de a lucra cu matrici iar API-ul lor era mai 
simplu decat cel OpenCV.\@ Singura optiune ramasa a fost C++17 care era incompatibila cu versiunea 
de DBoW2, aceasta folosea o versiune mai veche a functiei throw pentru erori. Prin modificarea
modului in care functiile returnau erorile, metoda de biblioteca assert si a 
utilizarea codurilor de eroare, am eliminat complet cazurile de utilizare pentru directiva de throw
si am putut recompila codul ca librarie. Voi incepe prin a descrie modul in care am utilizat 
fiecare din bibliotecile prezentate: \\
OpenCV este o librarie de computer vision. Contine implementari ale algoritmilor de extragere de
trasaturi precum FAST, ORB, SIFT, SURF, API-uri pentru procesare video: citirea unui video cadru
cu cadru, procesarea de imagini: aplicarea de filtre, transformarea in grayscale, eliminarea distorsiunii
cauzata de camera dar cele mai importante sunt structurile ce abstractizeaza matricile si parametrii 
prin care se indentifica trasaturile: cv::Mat si cv::KeyPoint. Structura KeyPoint este deosebit de
importanta pentru implementarea algoritmului deoarce stocheaza numeroase informatii despre zona
pe care o reprezinta: orientarea acesteia, coordonatele centrului si nivelul la care a fost observat, 
parametrii de care am avut nevoie in fiecare etapa de procesare a cadrelor. Pe langa aceste 
lucruri, OpenCV are un modul dedicat pentru citirea parametrilor retelelor neurale din fisierele care 
urmeaza un format de tip ONNX, fiind o alternativa potrivita daca vreau sa utilizez un model doar 
pentru sarcini de inferenta.  \\
Ca librarie de optimizare am avut de ales intre Ceres si g2o. In implementarea oficiala 
g2o era cel folosit. Motivul principal fiind ca permite abstractizarea parametrilor care 
trebuie optimizati si a relatiilor dintre acestia sub forma unui graf neorientat. 
API-ul de g2o permite activarea si dezactivarea anumitor noduri, pentru a face implementarea 
mai robusta impotriva outlier-elor si a putea reintroduce noduri eliminate temporar in graful 
de optimizare. Ceres din pacate nu permite acest lucru. O data create conditiile initiale acestea 
pot fi dezactivate si nu mai este permisa reutilizarea lor iar la finalizarea algoritmului 
memoria folosita de catre noduri este eliberata. Cu toate acestea, Ceres are un API mai
usor de utilizat si prin rulari repetate am observat ca este cu putin mai rapid decat g2 \\
Folosesc libraria Eigen deoarece este mai simplu API-ul de calcul cu matrici decat 
cel din OpenCV.\@ Pentru a accesa elementele unei matrici in OpenCV se foloseste o referinta la 
vectorul de date facand accesarea elementelor mult mai nesigura iar verificarea indicelui este 
facuta la runtime. In cazul matricilor din Eigen, accesarea elementelor si operatiile
cu matrici sunt verificate la compile time, prevenind astfel erorile inainte de a rula programul. \\
Sophus este o librarie care imi permite sa lucrez cu algebra de tip Lie. In loc de a vedea
estimarile pozitiei ca pe niste matrici de $ 4 \times 4 $, le pot vedea ca pe un vector alcatuit din 7 
elemente. Primii 4 parametrii alcatuind un quaternion, aceasta fiind o exprimare vectoriala 
a unei matrici $ 3 \times 3 $ de rotatie, iar ultimii 3 parametrii reprezentand un vector de translatie.
Biblioteca implementeaza operatii care imi permit sa lucrez cu acesti vectori, care fac parte 
dintr-un grup numit \(se(3)\) si garanteaza ca rezultatul obtinut este scalat corespunzator 
pentru a face parte in continuare din aceeasi categorie. \\

DBoW2 este o metoda de tip bag of words pentru compararea imaginilor intre ele. Este utilizat
pentru operatii precum feature matching intre imagini consecutive, relocalizari si recunoastera
zonelor prin care a trecut pentru a inchide buclele create de mai multe cadre cheie salvate
in harta. Acesta este alcatuit dintr-o structura de tip arbore. Fiecare nivel fiind obtinut
din realizarea unui algoritm de clusterizare a descriptorilor de tip ORB ca de exemplu 
kmeans++, separarea tuturor descriptorilor in functie de centroizi si reluarea aceleasi
operatii in fiecare dintre clusterele nou create. Nodurile de tip frunza sunt alcatuite 
dintr-un singur descriptor. Construirea arborelui se realizeaza intr-o etapa offline, in cazul
librariei DBoW2, setul de date folosit a fost Bovisa 2008{-}09{-}01. Au fost alese 10K imagini
iar pentru fiecare cadru in parte extrase 1000 de descriptori ORB.\@ Acestea au fost folosite 
pentru a crea un arbore de adancimea 6 iar numarul de clustere pe care le creeaza fiecare 
iteratie a algoritmului kmeans++ este de 10. Pe ultimul strat exista un milion de frunze, si 
tot aceeasi lungime o va avea si vectorul de feature-uri care va reprezenta o imagine. Fiecare
descriptor va primi o valoare numerica numita greutate, invers proportionala cu frecventa pe 
care o are acesta. Cu cat este mai rar un anumit descriptor,cu atat este mai util pentru a 
diferenta o imagine de multe altele. Scopul principal al librariei este sa primeasca ca data
de intrare descriptorii ORB ai unei imagini si sa calculeze vectorul sau bag-of-words.
Vectorul bag-of-words este alcatuit in principal din valori de 0. Fiecare descriptor al 
imaginii parcurge arborele de la radacina spre frunze, parcurgerea realizandu-se prin 
calcularea distantei Hamming dintre descriptor si toate nodurile de pe un anumit nivel, 
si alegerea nodului cu distanta Hamming minima. Nodul frunza la care va ajunge va avea asociat
un index, in cazul de fata cu valori de la 0 la un milion. La acelasi index va fi modificata
valoarea din vectorul bag-of-words in valoarea greutatii descriptorului stocat in arbore.
Apelul de biblioteca returneaza de asemenea un vector de feature-uri in care fiecare element
este o pereche de forma $ (int, vector<descriptori>) $, primul element este indexul 
clusterului de la nivelul 4 al arborelui DBOW2, iar cel de-al doilea element reprezinta un 
vector de descriptori din imaginea curenta care se potrivesc in acelasi cluster, au distante 
Hamming aproape de 0 intre ei. Doua imagini pot fi comparate intre ele prin intermediul acestui 
vector de feature-uri, lucru care va fi detaliat in descriera clasei OrbMatcher.  
In implementarea ORB-SLAM2, nu este practica creearea unui vector de tip bag of words 
cu un milion de elemente, mai ales ca majoritatea valorilor sunt 0, asa ca o reducere 
a dimensionalitatii vectorului ar creste viteza de calcul a sistemului. Din aceasta cauza,
compararea descriptorilor se realizeaza doar pana la nivelul 4 in arbore, vectorul bow avand 
doar 1000 de elemente, iar cel de feature-uri avand acelasi numar de elemente cu numarul de 
descriptori. 

Structura de fisiere este una simpla, in folderul radacina se regaseste fisierul de 
CmakeLists.txt care va fi interpretat de utilitarul cmake pentru a genera automat Makefile-ul. 
Acest Makefile va contine regulile de build si de clean pentru proiectul meu. Am fisierul de 
main.cpp unde vor fi initializate componentele si se va selecta pe care dintre cele 2 seturi
de date se va aplica algoritmul. Tot aici se regaseste si fisierul fast\_depth.onnx, in este 
stocata arhitectura si parametrii antrenati ai retelei neurale FastDepth pentru estimarea 
adancimii. In plus am fisiele de include unde se afla antetele claselor pe care le voi 
implementa si fisierul de src unde se afla codul de C++ si logica programului. Am observat ca 
separarea codului in acest fel este o practica des intalnita in proiectele de mari dimensiuni 
si garanteaza flexibilitate in includirea dependintelor intre fisiere. Algoritmul ORB-SLAM2 \
este unul complex, depinzand de o multitudine de parametrii care pot influenta acuratetea.
Cei mai importanti sunt cei corelati cu camera. In fisierul config.yaml se regaseste matricea \(K\), 
parametrii de distorsiune ai imaginii si alte constante pe care le-am considerat ca fiind niste
hiperparametrii ai algoritmului, care vor trebui modificati in functie de mediul in care va rula
ORB-SLAM2 pentru a garanta functionarea corecta. \\

In main.cpp se face citirea fisierului ORBvoc.txt, acesta contine datele pe care le va folosi
clasa ORBVocabulary pentru a calcula vectorii de KeyPoint-uri pentru fiecare dintre cadrele cheie.
Acesti vectori de KeyPoint-uri vor fi comparati intre ei pentru a determina daca pozele respective 
provin din acelasi loc pentru a face corelatii intre cadre, relocalizari sau operatii de loop
closure. Tot in main.cpp se va face selectia pentru unul din cele 2 seturi de date pe care le va
folosi algoritmul in rularea lui. Aceste seturi de date contin de fapt cadrele dintr-un video 
facut cu o camera RGBD Microsoft Kinetic impreuna cu matricile de adancime si pozitiile acestora
in spatiu pentru fiecare cadru in parte. Aceste seturi de date sunt suficient de complexe pentru 
a-mi permite evaluarea functionarii algoritmului de ORB-SLAM2. \\

Clasa SLAM initializeaza componentele principale ale algoritmului si monitorizeaza durata 
fiecarei operatii pentru debugging. Implementarea se bazeaza pe compunerea de clase, fiecare 
fiind responsabila cu realizarea unei anumite sarcini. Voi prezenta clasele folosite plecand 
de la cele mai elementare catre cele mai complexe. \\

Clasa TumDatasetReader este responsabila de achizitia de date si de scrierea 
in fisier a traiectoriei pe care o estimeaza algoritmul cadru cu cadru. Achizitia de date 
presupune citirea din memorie a matricei RGB, convertirea acesteia in grayscale pentru 
o procesare mai rapida de catre algoritmul ORB si de obtinerea hartii de adancime pentru 
cadrul respectiv. Acest lucru poate fi realizat in 2 feluri: matricea de distante este citita
din setul de date TUM RGBD si a fost inregistrata cu o camera RGBD tip Microsoft Kinetic,
sau se foloseste reteaua neurala FastDepth prin care se estimeaza in timp real distanta pentru
fiecare pixel din cadrul curent. Ambele structuri, matricea cadrului curent si harta de adancime
vor fi transmise ca parametrii clasei Tracker. Tot TumDatasetReader stocheaza estimarile 
pozitiilor camerei pentru fiecare cadru in parte. Cadrele cheie, cele salvate 
in clasa Map, vor avea matricea de pozitie stocata nealterat in memorie, ele sunt deja relative 
fata de primul cadru citit. Pentru celelalte cadre, matricea de pozitie salvata in clasa
TumDatasetReader este relativa la un cadru cheie, de preferat ultimul cadru cheie creat pana
la citirea imaginii curente. Motivul pentru care se realizeaza salvarea pozitiilor in acest 
fel, este ca doar cadrele cheie sunt cele care isi pot modifica pozitia datorita operatiilor 
de optimizare precum Bundle Adjustment asa ca doar acestea ar trebui sa aiba valoarea lor 
salvata explicit. \\

Clasa MapPoint este fundamentala pentru buna functionare a algoritmului ORB-SLAM2. Aceasta este 
formata cu ajutorul unui KeyPoint si al unui KeyFrame asociat acestuia. In etapa anterioara, 
am prezentat modul in care se face proiectia coordonatelor unui punct cheie in spatiu, acestea 
devenind coordonatele globale ale MapPoint-ului pec are il creem. Punctului din spatiu 
i se asociaza de asemenea descriptorul keyPoint-ului care l-a creat, pentru compararea ulterioara 
cu alte KeyPoint-uri din alte imagini. Un MapPoint are nevoie de un vector de orientare, acesta 
ajuta in verificarea proprietatii unui MapPoint de a fi sau nu vizibil dintr-un KeyFrame. Pentru a 
calcula acest vector de orientare prima data se determina coordonatele globale ale centrului camerei
pentru cadru cheie care a creat KeyPoint-ul, acest lucru se realizeaza in felul urmator:
\begin{equation}
    \begin{bmatrix}
        X_{c} \\
        Y_{c} \\
        Z_{c} \\
        \end{bmatrix} = -R_{wc}^t * t_{wc}, \quad{}  
        T_{wc} =     
     \begin{bmatrix}
            R_{wc} & t_{wc} \\
            0 & 1
        \end{bmatrix}
\end{equation} 
Normalizarea diferentei intre coordonatele globale ale centrului camerei si ale MapPoint-ului creeaza
vectorul de orientare. Acesta poate fi modificat, daca se constata ca mai multe cadre observa acelasi
punct. In situatia respectiva, vectorul de orientare final va fi media aritmetica a celorlalti vectori 
de orientare individuali.

Clasa Feature, aceasta componenta nu exista in implementarea oficiala a ORB-SLAM, dar am considerat 
ca utilizarea acesteia ar simplifica codul. Extinde clasa KeyPoint fara a o mosteni explicit,
are asociata distanta, extrasa din matricea de adancime, descriptorul
si o valoare de tip boolean care setata pe true, inseamna ca punctul este monocular altfel inseamna
ca este stereo. Aceasta clasificare se obtine prin compararea adancimii cu o valoare foarte apropiata
de 0, de exemplu  $ 1e-1 $. Daca distantei este mai mica decat 0.1, consider ca valoarea estimata de 
camera RGBD ori de reteaua neurala nu este corecta si ca punctul respectiv este monocular,
altfel voi considera ca este un punct de tip stereo. Pentru fiecare KeyPoint extras, se va 
crea un astfel de Feature, aceasta fiind clasa care va fi stocata direct in KeyFrame. Fiecare
 Feature are setat pe null la initilializare o referinta la un obiect 
de tip MapPoint. Pentru a garanta functionarea in real time a algoritmilor, asocierea (KeyPoint,
 MapPoint) trebuie sa poata fi accesata in $ O(1) $. Vectorul de elemente
de tip Feature, impreuna cu o structura de tip dictionar unde cheia va fi MapPoint si valoarea 
un Feature, vor face acest lucru posibil. Singura problema este ca cele 2 structuri incearca sa 
reprezinte aceleasi corelatii, in cazul vectorului am indicele unui Feature drept cheie si incerc
sa accesez MapPoint-ul asociat, iar in cazul dictionarului, am referinta unui MapPoint si incerc sa
obtin adresa Feature-ului. Ambele structuri trebuie sa contina aceleasi perechi, altfel 
comportamentul algoritmului devine nedefinit. \\

Clasa KeyFrame, aceasta contine mai multe elemente legate de cadrul curent. Pentru a mentine 
functionarea sistemului in timp real, trebuie sa stocam in memorie rezultatele calculelor noastre.
Din aceasta cauza, in aceasta clasa se vor regasi matricea de adancime, vectorul de Feature-uri,
vectorul de feature-uri calculat de metoda bag-of-words implementata in DBOW2 si cadrul initial, 
convertit in format grayscale ce va fi folosit ulterior pentru afisarea in timp real a performantelor
algoritmului. In interiorul constructorului acestei clase sunt mai multe operatii realizate,
majoritatea necesare pentru a creste eficienta accesarii datelor. De exemplu: vectorul de Feature-uri
in medie contine 1000 de elemente care nu sunt sortate. In situatia in care proiectam un MapPoint in 
plan ar trebui sa comparam coordonatele proiectiei cu pozitia fiecarui Feature in parte pentru a 
stabili care este cel mai apropiat. O modalitate de a rezolva acest lucru este segmentarea suprafetei
in \(K\) zone, in cazul meu am ales $ K = 100$, fiecare reprezentand o portiune din imaginea initiala,
avand asociate referintele Feature-urilor care se gasesc pe suprafata respectiva. In acest fel,
in functie de zona in care este proiectat MapPoint-ul, vom stii ce Feature are o posibilitate
mare de a corespunde corect cu MapPoint-ul respectiv, reducand astfel numarul de comparatii. 
Considerand ca toate valorile de tip Feature sunt dispuse in mod egal pe suprafata imagini atunci
complexitatea devine, $ O(N / K) $ unde \(N\) reprezinta numarul de Feature-uri iar \(K\) reprezinta
numarul de zone in care a fost impartita imaginea. Constructorul este responsabil de initializarea
structurilor de tip Feature, partitionarea lor in functie de coordonatele in imagine, si de memorarea
estimarii curente a pozitie camerei si a centrului camerei in coordonate globale. Tot in aceasta 
clasa se regaseste structura de tip dictionar (MapPoint, Feature), care va fi adaptata pe tot 
parcursul algoritmului. Alta metoda importanta este: $ get\_vector\_keypoints\_after\_reprojection $.
Aceasta primeste ca date de intrare coordonatele proiectiei unui MapPoint, valoarea ferestrei 
de proiectie, si octava minima si maxima. Octavele reprezinta nivelul la care a fost observat 
un Keypoint in imagine si o estimare grosiera a distantei dintre camera si punctul din spatiu 
observat. Acesta poate sa aiba valori inte 0 si 7 inclusiv si ne spune de cate ori 
s-a facut resize la imagine pentru a surprinde o anumita trasatura in imagine. De exemplu: daca 
un Keypoint are valoarea octavei 0, inseamna ca algoritmul a detectat-o din imaginea nemodificata.
Daca ar fi 1, atunci dimensiunea imagini a fost redusa o singura data cu 0.8 din valoarea initiala 
si asa mai departe. Feature-ul cu care este asociat MapPoint-ul trebuie sa aiba valori ale octavei
apropiate intre ele. Daca aceasta situatie nu s-ar respecta, ar introduce erori de estimare 
a distantei, ne asteptam ca Feature-uri care au corelatie puternica intre ele, sa faca parte 
din aproximativ acelasi loc, daca estimarea distantei in spatiu ar avea o diferenta prea mare intre 
cele 2 ar putea inseamna ca cele 2 puncte din spatiu sunt diferite, dar mediul are o structura 
simetrica, de exemplu: o sala de clasa cu bancile aliniate una in fata celeilalte, algoritmul 
ar putea observa 2 colturi ce apartin de 2 mese diferite, daca nu ar avea aceasta separare 
pe baza octavei, urmatorul cadru care observa aceleasi mese ar putea sa asocieze eronat punctele 
intre ele, afectand estimarea pozitiei. Fereastra de proiectie reprezinta cat de departe poate 
sa fie Feature-ul de coordonatele punctului de proiectie pentru a fi considerat corect. In functie 
de dimensiunea ferestrei aceasta poate intersecta 1, 2 sau 4 subsectiuni din cele 100 in care 
este impartita imaginea. Problema cea mai mare pe care am avut-o cu clasele KeyFrame si MapPoint
era dependenta circulara. MapPoint-urile aveau nevoie de un KeyFrame si un Feature pentru a fi 
create si trebuia sa mentina o lista a KeyFrame-urilor care observa MapPoint-ul respectiv.
In cazul KeyFrame-ului, acesta trebuie sa pastreze referinte asupra tuturor MapPoint-urilor pe 
care le observa. Pentru a rezolva aceasta problema am folosit o clasa aditionala care face 
operatii cu cele 2 structuri si am folosit forward declaration.  \\  

Clasa Map implementeaza harta pe care o foloseste algoritmul ORB-SLAM2. Aceasta este responsabila
de stocarea corecta a KeyFrame-urilor, a MapPoint-urilor si rezolva problema dependentei circulare
a celor 2 clase. Aici am implementate metodele de adaugare/stergere a unui MapPoint dintr-un KeyFrame.
De asemenea, clasa de MapPoint continea referinte la toate KeyFrame-urile care o observa. Aceste 
referinte sunt adaugate / sterse de catre 2 metode care se regasesc aici. Clasa Map creaza o 
structura de tip graf ponderat neorientat, in care nodurile sunt reprezentate de KeyFrame-uri. 
Arcele arata daca exista mai mult de 15 puncte comune intre 2 KeyFrame-uri iar ponderea lor este 
determinata de numarul de MapPoint-uri comune. Tot aceasta clasa realizeaza operatii pe acest graf,
adauga/sterge noduri si face interogari pentru a afla vecinii directi sau cei pe nivel 2. Am ales 
sa implementez aceasta structura folosind $ std::unordered\_map $. Drept cheie va avea KeyFrame-ul 
curent iar valoarea returnata de structura de tip dictionar va fi un alt $ std::unordered\_map $, 
ce va contine toate celelalte KeyFrame-uri cu care este direct conectata dar si ponderea conexiunii.
In acest fel accesarea vecinilor de ordinul 1 va fi o operatie ce se poate realiza in timp constant.
Functia $ track\_local\_map $ este folosita de catre clasa Tracking. Aceasta primeste ca date 
de intrare cadrul curent, ultimul cadru cheie salvat. Nu returneaza nimic, doar incearca 
sa gaseasca cate un MapPoint pentru Feature-urile care inca nu au fost corelate cu un punct 
din spatiu. Aceasta operatie este costisitoare si functioneza in felul urmator:
\begin{enumerate}
    \item sunt cautate toate KeyFrame-urile vecine de gradul 1 si 2 cu ultimul KeyFrame adaugat
    \item din aceste KeyFrame-uri sunt extrase toate MapPoint-urile observate de catre ele 
    \item toate MapPoint-urile sunt proiectate si sunt cautate potriviri pentru Feature-urile
care inca nu au MapPoint-uri asociate.      
\end{enumerate}
Pentru a nu fi necesar sa calculam de fiecare data KeyFrame-urile vecine si totalitatea 
harta locala de MapPoint-uri, le stochez ca variabile in interiorul clasei Map. Acestea vor 
fi modificate in momentul in care un KeyFrame este adaugat in harta. Intr-un caz ideal, ar 
trebui ca pentru fiecare Feature adaugat sa se gaseasca un MapPoint, dar acest lucru rareori 
se intampla. In situatia in care s-au gasit mai putin de 30 de puncte din spatiu care s-au 
proiectat corect in imagine, se considera ca a aparut o eroare de urmarire si se returneaza 
o eroare care forteaza algoritmul sa inceapa o etapa de relocalizare. \\

Clasa OrbMatcher este responsabila de realizarea urmarii feature-urilor asemanatoare intre 
cadre consecutive. Inainte de a incepe prezentarea metode implementate, voi descrie
pipeline-ul de procesare al unui punct din spatiu pentru a fi considerat observabil de 
catre camera. Avem o instanta a obiectului MapPoint $ mp $, daca una dintre operatiile
prezentate esuaeaza, punctul respectiv este ignorat de catre KeyFrame-ul curent.
\begin{enumerate}
    \item Se proiecteaza coordonatele globale ale $ mp $ in planul imaginii folosind matricea 
de estimare a pozitiei $ T_{cw} $ si matricea parametrilor camerei $ K $. Se verifica daca 
coordonatele proiectiei sunt valide pentru imagine.     
    \item Se calculeaza distanta $ d $ de la centrul camerei la $ mp $. In functie de valoarea 
octavei stocata in acest MapPoint, se pot estima o limita minima si maxima pentru $ d $. 
Daca valoarea obtinuta nu se incadreaza in acest interval se considera ca punctul este invalid.
    \item Cu ajutorul geometriei analitice se obtine ecuatia dreptei care uneste $ mp $ 
si centrulul camerei. Aceasta dreapta si vectorul de directie al MapPoint-ului,
trebuie sa creeze un unghi cu o valoare mai mare de 60 de grade pentru a fi considerat $ mp $
observabil in cadrul curent.
\end{enumerate} 
Daca aceste 3 verificari au fost realizate cu succes se considera ca punctul poate fi observat
de catre camera. \\
Exista 2 functii responsabile de asocierile intre cadrele curente, scopul acestora 
este ca gaseasca corespondente intre Feature-urile din cadrul curent si MapPoint-urile 
din spatiu. Pentru a se gasi perechea Feature $ f $ si MapPoint $ mp $, trebuie ca
$ mp $ sa se proiecteze in vecinatatea $ f $ iar descriptorii asociati atat Feature-ului 
cat si al MapPoint-ului sa aiba distanta Hamming sub un prag, setat in aceasta implementare la 50.     
O metoda ar fi compararea tuturor Feature-urilor din spatiu, cu totalitatea MapPoint-urilor 
observate de cadrul curent. Dar aceasta metoda ar fi ineficienta. O alta abordare ar fi separarea
Feature-urilor in clustere in functie de distanta Hamming a descriptorilor, abordare stabila dar 
lenta si preferabil de utilizat cand nu ne putem baza pe estimarea matricei de pozitie a cadrului anterior. 
Iar cealalta abordare o reprezinta clusterizarea un functie de coordonatele in imagine 
ale Feature-ului. \\ 
Functie $ match\_frame\_reference\_frame $ implementeaza prima metoda. Aceasta primeste ca 
parametru 2 vectori de feature-uri cal3culati de libraria DBoW2, unul asociat cadrului curent, pentru     
care estimam matricea de pozitie si unul asociat cadrului anterior, pentru care cunoastem deja matricea de pozitie
si asocierile de tip (Feature, MapPoint). Elementele acestor vectori sunt de tip $ (int, vector<descriptori>) $.
Daca 2 astfel de perechi au prima valoare egala intre ele, inseamna ca cei 2 vectori de descriptori fac parte din 
acelasi cluster, conform arborelui din libraria DBOW2. Fiecare descriptor are asociata o instanta a clasei 
Feature. In cadrul anterior, instanta poate avea sau nu, un MapPoint corespondent. Daca exista acel
MapPoint se poate proiecta in imagine, descriptorul intern al MapPoint-ului este comparat cu ceilalti 
descriptori din acelasi cluster din cadrul curent si se aplica testul de proportionalitate Lowe pentru a
garanta ca descriptorul cu distanta Hamming minima este cel mai bun.   
Functia $ match\_consecutive\_frames $ este mai simpla si implementeaza a doua metoda. MapPoint-ul din 
spatiu este proiectat in imagine, si toate Feature-urile dintr-o zona circulara de raza de variabila. 
sunt considerati posibili candidati pentru a crea o asociere (Feature, MapPoint). Se calculeaza distanta 
Hamming intre descriptorul MapPoint-ului si cel al Feature-ului. Iar descriptorul cu distanta minima si 
mai mica decat un prag setat la 100 este considerat ca fiind cel mai potrivit. Feature-ul asociat acelui
descriptor, va pastra o referinta a MapPoint-ului. \\

Clasa MotionOnlyBA implementeaza in Ceres algoritmul Motion Only Bundle Adjustment,
primeste ca date de intrare KeyFrame-ul curent si returneaza matricea de pozitie optimizata.
Libraria lucreaza cu o notiune din C++ numita functori. Acestea sunt clase/structuri pentru
care s-a facut overload la operatorul $ () $. Clasa BundleError se afla din aceeasi categorie
si implementeaza functia de eroare obtinuta din proiectarea unui MapPoint si asocierea 
acestuia cu un Feature. Pentru a crea problema de optimizare, clasa ceres::Problem trebuie 
sa stie care parametrii trebuie optimizati si functia de eroare pe care trebuie sa o minimizeze.
In cazul acestui algoritm, singurul lucru care va fi modificat este matricea de pozitie a KeyFrame-ului
pe care o voi converti in forma $ se(3) $, transformand-o intr-un vector de 7 elemente. Iar pentru 
functia de eroare, nu voi scrie explicit ca este suma erorilor de proiectie, voi initializa pentru
fiecare asociere de tip (Feature, MapPoint) cate un element al clasei BundleError. Algoritmul de 
optimizare implementat de libraria ceres, va incerca in mod independent sa reduca valoarea erorii 
pentru fiecare pereche in parte, modificand pe rand vectorul pozitiei. Exista un motiv pentru 
care schimb modul in care este exprimata pozitia camerei, matricea de pozitie contine 2 componente:
matricea de rotatie $ R $ si un vector de translatie $ t $. Pentru $ t $ nu exista restrictii 
de modificare atata timp cat aceasta nu aduce modificari mari intre pozitiile a doua cadre consecutive,
orice mod in care ar varia parametrii acestui vector, in continuare semnificatia
lui de vector de translatie ramane nealterata, in aceasta situatie putem spune ca parametrii 
sunt alterati de catre libraria Ceres folosind $ EuclidianManifold $, mici modificari bazate pe
calcularea derivatelor partiale ale acestora din functia de eroare definita in BundleError, asemanator 
modului in care sunt modificati parametrii in retele neurale. Pentru matricea de rotatie $ R $ nu se 
mai poate aplica aceeasi logica. Aceasta trebuie sa faca parte din structura de tip grup numit 
$ SO(3) $, adica sa respecte egalitatea $ R * R^t = R^t * R = I $ si trebuie sa reprezinte o rotatie 
reala pe cele 3 axe. Alterarea aleatorie a parametrilor ar duce la o matrice invalida. Din aceasta
cauza, modificarea rotatiei trebuie facuta cu un anumit unghi
iar acest lucru se poate realiza printr-o inmultire de 2 matrici de rotatie valide. Din pacate nu 
exista implementare in forma matriceala pentru schimbarea unghiului de rotatie, dar este pentru
Quaternioni. Din aceasta cauza fac conversia din matrice de pozitie in vector din 
categoria $ se(3) $, iar pentru primii 4 parametrii asociati rotatiei, optimizarea lor se realizeaza
folosind $ QuaternionManifold $. Aceasta abordare rezolva problema instabilitatii numerice si 
garanteaza ca rezultatul operatiei de optimizare este un element valid in $ se(3) $, ce poate fi 
ulterior convertit in forma matriceala. In functie de categoria din care face parte Feature-ul, 
acesta este considerat monocular sau stereo. Functia de eroare implementata de clasa BundleError 
este identica pentru ambele, cu exceptia ca pentru punctele stereo, este verificata si distanta 
la care se afla punctul fata de valoarea la care a fost estimata de camera RGBD.\@ Pentru a preveni
instabilitatea cauzata de ppunctele de tip outlier, functia Huber descrisa in capitolul anterior este
folosita in calcularea finala a erorii de proiectie. In implementara oficiala realizata de g2o, 
agloritmul de optimizare este rulat de 4 ori, si dupa fiecare executie sunt eliminate punctele de tip 
outlier. Experimental, am observat ca etapa de optimizare cadru cu cadru este cea mai costisitoare 
operatie pe care o realizeaza algoritmul de ORB-SLAM2, executia acesteia de 4 ori, nu creste 
semnificativ acuratea si reduce viteza de prelucrarea la aproximativ 5 cadre pe secunda, facandu-l 
nepotrivit pentru un sistem in timp real. Am observat ca obtin rezultate foarte bune, ruland o singura
data Motion Only Bundle Adjustment, urmat apoi de o etapa de eliminare a corelatiilor 
(Feature, MapPoint) de tip outlier. Daca mai putin de 3 asocieri raman, se considera ca algoritmul
a acumulat prea multe erori in urmarirea cadru cu cadru si trece intr-o stare de relocalizare. \\

Clasa Tracker realizeaza urmarirea traiectoriei cadru cu cadru. Aceasta integreaza 
fiecare dintre componentele definite anterior, si este responsabila de captarea cadrului curent,
transformarea acestuia in KeyFrame si luarea deciziei daca va fi salvat in Map pentru a completa 
harta mediului inconjurator. Pasii urmatori se executa pentru fiecare cadru in parte:
\begin{enumerate}
    \item Se creeaza KeyFrame-ul curent.
    \item Se estimeaza matricea de pozitie pe baza legii de miscare.
    \item Se realizeaza asocierea intre Feature-urile (puncte 2D) din cadru curent si 
MapPoint-urile observate de cadru anterior (puncte 3D)
    \item Pe baza asocierilor respective realizate anterior, se optimiza matricea de pozitie
a KeyFrame-ului curent, sunt eliminate asocierile de tip outlier
    \item este proiectata harta locala pe cadrul curent, si se gasesc noi asocieri 
(Feature, MapPoint), se executa din nou aceeasi operatie de optimizare Motion Only Bundle Adjustment
    \item Este evaluat KeyFrame-ul curent, se verifica daca trebuie salvat in clasa Map. 
\end{enumerate}
Cadrul curent si matricea de adancime sunt citite de TumDatasetReader. In imaginea RGB 
se foloseste ORB pentru a extrage un vector de KeyPoint-uri si un vector de descriptori. 
Acestea sunt folosite pentru a initializa un obiect de tip KeyFrame. Pentru algoritmul ORB se 
foloseste o versiune modificata implementata in clasa ORBextractor si este conceputa sa extraga 
aproximativ 1000 de puncte cheie, acestea fiind distribuite cat mai egal pe suprafata imaginii. 
Daca un numar foarte mare de keypoint-uri s-ar afla in aceeasi zona, acuratetea estimarii ar 
avea de suferit, pixelii din zonele aflate mai aproape de imagine se misca cu o viteza mai mare 
decat cei aflati in departare, daca am considera doar punctele dintr-o anumita in zona in realizarea
estimarii, am obtinute variatii in miscare prea bruste / lente depenzind de locul unde s-au gasit 
majoritatea punctelor.  Parametrii setati pentru algoritmul ORB sunt urmatorii: 1000 de feature-uri,
factorul de scalare al imaginii este 1.2, exista maxim 8 nivele, si algoritmul FAST care face 
extragerea initiala de KeyPoint-uri sa ia in considerare zona respectiva daca diferenta de intensitate
intre pixeli este de la 20 in sus. Daca in schimb, zona este slab texturata atunci poate sa seteze
aceasta diferenta la 7, pentru a garanta ca vor fi gasite feature-uri chiar si in cele mai 
dezavantajoase zone din imagine. De-a lungul duratei de viata a algoritmului, clasa 
Tracker pastreaza 4 referinte de tip KeyFrame: cadrul curent care este analizat, 2 cadre imediat 
anterioare care vor fi folosite la estimarea pozitiei si ultimul cadru referinta care a fost creat.
Cadrul referinta este ultimul KeyFrame adaugat in Map si indica aproximativ in ce 
zona se afla camera si care MapPoint-uri ar trebui sa fie vizibile. Cadrele mai vechi care nu au 
fost salvate in Map au fost sterse pentru a reduce cantitatea de memorie folosita. Pentru ultimul
KeyFrame creat urmeaza etapa de estimare a pozitiei curente, aceasta se face pe baza legii 
de miscare, iar valorile matricii vor fi calculate folosindu-ne de cele 2 cadre salvate in Tracker.
Important aici de observat ca pentru primul cadru citit, pozitia acestuia este matricea identitate
4*4, acest lucru sugerand ca dispozitivul care inregistreaza mediul considera ca primul KeyFrame
este chiar originea sistemului de coordonate, iar toate matricile de pozitie viitoare sunt de fapt
transformari relative fata de origine. Primul KeyFrame va fi salvat intotdeauna in clasa Map si 
este utilizat pentru a initializa primele puncte de tip MapPoint: pentru toate Feature-urile de 
tip stereo din imagine, se vor creea puncte in spatiu. Din cauza acestui mod de initializare, 
ORB-SLAM2 este sensibil pana la aparitia urmatorului cadru cheie, estimarile facute de acesta 
in prima etapa fiind predispuse la erori. Uneori algoritmul isi pierde orientarea cu totul, fiind
necesara o etapa de relocalizarea, sau de reluare a executiei acestuia. ORB-SLAM3, implementeaza 
o metoda mult mai robusta de initializare, generand mai multe harti locale in situatia in care
urmarirea cadru cu cadru esueaza si le uneste intre ele in momentul in care recunoaste o zona pe 
care a vizitat-o deja. Dupa ce a fost create KeyFrame-ul si a fost facuta estimarea initiala a 
pozitiei, clasa OrbMatcher este folosita pentru a gasit corelatii intre Feature-uri si MapPoint-uri.
Alegerea metodei care indeplineste acest lucru fiind determinata de numarul de KeyFrame-uri 
create de la ultima relocalizare sau de la adaugarea unui nou cadru cheie in Map. Daca nu se 
vor gasi minim 15 asocieri, se va considera ca algoritmul si-a pierdut orientarea, altfel, 
asocierile respective vor fi utlizate de catre MotionOnlyBA pentru a realiza optimizarea pozitiei.
Perechile de tip outlier vor fi eliminate si noua pozitie a KeyFrame-ului va fi returnata. Daca 
vor ramane mai putin de 3 asocieri se va considera, din nou, ca algoritmul si-a pierdut orientarea.
In final, se foloseste clasa Map pentru a proiecta toate punctele din harta locala pe cadrul curent
iar asocierile gasite vor trece din nou printr-un proces de optimizare. Daca nu se gasesc minim
50 de perechi (Feature, MapPoint) inseamna ca urmarirea cadrului curent a esuat. Altfel se 
trece la etapa urmatoare si se va decide daca vom stoca in Map KeyFrame-ul curent. Acest lucru 
se va intampla daca urmatoarele conditii vor avea loc simultan.    
\begin{enumerate}
    \item au trecut mai mult de 30 de cadre de la ultimul KeyFrame adaugat in Map
    \item numarul de MapPoint-uri in cadrul curent este 25\% din numarul urmarit de cadrul de referinta
    \item cadrul curent are cel putin 70 de Feature-uri de tip stereo, cu distanta dintre centrul
camerei si punct este mai mica de 3.2 metri si urmareste cel putin 100 de MapPoint-uri
\end{enumerate}

Clasa LocalMapping este responsabila de optimizarea hartii algoritmului. Aceasta sterge/adauga KeyFrame-uri
si MapPoint-uri, iar la fiecare cadru cheie nou, realizeaza operatia de Local Bundle 
Adjustment. Aceasta metoda optimizeaza matricile de pozitie si toate MapPoint-urile vecinilor 
directi si cei de categoria a doua pentru KeyFrame-ul abia adaugat. In momentul in care 
thread-ul de Tracking considera ca un nou cadru cheie trebuie de adaugat in harta, se executa
metoda principala $ local\_map $, aceasta indeplineste urmatoarele operatii:
\begin{enumerate}
    \item Creeaza noi MapPoint-uri din primele 100 de Feature-uri de tip stereo, 
sortate in ordine crescatoare dupa distanta la care se afla acestea de centrul camerei  
    \item Adauga cadrul curent in graful de KeyFrame-uri stabilind vecinii directi ai
acestuia 
    \item Noile MapPoint-uri create sunt adaugate intr-o lista numita $ recently\_added $,
pentru a iesi din aceasta lista, punctele trebuie sa treaca un test care dovedeste ca 
nu sunt rezultatul unui Feature eronat detectat de catre algoritmul ORB, si ca pot fi 
folosite cu incredere
    \item Se executa operatia de $ culling $, punctele sunt verificate daca sunt valide iar
daca nu, memoria lor este eliberata.
    \item Se foloseste operatia de triangulare pentru a crea noi MapPoint-uri din 
Feature-urile care se potriveau intre ele si faceau parte din cadre cheie diferite.
    \item Se detecteaza entitatile de tip MapPoint care reprezinta acelasi punct
din spatiu, iar una dintre referinte este stearsa pentru creste coorenta hartii si 
a creste ponderea conexiunii dintre KeyFrame-urile adiacente    
    \item Se executa operatia de KeyFrame culling, se verifica daca informatiile pe 
care le detine un KeyFrame, adica totalitatea valorilor de tip MapPoint pe care le 
detine, sunt observate si din alte cadre. Daca peste 90\% din punctele observate de un 
anumit cadru sunt vizibile si din alte cadre, KeyFrame-ul analizat este considerat redundant
si memoria lui este eliberata. Acest lucru garanteaza ca graful clasei Map, contine 
doar cadre esentiale pentru reprezentarea norului de puncte. 
\end{enumerate}
In etapa a 4-a se executa operatia de $ culling $, aceasta elimina punctele care nu sunt
de incredere. Singurele puncte care nu vor trece prin aceasta etapa de verificare sunt
cele generate de primul KeyFrame, tot primul KeyFrame nu poate fi sters deoarece 
ar da peste cap sistemul de coordonate local sub care lucreaza ORB-SLAM2. Un punct 
este considerat de incredere daca din momentul in care a fost creat, el a fost 
observat in 3 cadre cheie consecutive si daca a fost observat in cel putin 25\% din 
numarul total de cadre care au trecut de la creearea acestuia. Ambele conditii 
trebuie sa fie respectate simultan in momentul in care se face verificarea punctului
respectiv. Politica pe care o urmeaza familia de algoritmi ORB-SLAM este sa genereze 
multe puncte, fara a impune restrictii, pe care apoi le va supune acestui test de 
relevanta. \\

Ultima clasa este cea de MapDrawer pe care o folosesc pentru a afisa norul de MapPoint-uri,
cadrul curent analizat si pozitiile cadrelor cheie observate. Folosesc libraria Pangolin si 
OpenGL pentru desenarea fiecarei structuri, camera urmareste cadrul curent. Interfata grafica 
scade viteza de procesare a cadrelor dar este o modalitate eficienta de a intelege vizual ce 
se petrece in algoritm. Implementarea pentru interfata grafica am realizat-o spre final, cand 
aveam celelalte componente finalizate, lucru care a ingreunat procesul de dezvoltare deoarece 
lucram cu valori numerice in terminal. Acum daca as reincepe implementarea, interfata grafica ar fi
printre primele lucruri pe care le-as realiza. Datorita acestei clase am reusit sa gasesc erori in
modul de constructieal grafului ponderat din clasa Map si al modului in care proiectam punctele 
in spatiu. 

Pentru reteaua Neurala FastDepth pipeline-ul de antrenare a fost scris folosind libraria Pytorch
iar pentru operatiile de preprocesare folosesc libraria Albumentations. Setul de date pe care 
am facut antrenarea se numeste Nyu Depthv2 Dataset si l-am obtinut de pe Kaggle. Rezultatul 
acestui pipeline trebuie sa fie un fisier de tip ONNX cu valorile parametrilor retelei FastDepth 
in urma antrenarii pe setul de date. O problema pe care am observat-o la setul de date este ca 
pentru datele de antrenare adancimile sunt exprimate ca fiind in intervalul $ [0, 255] $, pe cand in setul 
de date de validare, acestea se afla intre $ [0, 10000] $ reprezentand valorile in milimetrii ale
distantelor. O limitarea acestui set de date este ca nu poate detecta distante mai mari de 10 metri.
Dar considerand ca algoritmul trebuie sa functioneze pentru incaperi de mici dimensiuni, consider ca 
aceasta distanta maxima nu ar trebui sa reprezinte o problema. Pentru antrenare am ales sa urmez lucrarea
stiintifica si am setat hiperparametrii:
\begin{itemize}
    \item Optimizatorul folosit a fost implementarea din Pytorch pentru Stochastic Gradient Descent, 
torch.SGD, avand un learning rate de $ 1e-3 $, o valoare a momentumului de $ \beta=0.9 $ si
$ weight\_decay=1e-4 $.
    \item antrenarea s-a realizat pentru 50 de epoci iar durata antrenarii a fost adunat de aproximativ 
6 ore jumatate. Statia pe care am antrenat era un Asus TUF Gaming A15, avand un procesor AMD Ryzen 7 cu o frecventa 
de 4.2 GHz si placa video NVIDIA GeForce RTX 2060, cu o memorie de 6GB.\@
    \item Imaginile in setul de date au o dimensiune de $ (3, 460, 640) $. Pentru a creste viteza de procesare
am modificat dimensiunile la $ (3, 256, 320) $ si am aplicat o functie de normalizare de tip min\_max. Ambele 
transformari sunt aplicate atat pe setul de date de antrenare cat si pe cel de test.
    \item un batch de date are dimensiune de 8
    \item In lucrarea FastDepth functia de pierdere folosita este L1Loss, aceasta fiind suma diferentelor dintre valoarea 
reala si cea determinata de reteaua neurala in modul. In implementarea mea am ales sa folosesc o functie de pierdere 
mai robusta conform acestei lucrari stiintifice (voi cita aici lucrarea) 
\end{itemize}
Acuratetea a fost verificata  prin compararea diferentei relative intre valorile obtinute prin inferenta si cele 
reale cu un factor $ RELATIVE\_ERROR=0.15 $. Aceasta operatie a fost realizata pentru fiecare pixel in parte, iar
acuratetea reprezinta procentul de pixeli cu o valoare care se incadreaza in limita impusa de RELATIVE\_ERROR.\@
Pentru a preveni antrenarea pentru intervale lungi fara a obtine rezultate, am avut 2 metode pe care le-am implementat:
o strategia de early stopping: in situatia in care valoarea acuratetii nu ar fi crescut pentru 5 epoci antrenarea 
ar fi fost oprita si o strategie pentru modificarea learning rate-ului in timpul antrenarii. Daca acuratetea nu crestea 
pentru 3 epoci valoarea parametrului sa fie redusa la 0.3 din valoarea initiala. In practica am observat ca reteaua 
converge aproape monoton catre o valoare optima. Functia de pierdere primeste ca date de intrare matricea de adancime 
obtinuta de catre reteau neurala si matricea cu valori reala din setul de date, denumita groundtruth, si returneaza o valoarea 
numerica de tip double care exprima cat de departe se afla estimarea noastra de realitate. Ideea antrenarii unei retele 
neurale este minimizarea acestei valori. Functia de eroare este alcatuita dintr-o combinatie liniara a 
3 componente diferite: L1Loss, GradientEdgeLoss si Structural Similarity Loss, formula matematica este:
\begin{equation}
    loss = 0.6 \cdot L1Loss + 0.2 \cdot GradientEdgeLoss + StructuralSimilarityLoss 
\end{equation}  
Structural Similary Loss se asigura ca media si distributia standard pe care o urmeaza valorile estimate, se apropie
de media si distributia standard a matricei groundtruth. In comparatie cu celelalte 2 componente ale functiei de 
pierdere care sunt aplicate la nivel de pixel, aceasta abstractizeaza rezultatele ca fiind 2 distributii Normale 
cu parametrii $ \mathcal{N}(\mu, \sigma^2) $ care trebuie sa se suprapuna.                     
Principiul de functionare pentru GradientEdgeLoss este ca pixeli din regiuni apropiate trebuie sa aiba cam aceleasi 
valori de estimare ale distantei si ca diferenta intre pixeli adiacenti pe axele x si y, ar trebui sa fie identica
cu cea din imaginea groundtruth. Aceasta poate fi scrisa in felul urmator, unde $ N $ reprezinta numarul de pixeli din 
imagine, iar derivata valorilor pixelilor in raport cu axa de coordonate reprezinta diferenta intre matricea imaginii
initiale si aceeasi matrice avand un rand shiftat la dreapta  pentru axa X notata $ \frac{\partial I}{\partial x} $
si un rand shiftat vertical pentru axa Y notata $ \frac{\partial I}{\partial y} $.  
\begin{equation}
    L_{\text{edges}} = \frac{1}{N} \sum_{i=1}^{N} \left( 
    \left| \frac{\partial I_{\text{pred}}}{\partial x} - \frac{\partial I_{\text{true}}}{\partial x} \right| + 
    \left| \frac{\partial I_{\text{pred}}}{\partial y} - \frac{\partial I_{\text{true}}}{\partial y} \right|
    \right)
\end{equation}

\chapter{Evaluare}
\section{Setul de date TUM RGBD Dataset}
Setul de date utilizat pentru a realiza evaluarea se numeste TUM RGBD Dataset. 
Acesta contine numeroase subseturi, fiecare verificand un aspect diferit al implementarii, 
ajutand la creearea unui imagini de ansamblu cu privire la robustetea algoritmului in functie 
de mediu in care se lucreaza si de traiectori pe care o urmeaza. Cele 2 subseturi pe 
care le-am considerat potrivite pentru implementarea sunt:
\begin{itemize}
    \item Subsetul rgbd\_dataset\_freiburg1\_xyz contine cadrele unui video de 35 de secunde, in care
traiectoria este in principal alcatuita din translatii, exista foarte putine rotatii fiind
ideal pentru a verifica daca estimarea pozitiei in spatiu este corect realizata.    
    \item  Subsetul rgbd\_dataset\_freiburg1\_rpy are 27 de secunde si contine 
foarte putine translatii. Exista in schimb numeroase schimbari bruste de rotatie care reduc
acuratetea imaginii captate, testand la maxim capacitatea algoritmului ORB de a extrage 
feature-uri. Sistemul isi schimba orientarea pe toate cele 3 axe, fiind unul dintre cele
mai dificile subseturi de date pe care se poate face antrenarea. Algoritmul 
ORB-SLAM2 este sensibil la operatiile de rotatie, mai ales atunci cand camera isi schimba 
orientarea catre o zona necunoscuta. Pentru a crea harta zonei respective sunt generate 
numeroase KeyFrame-uri si MapPoint-uri, pe care algoritmul trebuie sa le filtreze in clasa
de LocalMapping, lucru care creste complexitatea temporala si spatiala si scade acuratetea
sistemului.  
\end{itemize}   
Videourile sunt realizate 
cu ajutorul unei camere RGBD Microsoft Kinetic, avand frecventa de 30 de cadre pe secunda, 
setul de date contine imaginile de tip RGB, hartile de adancime pentru fiecare cadru in parte,
vectorii de pozitie in forma $ se(3) $, primii 3 parametrii fiind pozitia in spatiu 
$ (tx, ty, tz) $ iar urmatorii 4 parametrii sunt asociati matricei de rotatie, scrisa sub
forma de Quaternion, $ (qw, qx, qy, qz) $ si timestamp-urile asociate momentului in care 
au fost inregistrate fiecare din valorile din setul de date. Cu ajutorul acestor timestammp-uri
putem crea asocieri de tip (imagine RGB, matrice de adancime, pozitie) pe care le putem 
transmite algoritmul ORB-SLAM2. Pozitia este considerata ca fiind valoarea ideala, groundtruth,
si va fi comparata cu rezultatele obtinute.Clasa TumDatasetReader este responsabila de citirea
datelor si stocarea matricilor de pozitie obtinute pentru fiecare cadru. Dupa parcurgerea 
intregului set de date, valorile estimate sunt salvate intr-un fisier de tip text unde vor 
fi comparate cu cele reale.
\section{Metrici utilizate}
Algoritmul ORB-SLAM2 scrie intr-un fisier estimarile matricilor de pozitie pentru fiecare 
cadru in parte. Pentru a realiza comparatia cu datele de tip groundtruth din setul de 
date, folosesc un pachet din python numit $ evo $, acesta este capabil sa creeze un grafic
al traiectoriei, permitand astfel o reprezentare vizuala a rezultatelor si o separare a 
acestora in functie de ceea ce vreau sa evaluez: viteza, translatia sau orientarea. De
exemplu, figura de mai jos reprezinta variatia translatie pe fiecare dintre cele 3 axe. 
Cu albastru este valoarea de tip groundtruth iar cu galben este estimarea realizata de 
implementarea mea pentru algoritmul ORB-SLAM.\@ Rezultatele sunt obtinute pentru subsetul 
de date rgbd\_dataset\_freiburg1\_xyz, acesta fiind special conceput pentru a testa 
corectitudinea estimarii translatiei intre cadre.

\begin{figure}[htbp] 
  \centering
  \includegraphics[width=1.0\textwidth]{./images/variate_translatie.png}
  \caption{Graficul translatiei pe fiecare din cele 3 axe, groundtruth si estimare ORB-SLAM2}
  \label{fig:exemplu_imagine}
\end{figure}
Consider ca o reprezentare grafica in care traiectoria groundtruth se suprapune exact cu
ceeea ce a obtinut estimarea ORB-SLAM2 poate fi considerat o metrica pe baza careia sa 
punem considera ca rezultatele obtinute sunt satisfacatoare, daca este nevoie de exactitate 
se poate folosi: APE (Absolute Pose Error) pentru a evalua acuratetea traiectoriilor estimate
de un algoritm SLAM sau VIO, comparându-le cu traiectoria de referință (ground truth). Ea
măsoară distanța euclidiană dintre pozițiile estimate și cele reale, la fiecare moment de timp.
In general valorile scorurilor APE obtinute pentru ambele seturi de date sunt pana in 0.05,
sugerand ca acuratetea este buna. Alte metrici pe care le folosesc pentru implementarea 
mea a algoritmului ORB-SLAM2 este numarul de secunde necesar pentru parcurgerea setului de 
date sau numarul de cadre pe secunda. Numarul de KeyFrame-uri create, in momentul in care 
sistemul nu mai poate realiza urmarirea cadru cu cadru, acesta insereaza un nou KeyFrame,
cu cat sunt mai putine KeyFrame-uri noi adaugate, se poate considera ca traiectoria este 
usor de interpretat si ca sistemul este stabil. O alta metrica este legata de numarul 
de relocalizari pe care a trebuit sa le faca algoritmul pentru a parcurge setul de date.
Relocalizarea apare in situatia in care urmarirea cadru cu cadru esueaza, si este cautat 
KeyFrame-ul care seamana cel mai bine cu cadrul curent folosind vectorul de feature-uri 
calculat de metoda bag-of-words. Ideal, numarul necesar de relocalizari ar trebui sa fie 0.\\
In cazul rularii implementarii mele pe setul de date rgbd\_dataset\_freiburg1\_xyz acesta 
dureaza in medie 67 de secude, functionand la aproximativ 15 cadre pe secunda, in medie 
este nevoie intre 5{-}7 Keyframe-uri noi pentru parcurgerea setului de date. Logica 
de relocalizare nu este deloc folosita, algoritmul fiind capabil sa realizeze urmarirea
cadru cu cadru.
Pe setul de date rgbd\_dataset\_freiburg1\_rpy a durat 73 de secunde, videoul avand 
27 de secunde, reprezinta aproximativ 10{-}11 cadre pe secunde. Acest lucru se datoreaza 
numeroaselor operatii de optimizare a hartii pe care trebuie sa le faca algoritmul deoarece 
in medie sunt adaugate intre 17{-}19 KeyFrame-uri pentru a parcurge intreg setul de date.
In continuare, numarul de relocalizari este 0. Mai jos, am atasat graficul care compara 
estimarea orientarii pentru fiecare cadru, estimarile fiind descompuse dupa cele 3 dimensiuni
ale rotatiei. Graficul realizat de algoritmul ORB-SLAM2 pare sa fie shiftat in timp fata 
de cel real, dar sa aiba aproximativ aceeasi forma cu cel al valorilor de tip groundtruth.
Consider ca problema poate sa porneasca de la modul in care sunt atasate timestamp-urile 
pentru fiecare cadru in parte, lucru care nu are legatura directa cu modul in care este
realizata implementarea, ci cu modul in care setul de date creeaza perechile (imagine RGB, 
vector pozitie, matrice de adancime).      
\begin{figure}[htbp] 
  \centering
  \includegraphics[width=1.0\textwidth]{./images/rpy_variatie_orientare.png}
  \caption{Graficul orientarii pentru setul de date rgbd\_dataset\_freiburg1\_rpy}
  \label{fig:exemplu_imagine}
\end{figure}
Am incercat utilizarea retelei neurale FastDepth pentru a estima adancimea in loc de a
folosi matricea de distante a setului de date TUM RGBD, voi lua doar imaginea si vectorul 
de pozitie simuland astfel o situatie in care sistemul ar avea doar o camera RGB.
Problema cu aceasta implementare este ca reteaua neurala nu este suficient de exacta iar 
estimarile distantelor intre cadre consecutive in continuare variaza mult. Pentru 
subsetul de date rgbd\_dataset\_freiburg1\_xyz implementarea intra in etapa de relocalizare
in medie dupa primele 50{-}60 de cadre procesate. Sistemul este mult prea instabil pentru
a realiza in mod corect urmarirea cadru cu cadru. \\

In etapele intiale ale dezvoltarii algoritmului am incercat utilizarea unui video realizat 
folosind camera telefonului pentru testarea implementarii. Au fost 3 probleme pe care le-am 
intalnit: nu puteam determina parametrii corecti ai camerei telefonului. Aveam nevoie 
de distanta focala, de coordonatele centrului imaginii si de parametrii de distorsiune.
A doua problema o reprezenta lipsa unei matrici de adancime pentru fiecare cadru inregistrat
in parte iar cea de-a treia, era lipsa un vector de pozitie pentru fiecare imagine. Chiar 
si in situatia in care obtineam parametrii camerei telefonului folosind algoritmi implementati
in OpenCV pentru a-i determina, in continuare nu puteam fi sigur daca estimarile realizate 
de mine sunt cele corecte. Din cauza acestor multe probleme, am ajuns la concluzia ca un set 
de date ar fi o varianta mai potrivita.
\chapter{Concluzii}

\end{document}